\section{Fiabilidad y disponibilidad}

\subsection{Fiabilidad}

\begin{frame}[t]{Fiabilidad}
\begin{itemize}
  \item El tiempo de vida de un sistema se puede representar como una variable aleatoria
        $X$.

  \item La fiabilidad del sistema se define como una función $R(t)$
\begin{displaymath}
R(t) = P(X > t) : R(0) = 1 \quad y \quad R(\infty) = 0
\end{displaymath}

  \mode<presentation>{\vfill\pause}
  \item Obtenida del estudio de fallos de componentes.

  \mode<presentation>{\vfill\pause}
  \item \textmark{Fiabilidad}:
        Probabilidad de que un dispositivo funcione adecuadamente
        durante un periodo dado de tiempo 
        bajo condiciones de operación específicas.
        
\end{itemize}
\end{frame}

\begin{frame}[t]{Distribuciones de fiabilidad}
\begin{itemize}
  \item Ejemplos de distribuciones usadas para fiabilidad:
    \begin{itemize}
      \mode<presentation>{\vfill}
      \item \textgood{Exponencial}:
        \begin{itemize}
          \item Si la tasa de errores es constante 
                (generalmente correcto para componentes eléctricos)
                la fiabilidad sigue una distribución exponencial.
        \end{itemize}
      \mode<presentation>{\vfill}
      \item \textgood{Weibull}:
        \begin{itemize}
          \item Modela la distribución de fallos cuando la tasa de fallos es
                proporcional a una potencia del tiempo.
        \end{itemize}
    \end{itemize}
\end{itemize}
\end{frame}


\begin{frame}[t]{Sistemas serie}
\begin{itemize}
  \item Sea $R_i(t)$ la fiabilidad del componente \textmark{i}.
  \item El sistema falla cuando algún componente falla.
\end{itemize}

\begin{center}
\input{int/m2-01-trends/rel-serial.tkz}
\end{center}

\pause
\begin{itemize}
  \item Si los fallos son independientes, entonces:
\end{itemize}
\begin{equation*}
R(t) = \prod_{i=1}^{N} R_i(t)
\end{equation*}

\mode<presentation>{\pause}
\begin{itemize}
  \item La fiabilidad del sistema es menor:
\end{itemize}
\begin{equation*}
R(t) < R_i(t) \forall i
\end{equation*}
\end{frame}

\begin{frame}[t]{Sistema paralelo}
\begin{itemize}
  \item El sistema falla cuando todos los componentes fallan.
\end{itemize}

\begin{equation*}
R(t) = 1 - \prod_{i=1}^N Q_i(t) : Q_i(t) = 1 - R_i(t)
\end{equation*}

\begin{center}
\input{int/m2-01-trends/rel-parallel.tkz}
\end{center}
\end{frame}

\begin{frame}[t]{Ejemplo}
\begin{center}
\input{es/m2-01-trends/rel-ex-legend.tkz}
\end{center}

\begin{columns}[T]

\column{.5\textwidth}

\input{int/m2-01-trends/rel-serial-ex.tkz}

\begin{equation*}
R(t) = 0.9 \cdot 0.9 \cdot 0.9 = 0.729
\end{equation*}

\pause
\column{.5\textwidth}

\input{int/m2-01-trends/rel-parallel-ex.tkz}

\begin{equation*}
R(t) = 1 - (1 - 0.9)^3 = 0.999
\end{equation*}

\end{columns}

\end{frame}

\subsection{Disponibilidad}

\begin{frame}[t]{Disponibilidad}
\begin{itemize}
  \item En muchos casos, es más interesante conocer la disponibilidad.
  \item La disponibilidad de un sistema $A(t)$ se define como
        la probabilidad de que el sistema esté funcionando correctamente
        en el instante $t$.
    \begin{itemize}
      \item La fiabilidad considera el intervalo $[0,t]$.
      \item La disponibilidad considera un instante concreto en el tiempo.
    \end{itemize}
  \item Un sistema modelado como el siguiente diagrama de estados.
\end{itemize}
\begin{center}
\input{es/m2-01-trends/avail.tkz}
\end{center}
\end{frame}

\begin{frame}[t]{Medida de la disponibilidad}
\begin{itemize}
  \item \textgood{FIT} (\emph{Failures in Time}): 
        Fallos en un periodo (expresado en $10^9$ horas).
  \item \textgood{MTTF} (\emph{Mean Time To Failure}):
        Tiempo medio hasta fallo.
  \item \textgood{MTTR} (\emph{Mean Time To Repair}):
        Tiempo medio de reparación.
\end{itemize}

\begin{equation*}
FIT = \frac{10^9}{MTTF}
\quad\quad 
A = \frac{MTTF}{MTTF + MTTR}
\end{equation*}

\begin{itemize}
  \item ¿Qué significa una fiabilidad del 99\%?
    \begin{itemize}
      \item En 365 días, funciona correctamente $\frac{99 \cdot 365}{100} = 361.35$ días.
      \item Fuera de servicio $3.65$ días.
    \end{itemize}
\end{itemize}
\end{frame}

\begin{frame}{Tiempo anula sin servicio}
\begin{center}
{\small
\begin{tabular}{|l|l|}
\hline
Disponibilidad (\%) & Días sin servicio en un año\\
\hline
\hline
98\% & 7.3 días\\
\hline
99\% & 3.65 días\\
\hline
99.8\% & 17 horas y 30 minutos\\
\hline
99.9\% & 8 horas y 45 minutos\\
\hline
99.99\% & 52 minutos y 30 segundos\\
\hline
99.999\% & 5 minutos y 15 segundos\\
\hline
99.9999\% & 31.5 segundos\\
\hline
\end{tabular}
}
\end{center}
\end{frame}

\begin{frame}{Cálculo de disponibilidad}
\begin{itemize}
  \item Disponibilidad de elementos:
    \begin{itemize}
      \item HW: 99.99\%
      \item Disco: 99.9\%
      \item SO: 99.99\%
      \item Aplicación: 99.9\%
      \item Comunicaciones: 99.9\%
    \end{itemize}

  \mode<presentation>{\vfill\pause}
  \item Disponibilidad del sistema:
    \begin{itemize}
      \item Producto de disponibilidades elementales:
    \end{itemize}
\end{itemize}
\begin{equation*}
A(t) = \prod_{i=1}^{N} A_i(t) = 99.6804 \Rightarrow 1.17 \text{días sin servicio}
\end{equation*}
\end{frame}
