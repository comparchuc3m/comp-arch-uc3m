\section{Tendencias en tecnología}

\begin{frame}[t]{Impacto de la tecnología}
\begin{itemize}
  \item Una \textmark{Arquitectura de Juego de Instrucciones}
        (\emph{Instruction Set Architecture}) puede durar décadas
        mediante \textemph{evolución}.
    \begin{itemize}
      \item Necesidad de planificar para evolución a \textmark{largo plazo}.
      \item Aprender de la evolución de \textmark{tecnologías críticas}.
    \end{itemize}

  \mode<presentation>{\vfill}
  \item \textgood{Tecnologías críticas}:
    \begin{itemize}
      \item Lógica de circuitos integrados.
      \item DRAM de semiconductores.
      \item Flash de semiconductores.
      \item Discos magnéticos.
      \item Redes.
    \end{itemize}
\end{itemize}
\end{frame}

\begin{frame}[t]{Lógica de circuitos integrados}
\begin{itemize}
  \item Evolución histórica de tecnología de transistores:
    \begin{itemize}
      \item Densidad: +35\% anual.
      \item Incremento de tamaño de pastilla: 10\% - 20\% anual.
      \item Efecto combinado: 40\% - 55\% número de transistores por año.
      \item \textemph{Ley de Moore}.
    \end{itemize}

  \mode<presentation>{\vfill\pause}
  \item Evolución de la \textemph{Ley de Moore}
    \begin{itemize}
      \item 1965: Número de transistores por chip se duplica cada año.
      \item 1975: Número de transistores por chip se duplica cada dos años.
      \item \textbad{Ya no se cumple!}
        \begin{itemize}
          \item 2010: Procesador Intel: 1,170 millones de transistores.
          \item 2016 (predicción): 18,720 millones de transistores.
          \item 2016 (realidad): 1,750 millones de transistores.
        \end{itemize}
    \end{itemize}
\end{itemize}
\end{frame}

\begin{frame}[t]{DRAM de semiconductores}
  \begin{itemize}
    \item \textbad{Actividad}:
      \begin{enumerate}
        \item \emph{Lee} la \textmark{Sección 2.2} -- 
              Memory Technology Optimizations (páginas 84--90).
          \begin{itemize}
            \item Solamente SRAM, DRAM y RAMs Gráficas.
            \item \bibhennessy
          \end{itemize}

        \mode<presentation>{\vfill}
        \item \emph{Lee} \textgood{\url{https://en.wikipedia.org/wiki/DDR_SDRAM}}
          \begin{itemize}
            \item Mire en la tabla de generaciones de DDR SDRAM.
          \end{itemize}

        \mode<presentation>{\vfill}
        \item \textgood{Aspectos clave}:
          \begin{itemize}
            \item Evolución de tasa de reloj.
            \item Evolución de ancho de banda (MT/s y MB/s).
          \end{itemize}
      \end{enumerate}
  \end{itemize}
\end{frame}

\begin{frame}[t]{Memoria Flash}
  \begin{itemize}
    \item \textbad{Actividad}
      \begin{enumerate}
        \item \emph{Continúa leyendo} la \textmark{Sección 2.2} -- 
              Memory Technology Optimizations (páginas 92--93).
          \begin{itemize}
            \item Flash Memory and Phase-Change Memory Technology
            \item \bibhennessy
          \end{itemize}

        \mode<presentation>{\vfill}
        \item Trata de encontrar información sobre NVMe.
          \begin{itemize}
            \item ¿Cuál es la relación con la memoria flash?
            \item ¿Dónde se usan los NVMe?
          \end{itemize}
      \end{enumerate}
  \end{itemize}
\end{frame}

\begin{frame}[t]{Ancho de banda y latencia}
  \begin{itemize}
    \item \textgood{Ancho de banda} o \textgood{\emph{throughput}}.
      \begin{itemize}
        \item \textmark{Cantidad de trabajo} realizado \textmark{por unidad de tiempo}.
        \item \textemph{Procesadores}: Crecimiento entre 32,000 y 40,000 veces.
        \item \textemph{Memoria/discos}: Crecimiento entre 400 y 2,400 veces.
      \end{itemize}

    \mode<presentation>{\vfill\pause}
    \item \textgood{Latencia}:
      \begin{itemize}
        \item Tiempo entre \textgood{inicio} y \textgood{fin} de un evento.
        \item \textemph{Procesadores}: Mejora entre 50 y 90 veces.
        \item \textemph{Memory/disks}: Mejora entre 8 y 9 veces.
      \end{itemize}

    \mode<presentation>{\vfill\pause}
    \item Ancho de banda mejora mucho más rápido que la latencia.

    \mode<presentation>{\vfill\pause}
    \item \textbad{Nota}: Vea las figuras 1.9 (página 21) y 1.10 (página 22).
      \begin{itemize}
        \item \bibhennessy
      \end{itemize}
  \end{itemize}
\end{frame}
