\begin{acexercise}\end{acexercise}

Un procesador tipo RISC-V tiene un banco de registros de aritmética entera de 32 registros.
También dispone de otro banco de registros de coma flotante con otros 32 registros.

Se supone que se dispone de suficiente ancho de banda de captación y
decodificación como para que no se generen detenciones por esta causa y que se
puede iniciar la ejecución de una instrucción en cada ciclo, excepto en los
casos de detenciones debidas a dependencias de datos.

La siguiente tabla muestra las latencias adicionales de algunas categorías de
instrucciones, en caso de que haya dependencia de datos. Si no hay dependencia
de datos la latencia adicional no es aplicable.

\begin{table}[htb]
\begin{tabular}{|l|r|}

\hline
\textbf{Instrucción} &
\textbf{Latencia adicional}
\\
\hline

\asminst{fld} & +2\\
\hline

\asminst{fsd} & +2\\
\hline

\asminst{fadd.d} & +4\\
\hline

\asminst{fmul.d} & +6\\
\hline

\asminst{addi} & +0\\
\hline

\asminst{subi} & +0\\
\hline

\asminst{bnez} & +1\\
\hline

\end{tabular}

\end{table}

La instrucción \asminst{bnez} usa bifurcación retrasada con una ranura de
retraso (\emph{delay slot}).

Se desea ejecutar en esta máquina el siguiente fragmento de código:

\lstinputlisting[language=generalasm2]{int/ex/m4-02/unroll-2-ex.asm}

Inicialmente se tienen los siguientes valores en los registros:

\begin{itemize}
  \item \asmreg{t1}: \asmlabel{0x00100000}.
  \item \asmreg{t2}: \asmlabel{0x00140000}.
  \item \asmreg{t3}: \asmlabel{0x00180000}.
  \item \asmreg{t4}: \asmlabel{0x00000100}.
\end{itemize}

Responda a las siguientes cuestions:

\begin{enumerate}

\item ¿Cuántas iteraciones se ejecuta el bucle?

\item Identifique las dependencias RAW que pueda encontrar en el código
anterior.

\item Identifique todas las detenciones que se producen en el fragmento de código.
      Calcule el número de ciclos por iteración y el número total de ciclos
      para todas las iteraciones.

\item Modifique el bucle usando planificación de instrucciones.
      Calcule el número de ciclos por iteración y el número total de ciclos
      para todas las iteraciones.

\item Aplique un desenrrollamiento de bucle procesando dos elementos por
      iteración y usando además planificación de instrucciones.
      Calcule el número de ciclos por iteración y el número total de ciclos
      para todas las iteraciones.
      Determine el factor de aceleración.

\item Si se desease aplicar un factor de desenrrollamiento procesando cuatro
      elementos por iteración ¿Qué compliación podría encontrarse?

\end{enumerate}

\begin{acsolution}\end{acsolution}

\paragraph{Número de iteraciones}.
El bucle se repite hasta que el registro \asmreg{t4} alcanza el valor
\asmlabel{0}. Dicho registro se decrementa en una unidad en cada iteración.
El valor inicial de dicho registro es $2^8 = 256$. Por tanto, se ejecutan
un total de \asmlabel{256} iteraciones.

\paragraph{Dependencias RAW}.
En el fragmento original se identifican las siguientes 
dependencias:

\begin{enumerate}

\item \asmreg{ft0}: \asmlabel{I1} $\rightarrow$ \asmlabel{I3}.
\item \asmreg{ft1}: \asmlabel{I2} $\rightarrow$ \asmlabel{I3}.
\item \asmreg{ft2}: \asmlabel{I3} $\rightarrow$ \asmlabel{I4}.
\item \asmreg{ft2}: \asmlabel{I4} $\rightarrow$ \asmlabel{I5}.
\item \asmreg{t4}: \asmlabel{I8} $\rightarrow$ \asmlabel{I9}.
\end{enumerate}

\paragraph{Ejecución de fragmento original}.

\lstinputlisting[language=generalasm2]{int/ex/m4-02/unroll-2-stalls.asm}

En total se requieren 22 ciclos por iteración.

Para ejecutar 256 iteraciones se requieren:

\[
ciclos = 256 \times 22 = 5632
\]

\paragraph{Planificación de bucle}.
Si se reordenan las instrucciones se tiene:

\lstinputlisting[language=generalasm2]{int/ex/m4-02/unroll-2-sched.asm}

En total se requieren 19 ciclos.

Para ejecutar 256 iteraciones se requieren:

\[
ciclos = 256 \times 19 = 4864
\]
\[
S = \frac{5632}{4864} = 1.16
\]

\paragraph{Desenrrollamiento de bucle}

\lstinputlisting[language=generalasm2]{int/ex/m4-02/unroll-2-unrollx2.asm}

En total se requieren 21 ciclos por cada dos elementos

Para ejecutar las 128 iteraciones que se necesitan (2 elementos por iteración)
se requieren:

\[
ciclos = 128 \times 21 = 2688
\]
\[
S = \frac{5632}{2688} = 2.1
\]

\paragraph{Desenrrollamiento de bucle con factor 4}

En este caso se requerirían un mayor número de registros de coma flotante y
no habría registros suficientes con los registros temporales (\asmreg{ft}),
por lo que habría que guardar algún registro en la pila y posteriormente
restaurarlo.
