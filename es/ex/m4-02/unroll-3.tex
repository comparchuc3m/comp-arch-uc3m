\begin{acexercise}
\end{acexercise}

Dada la siguiente sección de código:

\lstinputlisting[language=generalasm2]{int/ex/m4-02/unroll-3-ex.asm}

Y teniendo en cuenta que se ejecuta en una máquina con las latencias
adicionales entre instrucciones indicada por la
tabla~\ref{tab:ex-may-2012:lat}.

\begin{table}[htbp]
\caption{Latencias adicionales}
\label{tab:ex-may-2012:lat}

\begin{tabular}{|p{.4\textwidth}|p{.4\textwidth}|p{.1\textwidth}|}

\hline
Instrucción que produce el resultado (previa) &
Instrucción que usa el resultado (posterior) &
Latencia
\\
\hline
\hline

Operación ALU FP & 
Operación ALU FP &
\multicolumn{1}{r|}{5}
\\
\hline

Operación ALU FP &
Cargar/almacenar double &
\multicolumn{1}{r|}{4}
\\
\hline

Operación ALU FP &
Instrucción de salto &
\multicolumn{1}{r|}{4}
\\
\hline

Cargar doble &
Operación ALU FP &
\multicolumn{1}{r|}{2}
\\
\hline

Cargar doble &
Cargar doble &
\multicolumn{1}{r|}{1}
\\
\hline

Almacenar doble	&
Operación ALU FP &
\multicolumn{1}{r|}{2}
\\
\hline

\end{tabular}

\end{table}

La instrucción de salto (\emph{branch}) tiene una latencia de un ciclo y no
tiene \emph{delay slot}. 

Suponga además que la máquina en la que ejecuta es capaz de emitir una
instrucción por ciclo, salvo las esperas debidas a detenciones y que es un
procesador segmentado con un único camino de datos (\emph{pipeline}). 

\begin{enumerate}

  \item Determine todas las dependencias de datos.

  \item Determine el número de ciclos total que se necesita para ejecutar la
sección de código completa. 

  \item Modifique el código para reducir las detenciones mediante la técnica de
planificación de bucle. Determine la aceleración obtenida respecto a la versión
sin planificar (apartado 2). 

  \item Modifique el código realizando un desenrollamiento de bucle con dos
iteraciones por bucle. Determine la aceleración obtenida respecto a la versión
sin planificar (apartado 2). 

\end{enumerate}


\begin{acsolution}
\end{acsolution}

\paragraph{Apartado 1}

Se producen las siguientes dependencias:

\begin{itemize}
\item \asmlabel{I4} $\rightarrow$ \asmlabel{I2} (\asmreg{f0}: RAW), 
      \asmlabel{I4} $\rightarrow$ \asmlabel{I3} (\asmreg{f2}: RAW)
\item \asmlabel{I5} $\rightarrow$ \asmlabel{I4} (\asmreg{f4}: RAW)
\item \asmlabel{I6} $\rightarrow$ \asmlabel{I1} (\asmreg{r1}: WAR), 
      \asmlabel{I6} $\rightarrow$ \asmlabel{I2} (\asmreg{r1}: WAR), 
      \asmlabel{I6} $\rightarrow$ \asmlabel{I5} (\asmreg{r1}: WAR)
\item \asmlabel{I7} $\rightarrow$ \asmlabel{I3} (\asmreg{r2}: WAR)
\item \asmlabel{I8} $\rightarrow$ \asmlabel{I6} (\asmreg{r1}: RAW), 
      \asmlabel{I8} $\rightarrow$ \asmlabel{I1} (\asmreg{r3}: WAR)
\end{itemize}


\paragraph{Apartado 2}

A continuación se identifican las detenciones (\emph{stalls}) en la ejecución:

\input{int/ex/m4-02/unroll-3-lst1.asm}

Se necesita un ciclo para el código de iniciación. Cada iteración necesita 13
ciclos. Como el bucle se ejecuta 5 veces, se tiene un total de:

\[
1+5 \cdot 13 = 66
\]

Debido a que no existe \emph{delay slot} no es necesario añadir un ciclo de
\emph{stall} después de la instrucción de salto (\asminst{BLE}).

\paragraph{Apartado 3}

A continuación se muestra el código modificado:

\input{int/ex/m4-02/unroll-3-lst2.asm}

Ahora se requiere un tiempo total de $1 + 5 \cdot 11 = 56$. De nuevo, se
necesita un ciclo para el código de iniciación. Cada iteración necesita $13$
ciclos. Debido a que no existe delay slot no es necesario añadir un ciclo de
stall después de la instrucción de salto (\asminst{BLE}).

\[
\textrm{Speedup} = 66/56 = 1.17
\]

\paragraph{Apartado 4}

A continuación se muestra el código modificado:

\input{int/ex/m4-02/unroll-3-lst3.asm}

En total ahora se tienen 2 iteraciones dentro del bucle. Además la quinta
iteración del bucle original se hace ahora al final. El tiempo requerido es:

\[
1 + 2 \cdot 23 + 10 = 57
\]

Si además se planifica, se puede tener:

\input{int/ex/m4-02/unroll-3-lst4.asm}

El tiempo pasa a:
\[
1 + 2 \cdot 13 + 10 = 37 \textrm{ciclos}
\]

\clearpage
