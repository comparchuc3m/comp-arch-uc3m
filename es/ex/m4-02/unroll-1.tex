\begin{acexercise}
Con contribuciones de: Alejandro Calderón.
\end{acexercise}

Un procesador tipo RISC-V tiene un banco de registros de aritmética entera 
de 32 registros con los siguientes registros:

\begin{itemize}
  \item \asmreg{zero}: Siempre cero.
  \item \asmreg{t0}--\asmreg{t6}: Registros temporales.
  \item \asmreg{s0}--\asmreg{s11}: Registros guardados.
  \item \asmreg{a0}--\asmreg{a1}: Argumentos / resultado.
  \item \asmreg{a2}--\asmreg{a7}: Argumentos.
  \item \asmreg{ra}: Dirección de retorno.
  \item \asmreg{sp}: Puntero de pila.
  \item \asmreg{gp}: Puntero global.
  \item \asmreg{tp}: Puntero de hilo.
\end{itemize}

Así mismo, dispone de un banco de registros para valores en coma flotante. Todos los registros
se pueden usar tanto para valores en simple como doble precisión.

\begin{itemize}
  \item \asmreg{ft0}--\asmreg{ft11}: Registros temporales.
  \item \asmreg{fs0}--\asmreg{fs11}: Registros guardados.
  \item \asmreg{fa0}--\asmreg{fa1}: Argumentos / resultado.
  \item \asmreg{fa2}--\asmreg{fa7}: Argumentos.
\end{itemize}

Se asume que en este procesador en concreto se tienen las latencias entre instrucciones
expresadas en la tabla~\ref{tab:unroll-1:latencies}. Asuma que no hay otras detenciones
que las derivadas de las latencias por dependencia de datos y que el procesador puede
captar una instrucción en cada ciclo de reloj.

\begin{table}[htb]
\begin{tabular}{|l|l|c|}
\hline
\textbf{Instrucción productora} &
\textbf{Intrucción consumidora} &
\textbf{Latencia}
\\
\hline

Operación FP & Operación FP & 6\\
\hline

Operación FP & Escritura en memoria & 3\\
\hline

Lectura de memoria & Operación FP & 2\\
\hline

Lectura de memoria & Escritura en memoria & 0\\
\hline

\end{tabular}
\caption{Latencias entre instrucciones}
\label{tab:unroll-1:latencies}
\end{table}

En este procesador se desea ejecutar el siguiente fragmento de código.

\lstinputlisting[language=generalasm2]{int/ex/m4-02/unroll-1-ex.asm}

En este programa los registros \asmreg{t1}, \asmreg{t2} y \asmreg{t3} contienen
las direcciones de comienzo de tres arrays de valores numéricos en doble
precisión. El registro \asmreg{t4} contiene el valor \asmreg{t1}\cppid{+4000}.

Se pide responder a las siguientes cuestiones:

\begin{enumerate}

  \item Determine cuántos ciclos de reloj se requiren para ejecutar todas
        las iteraciones del bucle.

  \item Determine cuántos ciclos de reloj se requieren para ejecutar todas
        las iteraciones del bucle, si se realiza planificación de las instrucciones
        por el compilador. 
        Determine el factor de aceleración con respecto a la versión inicial.

  \item Determine cuántos ciclos de reloj se requieren para ejecutar todas
        las iteraciones del bucle, si se aplica un desenrrollamiento para cada dos iteraciones
        y se realiza planificación de las instrucciones por el compilador.
        Determine el factor de aceleración con respecto a la versión inicial.

  \item Determine cuántos ciclos de reloj se requieren para ejecutar todas
        las iteraciones del bucle, si se aplica un desenrrollamiento para cada cuatro iteraciones
        y se realiza planificación de las instrucciones por el compilador.
        Determine el factor de aceleración con respecto a la versión inicial.

\end{enumerate}

\begin{acsolution}\end{acsolution}

\paragraph{Ejecución inicial}. 
El código suministrado recorre un espacio de \cppid{4000} bytes.
Como en cada iteración se avanzan \cppid{8} posiciones, se tiene
un total de $\frac{4000}{8} = 500$ iteraciones.

Al ejecutar el código se identifican detenciones:

\lstinputlisting[language=generalasm2]{int/ex/m4-02/unroll-1-stalls.asm}

En total, se necesitan 13 ciclos por iteración.
Por tanto, el número total de ciclos será:

\[
ciclos = 13 \times 500 = 6500
\]

\paragraph{Planificación de instrucciones}.
Para planificar las instrucciones del fragmento de código se busca aprovecar los
huecos generados por las detenciones.

\lstinputlisting[language=generalasm2]{int/ex/m4-02/unroll-1-sched.asm}

En total, se necesitan 10 ciclos por iteración.
Por tanto, el número total de ciclos será:

\[
ciclos = 10 \times 500 = 5000
\]
\[
S = \frac{6500}{5000} = 1.3
\]

\paragraph{Desenrollamiento con factor 2}.
Partiendo del código no planificado se obtiene una primera
version de bucle desenrrollado.

\lstinputlisting[language=generalasm2]{int/ex/m4-02/unroll-1-unrollx2.asm}

En esta primera versión, se repite el cuerpo del bucle ajustado los
desplazamientos y utilizando registros distintos del banco de registros,
cuando procede.

En total, se necesitan 22 ciclos por cada dos iteraciones originales.
Teniendo en cuenta, que el número de iteraciones del nuevo bucle es ahora 250,
el número total de ciclos será:

\[
ciclos = 22 \times 250 = 5500
\]

Planificando estas instrucciones, se tiene:

\lstinputlisting[language=generalasm2]{int/ex/m4-02/unroll-1-unrollx2-sched.asm}

En total, se necesitan 12 ciclos para cado dos iteraciones originales.
Teniendo en cuenta, que el número de iteraciones del nuevo bucle es 250,
el número total de ciclos será:

\[
ciclos = 12 \times 250 = 3000
\]
\[
S = \frac{6500}{3000} = 2.17
\]

\paragraph{Desenrollamiento con factor 4}.
Partiendo del código no planificado se obtiene una primera
version de bucle desenrrollado.

\lstinputlisting[language=generalasm2]{int/ex/m4-02/unroll-1-unrollx4.asm}

En esta primera versión, se repite el cuerpo del bucle ajustado los
desplazamientos y utilizando registros distintos del banco de registros,
cuando procede.

En total, se necesitan 40 ciclos por cada cuatro iteraciones originales.
Teniendo en cuenta, que el número de iteraciones del nuevo bucle es ahora 125,
el número total de ciclos será:

\[
ciclos = 40 \times 125 = 5000
\]

Planificando estas instrucciones, se tiene:

\lstinputlisting[language=generalasm2]{int/ex/m4-02/unroll-1-unrollx4-sched.asm}

En total, se necesitan 20 ciclos para cado cuatro iteraciones originales.
Teniendo en cuenta, que el número de iteraciones del nuevo bucle es 125,
el número total de ciclos será.

\[
ciclos = 20 \times 125 = 2500
\]
\[
S = \frac{6500}{2500} = 2.6
\]
