\begin{acexercise}\end{acexercise}

Sea un procesador con arquitectura \textbf{Intel P6 o posterior}, en el que se
tienen dos variables (\cppid{z} y \cppid{t}) que inicialmente tienen el valor
\cppid{42}. En dicho procesador dos hilos ejecutan concurrentemente.

El hilo 1 ejecuta:
\begin{lstlisting}[language={[x86masm]Assembler},basicstyle=\normalsize]
mov [_z], 1
mov r1, [_z]
mov r2, [_t]
\end{lstlisting}

El hilo 2 ejecuta:
\begin{lstlisting}[language={[x86masm]Assembler},basicstyle=\normalsize]
mov [_t], 1
mov r3, [_t]
mov r4, [_z]
\end{lstlisting}

¿Es posible que al final de la ejecución del hilo 2 los registros \asmreg{r2} y \asmreg{r4}
tengan ambos el valor de \asmlabel{42}? Justifique su respuesta.

\begin{acsolution}\end{acsolution}

Si es posible porque las escrituras \textbf{pueden percibirse en distinto orden
por cada procesador}. De esta manera en el hilo 1, puede tenerse \cppid{r1=1} y
\cppid{r2=42}, mientras que en el hilo 2, puede tenerse \cppid{r3=1} y
\cppid{r4=42}.
