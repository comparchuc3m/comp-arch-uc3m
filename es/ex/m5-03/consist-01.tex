\begin{acexercise}\end{acexercise}


El siguiente programa se ejecuta sobre una máquina que dispone de 3
procesadores. Sus variables se encuentran en memoria compartida e iniciadas a
\cppid{0}.

El procesador 1 ejecuta:
\begin{lstlisting}
X=1;
\end{lstlisting}

El procesador 2 ejecuta:
\begin{lstlisting}
Print(X);
Print(Y);
\end{lstlisting}

El procesador 3 ejecuta:
\begin{lstlisting}
X=1;
Print(X);
\end{lstlisting}

Asumiendo que todos los accesos a las variables se ejecutan de forma atómica,
conteste a las siguientes preguntas:

\begin{itemize}
  \item Para cada uno de los siguientes resultados indique:
    \begin{itemize}
      \item ¿Es posible que se obtenga el resultado?
      \item Si el resultado es posible ofrezca un orden total de las instrucciones que produzcan el resultado.
    \end{itemize}
\end{itemize}

\bigskip
\begin{tabular}{|l|l|l|}
\hline
\cppid{Print(X)} [P2] & \cppid{Print(Y)} & \cppid{Print(X)} [P3]\\
\hline
0 & 0 & 0\\\hline
0 & 0 & 1\\\hline
0 & 1 & 0\\\hline
0 & 1 & 1\\\hline
1 & 0 & 0\\\hline
1 & 0 & 1\\\hline
1 & 1 & 0\\\hline
1 & 1 & 1\\\hline
\end{tabular}
\bigskip


\begin{itemize}
  \item Indique un orden global de instrucciones (y su correspondiente resultado) 
        que sea imposible bajo consistencia secuencial, pero posible bajo otros 
        modelos más relajados.
\end{itemize}

\begin{acsolution}\end{acsolution}


La solución a este ejercicio pasa por dibujar un grafo de dependencia en orden
de programa y para cada caso añadir las relaciones de orden adicionales.

Caso 0,0,0:

\begin{lstlisting}
Print X(2)
Print Y
Y=1
Print X(3)
X=1
\end{lstlisting}

Caso 0,0,1

\begin{lstlisting}
Print X(2)
Print Y
Y=1
X=1
Print X(3)
\end{lstlisting}

Caso 0,1,0

\begin{lstlisting}
Print X(2)
Y=1
Print Y
Print X(3)
X=1
\end{lstlisting}

Caso 0,1,1

\begin{lstlisting}
Print X(2)
Y=1
Print Y
X=1
Print X(3)
\end{lstlisting}

Caso 1,0,0: {\color{red}No es posible}

Caso 1,0,1:

\begin{lstlisting}
X=1
Print X(2)
Print Y
Y=1
Print X(3)
\end{lstlisting}

Caso 1,1,0

\begin{lstlisting}
Y=1
Print X(3)
X=1
Print X(2)
Print Y
\end{lstlisting}

Caso 1,1,1

\begin{lstlisting}
X=1
Y=1
Print X(2)
Print Y
Print X(3)
\end{lstlisting}

Por tanto la única opción que no se produce con consistencia secuencial es la combinación 1,0,0.

Sin embargo con consistencia relajada al eliminarse el requisito de orden de programa, se tiene:

\begin{lstlisting}
Print Y
Y=1
Print X(2)
Print X(3)
X=1
\end{lstlisting}
