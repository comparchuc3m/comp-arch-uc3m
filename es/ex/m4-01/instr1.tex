\begin{acexercise}\end{acexercise}
\label{ex:m4-01:instr-01}

Dado el siguiente fragmento de código:

\lstinputlisting[language=generalasm2]{int/ex/m4-01/instr1-code-question.asm}

Se considera la ejecución en un pipeline de 5 etapas.
Para las instrucciones de bifurcación tanto la dirección de salto como el
sentido del salto se conocen al final de la etapa de decodificación y se
predicen todos los saltos a no tomado.

Se pide:
\begin{enumerate}
  \item Identifique los riesgos de tipo RAW en el código.

  \item Presente una temporización (cronograma) cuando no hay hardware de envio adelantado
        (\emph{forwarding}). Determine cuántos ciclos se requieren por iteración,
        asumiendo que el bucle se ejecuta durante 100 iteraciones.

  \item Presente una temporización cuando si hay hardware de envío adelantado
        (\emph{forwarding}). Determine cuántos ciclos se requieren por iteración,
        asumiendo que el bucle se ejecuta durante 100 iteraciones.

  \item Presente una temporización (cronograma) asumiendo ahora que todos los accesos 
        a la caché de datos requieren 2 ciclos para lecturas y 3 para escrituras,
        y que las lecturas de la caché de instrucciones siempre requieren dos ciclos.
        Evalúe los casos con y sin envío adelantado.
\end{enumerate}


\begin{acsolution}\end{acsolution}

Si se numeran las instrucciones, se tiene:

\lstinputlisting[language=generalasm2]{int/ex/m4-01/instr1-code-num.asm}

\paragraph{Cuestión 1}: Se identifican los siguientes riesgos RAW:

\begin{enumerate}
  \item \asmreg{t1}: \asmlabel{I1} $\rightarrow$ \asmlabel{I3}
  \item \asmreg{t2}: \asmlabel{I2} $\rightarrow$ \asmlabel{I3}
  \item \asmreg{t2}: \asmlabel{I3} $\rightarrow$ \asmlabel{I4}
  \item \asmreg{s0}: \asmlabel{I5} $\rightarrow$ \asmlabel{I6}
\end{enumerate}

\paragraph{Cuestión 2}: El cronograma se presenta en la tabla ~\ref{ex:m4-01:instr-01:chrono:noforward}.

\begin{table}[htb]
\input{int/ex/m4-01/instr1-noforward}
\caption{Diagrama de tiempos del ejercicio~\ref{ex:m4-01:instr-01} sin envío adelantado.}
\label{ex:m4-01:instr-01:chrono:noforward}
\end{table}

Para determinar el número de ciclos de una iteración se debe calcular cuantos ciclos
pasan desde que comienza la primera instrucción de una iteración hasta que se comienza
la primera instrucción de la siguiente iteración.

Para todas las iteraciones menos la última esto será hasta que se capta la
instrucción \asmlabel{I1}. Para la última iteración será hasta que se capta la
instrucción \asmlabel{I7}.

\[
ciclos_{iteracion} = \frac{10 + 13 \cdot 99}{100} = \frac{1297}{100} = 12.97
\]

\paragraph{Cuestión 3}: El cronograma se presenta en la tabla ~\ref{ex:m4-01:instr-01:chrono:forward}.

\begin{table}[htb]
\input{int/ex/m4-01/instr1-forward}
\caption{Diagrama de tiempos del ejercicio~\ref{ex:m4-01:instr-01} con envío adelantado.}
\label{ex:m4-01:instr-01:chrono:forward}
\end{table}

Se deben tener en cuenta las siguientes observaciones:
\begin{itemize}

\item En el ciclo 5, la instrucción \textmark{I2} tiene una detención
      porque puede recibir mediante reenvío el registro \asmreg{t1},
      pero no el registro \asmreg{t2}.

\item El registro \asmreg{t2} se reenvía de la instrucción \textmark{I2}
      (etapa \textgood{M}) a la instrucción \textmark{I3} 
      (etapa \textgood{EX}).

\item La instrucción \textmark{I2} tiene una detención en el ciclo 5
      porque la unidad de decodificación está ocupada.

\item En el ciclo 9, la instrucción \textmark{I6} tiene una detención
      en su etapa \textgood{ID} porque necesita el valor de \asmreg{s0}
      para determinar el sentido del salto.

\end{itemize}

\[
ciclos_{iteracion} = \frac{7 + 9 \cdot 99}{100} = \frac{898}{100} = 8.98
\]

Por tanto, el \emph{speedup} debido al envío adelantado es:

\[
S = \frac{12.97}{8.98} = 1.44
\]

\paragraph{Cuestión 4}: En este caso se deben repetir los cronogramas de los apartados anteriores
con las siguientes modificaciones:
\begin{itemize}
  \item Deben realizarse siempre dos ciclos para la etapa IF.
  \item La etapa M de las intrucciones que no acceden a memoria será siempre de un ciclo.
  \item La etapa M de las instrucciones de carga (como \asminst{lw}) será siempre de dos ciclos.
  \item La etapa M de las instrucciones de almacenamiento (como \asminst{sw}) será siempre de tres ciclos.
\end{itemize}
