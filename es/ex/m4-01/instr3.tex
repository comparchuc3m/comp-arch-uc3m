\clearpage
\begin{acexercise}
\label{ex:m4-01:instr-03}
\end{acexercise}
%Examen de octubre de 2013.

En un determinado procesador pretende ejecutar el siguiente segmento de código:

\lstinputlisting[language=generalasm2]{int/ex/m4-01/instr3-code-question.asm}

Asuma que el procesador tiene una arquitectura segmentada de 5 etapas
(captación, decodificación, ejecución, memoria y post-escritura) sin envío
adelantado. Todas las operaciones se ejecutan en un ciclo, excepto las
operaciones de carga y almacenamiento que requieren dos ciclos adicionales de
latencia en el acceso a memoria, y las instrucciones de salto que requieren un
ciclo adicional de ejecución. Asuma que tanto la dirección de salto como
la evaluación de la condición de bifurcación se conocen al final de la etapa
de decodificación.

\begin{enumerate}

  \item Determine los riesgos de datos RAW que presenta el código.
        Ignore posibles riesgos entre interaciones consecutivas.

  \item Muestre un diagrama tiempos con las fases de ejecución de cada
        instrucción para una iteración.

  \item Determine cuantos ciclos requiere la ejecución de una iteración del
        bucle si no hay ningún tipo de predicción de saltos.

  \item Determine cuantos ciclos requiere la ejecución de una iteración si se
        usa un predictor de saltos que predice siempre a tomado.

\end{enumerate}

\begin{acsolution}\end{acsolution}


\paragraph{Apartado 1}

\begin{itemize}
  \item \asmreg{t3}: \asmlabel{i0} $\rightarrow$ \asmlabel{i1}
  \item \asmreg{t1}: \asmlabel{i1} $\rightarrow$ \asmlabel{i2}
  \item \asmreg{t1}: \asmlabel{i2} $\rightarrow$ \asmlabel{i4}
  \item \asmreg{t4}: \asmlabel{i3} $\rightarrow$ \asmlabel{i5}
\end{itemize}


\paragraph{Apartado 2}

La Tabla~\ref{tab1:sol-oct-2013} muestra el correspondiente diagrama de
tiempos.

\begin{table}[htb]
\begin{tabular}{|l|*{15}{>{\footnotesize}c|}}
\hline
\textbf{Instr.} &
1 & 2 & 3 & 4 & 5 &
6 & 7 & 8 & 9 & 10 &
11 & 12 & 13 & 14 & 15 
\\
\hline
\hline

\asmlabel{I0}: \asminst{lw} \asmreg{t3}, 0(\asmreg{t0}) &
IF & ID & EX & M & M &
M & WB & & & &
& & & & 
\\
\hline

\asmlabel{I1}: \asminst{lw} \asmreg{t1}, 0(\asmreg{t3}) &
& IF & ID & -- & -- &
-- & -- & EX & M & M &
M & WB & & & 
\\
\hline

\asmlabel{I2}: \asminst{addi} \asmreg{t1}, \asmreg{t1}, 1 &
& & IF & -- & -- & --
& -- & ID & -- & -- &
-- & -- & EX & M & WB 
\\
\hline

\asmlabel{I3}: \asminst{sub} \asmreg{t4}, \asmreg{t3}, \asmreg{t2} &
& & & & &
& & IF & -- & -- &
-- & -- & ID & EX & M 
\\
\hline

\asmlabel{I4}: \asminst{sw} \asmreg{t1}, 0(\asmreg{t3}) &
& & & & &
& & & & &
& & IF & ID & -- 
\\
\hline

\asmlabel{I5}: \asminst{bnz} \asmreg{t4}, \asmlabel{I0} &
& & & & &
& & & & &
& & & IF & -- 
\\
\hline

\end{tabular}
\vspace{2em}

\begin{tabular}{|l|*{7}{>{\footnotesize}c|}}
\hline
\textbf{Instr.} &
16 & 17 & 18 & 19 & 20 &
21
\\
\hline
\hline

\asmlabel{I0}: \asminst{lw} \asmreg{t3}, 0(\asmreg{t0}) &
& & & & &
\\
\hline

\asmlabel{I1}: \asminst{lw} \asmreg{t1}, 0(\asmreg{t3}) &
& & & & &
\\
\hline

\asmlabel{I2}: \asminst{addi} \asmreg{t1}, \asmreg{t1}, 1 &
& & & & &
\\
\hline

\asmlabel{I3}: \asminst{sub} \asmreg{t4}, \asmreg{t3}, \asmreg{t2} &
WB & & & & &
\\
\hline

\asmlabel{I4}: \asminst{sw} \asmreg{t1}, 0(\asmreg{t3}) &
EX & M & M & M & WB &
\\
\hline

\asmlabel{I5}: \asminst{bnz} \asmreg{t4}, \asmlabel{I0} &
ID & EX & EX & -- & M &
WB
\\
\hline

\end{tabular}

\caption{Diagrama de tiempos del ejercicio~\ref{ex:m4-01:instr-03}.}
\label{tab1:sol-oct-2013}
\end{table}

\begin{itemize}

\item \asmlabel{I0}: Requiere tres ciclos de memoria

\item \asmlabel{I1}: Se produce una detención en la etapa de decodificación hasta que 
I0 hace WB de \asmreg{t3}.
Requiere 3 ciclos de memoria.

\item \asmlabel{I2}: 
No puede comenzar a decodificar hasta que se libera la correspondiente unidad por I1.
Se produce una detención en la etapa de decodificación hasta que I1 hace WB de \asmreg{t1}.

\item \asmlabel{I3}: 
No se puede captar la instrucción hasta que I2 libera la correspondiente unidad.
No se puede decodificar la instrucción hasta que I2 libera la correspondiente unidad.

\item \asmlabel{I4}:
No se puede captar la instrucción hasta que I3 libera la correspondiente unidad.
Se produce una detención en la etapa de decoficiación hasta que I2 hace WB de \asmreg{t1}.
Requiere tres ciclos de memoria.

\item \asmlabel{I5}: 
No se puede decodificar la instrucción hasta que I4 libera la correspondiente unidad.
Requiere dos ciclos de ejecución.
No se puede realizar la etapa de memoria hasta que I4 libera la correspondiente unidad.

\end{itemize}

\paragraph{Apartado 3}

En ese caso, la instrucción \asmlabel{I0} de la siguiente iteración no puede
captarse hasta que se conoce la dirección de bifurcación y el resultado
de la condición de bifurcación. Esto ocurre al final de la etapa de
decodificación de \asmlabel{I5}. Por lo que \asmlabel{I0} se capta en el ciclo 17 
y una iteración requiere 16 ciclos.

\paragraph{Apartado 4}

En ese caso, la predicción se realiza en la etapa de decodificación, que es
cuando se sabe que se trata de una instrucción de salto, por lo que \asmlabel{I0}
comenzaría en el ciclo 17, y una iteración requiere 16 ciclos.
Obsérvese que en este caso no hay diferencia entre no usar un predictor de saltos o predecir el salto a tomado.

