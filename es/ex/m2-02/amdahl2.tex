\begin{acexercise}\end{acexercise}

Se dispone de una aplicación que permite procesar una imagen de muy alta
resolución de la cual una cierta fracción es paralelizable y otra parte debe
ejecutarse secuencialmente. Se asume que no hay límite superior al número de
procesos en que se puede paralelizar.

Se desea obtener un speedup global de 10 en la versión paralela.  Exprese la
fracción de código que debe ser paralelizable en función del grado de
paralelismo (número de procesos en que se paraleliza).


\begin{acsolution}\end{acsolution}

\[
S = \frac{1}{\left(1 - F \right) + \displaystyle \frac{F}{n} }
\]

\[
S \times \left( (1-F) + \frac{F}{n} \right) = 1
\]

\[
S - S \times F + \frac{S \times F}{n} = 1
\]

\[
n \times S - n \times S \times F + S \times F = n
\]

\[
n \times S - n = n \times S \times F - S \times F
\]

\[
S \times F \times \left( n - 1 \right) = n \times \left( S - 1 \right)
\]

\[
F = \frac{n \times (S-1)}{S \times(n-1)}
\]

Para el caso de $S = 10$

\[
F = \frac{n \times (10-1)}{10 \times (n-1)} = 
\frac{9 \times n}{10 \times n - 10}
\]

Para que tenga sentido F debe ser menor o igual que 1. 

\[
F \leq 1
\]

\[
\frac{9 \times n}{10 \times n -10} \leq 1
\]

\[
9 \times n \leq 10 \times n-10
\]

\[
n \geq 10
\]


