\begin{acexercise}\end{acexercise}

Dado el siguiente fragmento de código:

\begin{lstlisting}
for (i=0; i<64; i++)
  for (j=0; j<1024; j=j+2 )
    v[i][j] = v[i][j] * v[i][j+1] +  b[i]
\end{lstlisting}

Se ejecuta en una arquitectura con una cache de tamaño 256 KB, con tamaño de
bloque 64 Bytes y de palabra de 8 Bytes en el que la memoria caché es
totalmente asociativa y utiliza un algoritmo de reemplazo LRU. Siendo \cppid{v}
y \cppid{b} matrices de números reales de 8 byte almacenadas por filas (estilo
C) y tamaños $64 \times 1024$ (\cppid{v}), $64$ (\cppid{b}). Se supondrá que
las variables índice se almacenan en registros y que al empezar el código la
caché está vacía.

Se pide determinar:

\begin{enumerate}
  \item El número de fallos de cache. 

  \item La tasa media de fallos.

  \item Razone si se produce alguna vez reemplazo de una línea de cache
precargada. No es necesario identificar qué reemplazos hay sino razonar la
existencia o no de reemplazos. 

  \item Razone si la técnica de optimización de intercambio de bucles
mejoraría la tasa media de fallos de la caché (no es necesario calcularla).

\end{enumerate}


\begin{acsolution}\end{acsolution}

\paragraph{Apartado 1}

Cache de 256 KB $\rightarrow$ $2^{15}$ elementos = $2^{12}$  bloques

Bloque de 64B = 8 elementos

La Tabla~\ref{tab:sol-oct-2013b} muestra la secuencia de accesos.

\begin{table}
\begin{tabular}{|l|p{0.3\textwidth}|l|p{0.3\textwidth}|}
\hline
Acceso a \cppid{v} &
Fallo caché &
Acceso a \cppid{b} &
Fallo caché
\\
\hline
\hline

0,0 & SI &
0 & SI\\
\hline

0,2 & No. Porque está en la misma línea de cache que \cppid{v[0,0]} &
0 & No, Porque está en la misma línea de cache que \cppid{b[0]}
\\
\hline


\ldots & &
0 & No
\\
\hline

0,7 & NO &
0 & No
\\
\hline

0,8 & SI &
0 & No
\\
\hline
	
0,10 & NO &
0 & No
\\
\hline

\ldots & & & \\

0,16 & SI & 
0 & No
\\
\hline

\ldots & & & \\
			
0,1022 & NO &
0 & No
\\
\hline

1,0 & SI &
1 & NO\\
\hline 

1,1 & No porque está en la misma línea de cache que \cppid{v[1,0]} &
1 & NO\\
\hline 

\ldots & & & \\
				
2,0 & SI & 2 & NO\\
\hline

\ldots &&&\\
				
8,0 & SI & 8 & SI\\
\hline

\ldots &&&\\

31,0 & SI & 31 & NO\\
\hline

\end{tabular}
\caption{Secuencia de accesos del ejercicio~\ref{sol:oct-2013b}.}
\label{tab:sol-oct-2013b}
\end{table}

Se  producen fallos de cache al acceder a 
\cppid{v[0,0]}, \cppid{v[0,8]}, \cppid{v[0,16]}, \ldots 
\cppid{v[1,0]}, \cppid{v[1,8]}, \cppid{v[0,16]}, \ldots
\cppid{[2,0]}, \ldots \cppid{v[31,0]}, \ldots 
y hay aciertos en todas las demás. Como la matriz tiene $64 \times 1024$
casillas y sólo hay error en una de cada 8 el total de fallos es  $\frac{64
\cdot 1024}{8}= 8192$ fallos 

En \cppid{b} se producen fallos en las posiciones 1, 8, 16, 24, 32, 40, 48 y 56.
En total se producen 8 fallos. 

El número total fallos es $8192 + 8 = 8200$ fallos.

\paragraph{Apartado 2}

Accesos totales.

En cada instrucción hay 4 accesos y la instrucción de cálculo se realiza
$\frac{64 \cdot 1024}{2}$ veces. Por tanto, tenemos $\frac{4 \cdot 64 \cdot 1024}{2} = 131072$ accesos u la tasa de fallos es:

\[
m = \frac{8200}{131072} = 0.0625 \Rightarrow 6,25 \%
\] 

\paragraph{Apartado 3}

En la cache caben 256 KB = $2^{15}$ elementos = $2^{12}$ bloques y
la matriz v tiene $64 \cdot 1024 = 2^{16}$ elementos = $2^{13}$ bloques.

Luego la matriz \cppid{v}  no cabe en la caché. 
Se producirá un reemplazo de la línea que contiene la posición \cppid{v[0,0]}
cuando lleguemos, aproximadamente, a la mitad de la matriz \cppid{v}.

Si no existiera la matriz \cppid{b} el reemplazo se produciría al llegar a la mitad de
la matriz (pues en caché sólo cabe la mitad de la matriz), es decir en la
posición \cppid{v[32,0]}.

Teniendo en cuenta que la matriz \cppid{b} también ocupa cache y cuando
llegamos a la posición \cppid{v[32,0]} hemos ocupado 4 líneas de caché con la
matriz \cppid{b} ( posiciones de la \cppid{b[0]} a la \cppid{b[32]}) podemos concluir que el reemplazo
se producirá 4 líneas de cache antes de llegar a la \cppid{v[32,0]}. Esto es en la
posición \cppid{v[$31,8 \cdot \frac{1024}{8}-4$]} = \cppid{v[31,992]}. 
Este dato no se pide explícitamente en el ejercicio. 

De igual forma se producirá un reemplazo de la línea de caché de \cppid{b[0]} en,
aproximadamente, el mismo momento. 

También habrá reemplazos sucesivos de \cppid{v[0,8]}, \cppid{v[0,16]}, \ldots
según se va necesitando sitio para las posiciones de \cppid{v} que están más allá de la
mitad del recorrido.

\paragraph{Apartado 4}

El intercambio de bucles disminuiría la localidad espacial en los accesos a
\cppid{v}, aumentando la tasa de fallos.
