\begin{acexercise}
Con contribuciones de: David Expósito.
\end{acexercise}
\label{ex:m5-02:dir-01}

Dados los siguientes fragmentos de código que ejecutan 3 hilos empleando el
protocolo de coherencia MSI.

\begin{lstlisting}
// Código del hilo 0
for(i=0;i<16;i++){
  a[i]= 16;
}
 
// Código del hilo 1
tmp=0;
for(i=0;i<16;i++){
  a[i]=tmp;
  tmp+=a[i];
}
 
// Código del hilo 2
cnt=0;
for(i=0;i<4;i++){
  cnt+=b[i];
}
\end{lstlisting}

El computador tiene una arquitectura CC-NUMA con:

\begin{itemize}
  \item 2 procesadores de 32 bits con único nivel de memoria caché que 
        es privado a cada procesador y tiene una correspondencia directa. 
        El tamaño de línea caché es de 16 bytes. 
  \item Las memorias caché están inicialmente vacías.
  \item Los hilos 0 y 2 se ejecutan en el procesador 0 mientras que el hilo 1 
        se ejecuta en el procesador  1.
  \item Las variables \cppid{i}, \cppid{tmp} y \cppid{cnt} se almacenan en 
        registros (no se almacenan en memoria). 
  \item El bloque que contiene \cppid{a[0]} está asociado a la misma línea 
        caché que el bloque que contiene \cppid{b[0]}.
\end{itemize}

Se pide rellenar las siguientes tablas y justificar la respuesta para los siguientes apartados:

\begin{enumerate}

\item Partiendo de la situación inicial, indicar en la siguiente tabla las
transiciones de estado del bloque que almacena \cppid{a[0]} cuando se ejecuta primero el
hilo 0 completamente y a continuación el hilo 1. Indique el tráfico de bus
asociado a cada hilo. 
  \begin{itemize}
    \item \textbf{Nota}: en el caso de haber varias transiciones asociadas a un hilo, 
          indique todas ellas.
  \end{itemize}

\begin{tabular}{|l|l|l|l|}
\hline
Código & Transición P0 & Transición P1 & Señales de bus\\
\hline\hline
Hilo 0 & & & \\
\hline
Hilo 1 & & & \\
\hline
\end{tabular}

\item Partiendo de la situación inicial, indicar en la siguiente tabla las
transiciones de estado del bloque que almacena \cppid{a[0]} cuando se ejecuta
primero el hilo 1 completamente y a continuación el hilo 0 Indique el tráfico de
bus asociado a cada hilo. 

  \begin{itemize}
    \item \textbf{Nota}: en el caso de haber varias transiciones asociadas a un hilo, 
          indique todas ellas.
  \end{itemize}

\begin{tabular}{|l|l|l|l|}
\hline
Código & Transición P0 & Transición P1 & Señales de bus\\
\hline\hline
Hilo 0 & & & \\
\hline
Hilo 1 & & & \\
\hline
\end{tabular}

\item Partiendo de la situación inicial, indicar en la siguiente tabla las
transiciones de estado del bloque que almacena \cppid{a[0]} cuando se ejecuta primero el
hilo 0 completamente, a continuación el hilo 2 completamente y finalmente el
hilo 1 Indique el tráfico de bus asociado a cada hilo. 

  \begin{itemize}
    \item \textbf{Nota}: en el caso de haber varias transiciones asociadas a un hilo, 
          indique todas ellas.
  \end{itemize}

\begin{tabular}{|l|l|l|l|}
\hline
Código & Transición P0 & Transición P1 & Señales de bus\\
\hline\hline
Hilo 0 & & & \\
\hline
Hilo 1 & & & \\
\hline
Hilo 2 & & & \\
\hline
\end{tabular}

\item Para cada uno de los escenarios anteriores, indique el número de fallos caché existente para cada proceso.


\begin{tabular}{|l|l|l|}
\hline
Escenario & Fallos caché P0 & Fallos caché P1\\
\hline
\hline
Hilo 0 $\rightarrow$ hilo 1 & & \\		
\hline
Hilo 1 $\rightarrow$ hilo 0 & & \\	
\hline
Hilo 0 $\rightarrow$ hilo 2 $\rightarrow$ hilo 1 &&\\
\hline
\end{tabular}

\end{enumerate}

\begin{acsolution}\end{acsolution}


\paragraph{Apartado 1}

Se muestran las transiciones en la Tabla~\ref{tab:sol-jan-2015a}.

\begin{table}[!htbp]

\begin{tabular}{|l|l|l|l|}

\hline
Código & Transición P0 & Transición P1 & Señales de bus\\
\hline
\hline

Hilo 0	&
I $\rightarrow$ E & I & Write miss
\\
\hline

Hilo 1 &
E $\rightarrow$ I & I $\rightarrow$ E , E $\rightarrow$ E &
Write miss; Write-back block
\\
\hline

\end{tabular}

\caption{Transiciones de apartado 1 de ejercicio~\ref{ex:m5-02:dir-01}}
\label{tab:sol-jan-2015a}
\end{table}

\paragraph{Apartado 2}

Se muestran las transiciones en la Tabla~\ref{tab:sol-jan-2015b}.

\begin{table}[!htbp]

\begin{tabular}{|l|l|l|l|}

\hline
Código & Transición P0 & Transición P1 & Señales de bus\\
\hline
\hline

Hilo 1 &
I &  I $\rightarrow$ E ; E $\rightarrow$ E & Write miss
\\
\hline

Hilo 0 &
I $\rightarrow$ E  & E $\rightarrow$ I & Write miss; Write-back block
\\
\hline

\end{tabular}

\caption{Transiciones de apartado 2 de ejercicio~\ref{ex:m5-02:dir-01}}
\label{tab:sol-jan-2015b}
\end{table}

\paragraph{Apartado 3}

Se muestran las transiciones en la Tabla~\ref{tab:sol-jan-2015c}.

Cada bloque tiene 4 palabras. Por lo que los hilos 0 y 1 acceden a 4 bloques y
mientras que el hilo 2 accede a 1 bloque. En el hilo 1 la primera línea causa
fallo caché mientras que la segunda, acierto.

\begin{table}[!htbp]

\begin{tabular}{|l|l|l|l|}

\hline
Código & Transición P0 & Transición P1 & Señales de bus\\
\hline
\hline

Hilo 0 &
I $\rightarrow$ E (almacena \cppid{a[0]}) & I &
Write miss
\\
\hline

Hilo 2 & 
E $\rightarrow$ S (almacena \cppid{b[0]}) & I &
Read miss; write back block
\\
\hline

Hilo 1 &
S $\rightarrow$ S (\cppid{b[0]}) &
I $\rightarrow$ E ; E $\rightarrow$ E &
Write miss
\\
\hline

\end{tabular}

\caption{Transiciones de apartado 3 de ejercicio~\ref{ex:m5-02:dir-01}}
\label{tab:sol-jan-2015c}
\end{table}

\paragraph{Apartado 4}

Se muestran las transiciones en la Tabla~\ref{tab:sol-jan-2015d}.

\begin{table}[!htbp]

\begin{tabular}{|l|l|l|}

\hline
Escenario & Fallos caché P0 & Fallos caché P1\\
\hline
\hline

hilo 0 $\rightarrow$ hilo 1 &
4 & 4
\\
\hline

hilo 1 $\rightarrow$ hilo 0 &
4 & 4
\\
\hline

hilo 0 $\rightarrow$ hilo 2 -> hilo 1 &
4 + 1 & 4
\\
\hline

\end{tabular}

\caption{Transiciones de apartado 4 de ejercicio~\ref{ex:m5-02:dir-01}}
\label{tab:sol-jan-2015d}
\end{table}

