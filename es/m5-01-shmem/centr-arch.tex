\section{Arquitecturas de memoria compartida centralizada}

\begin{frame}[t]{SMP y jerarquía de memoria}
\begin{itemize}
  \item ¿Por qué usar memoria centralizada?
    \begin{itemize}
      \item Cachés grandes multi-nivel reducen la demanda de ancho
            de banda sobre memoria principal.
    \end{itemize}

  \mode<presentation>{\vfill\pause}
  \item \textgood{Evolución}:
    \begin{enumerate}[1.]
      \item Mono-núcleo con memoria en \textmark{bus compartido}.
      \item Conexión de memoria a \textmark{bus separado} solamente para memoria.
    \end{enumerate}
\end{itemize}
\end{frame}

\begin{frame}[t]{Memoria caché}
\begin{itemize}
  \item \textgood{Tipos de datos} en memoria caché:
    \begin{itemize}
      \item \textmark{Datos privados}: Datos usados por un único procesador.
      \item \textmark{Datos compartidos}: Datos usados por varios procesadores.
    \end{itemize}

  \mode<presentation>{\vfill\pause}
  \item \textbad{Problema} con datos compartidos:
    \begin{itemize}
      \item \pause El dato puede replicarse en múltiples caché.
      \item \pause Reduce la contención.
        \begin{itemize}
          \item Cada procesador accede a su copia local.
        \end{itemize}
      \item \pause Si dos procesadores modifican sus copias \ldots
        \begin{itemize}
          \item ¿Coherencia de caché?
        \end{itemize}
    \end{itemize}
\end{itemize}
\end{frame}

\begin{frame}[t,fragile]{Coherencia de caché}

\begin{columns}[T]

\column{.3\textwidth}
\begin{block}{Thread 1}
\begin{lstlisting}[language=generalasm2]
lw t0, dirx
addi t0, t0, 1
sw t0, dirx
\end{lstlisting}
\end{block}

\column{.3\textwidth}
\begin{block}{Thread 2}
\begin{lstlisting}[language=generalasm2]
lw t0, dirx
\end{lstlisting}
\end{block}

\column{.3\textwidth}
\begin{itemize}
\item \asmreg{t0} inicialmente 1.
\item Asumiendo escritura inmediata.
\end{itemize}

\end{columns}

\mode<presentation>{\vfill\pause}
{\footnotesize
\begin{tabular}{l|l|l|l|l}

Proceso &
Instrucción &
Caché P1 &
Caché P2 &
Memoria principal
\\

\hline
\hline

\pause
T1 & Inicialmente & No presente & No presente & 1 \pause\\
\hline
T1 & \asminst{lw} \asmreg{t0}, \asmlabel{dirx} & 1 & No presente & 1 \pause\\
\hline
T1 & \asminst{addi} \asmreg{t0}, \asmreg{t0}, \asmlabel{1} & 1 & No presente & 1 \pause\\
\hline
T2 & \asminst{lw} \asmreg{t0}, \asmlabel{dirx} & 1 & 1 & 1 \pause\\
\hline
T1 & \asminst{sw} \asmreg{t0}, \asmlabel{dirx} & 2 & 1 & 1 \\
\hline

\end{tabular}
}

\end{frame}

\begin{frame}[t]{Incoherencia de caché}
\begin{itemize}
  \item ¿Por qué se da la incoherencia?
    \begin{itemize}
      \item \textgood{Dualidad de estado}:
        \begin{itemize}
          \item \textgood{Estado global} $\rightarrow$ \textmark{Memoria principal}.
          \item \textgood{Estado local}  $\rightarrow$ \textmark{Caché privada}.
      \end{itemize}
    \end{itemize}

  \mode<presentation>{\vfill\pause}
  \item Un sistema de memoria es \textgood{coherente} si cualquier lectura de una posición
        devuelve el valor más reciente que se haya escrito para esa posición.

  \mode<presentation>{\vfill\pause}
  \item \textgood{Dos aspectos}:
    \begin{itemize}
      \item \textmark{Coherencia}: ¿Qué valor devuelve una lectura?
      \item \textmark{Consistencia}: ¿Cuándo obtiene una lectura un valor escrito?
    \end{itemize}
\end{itemize}
\end{frame}

\begin{frame}[t]{Condiciones para la coherencia}
\begin{itemize}
  \item \textmark{Preservación de orden de programa}:
    \begin{itemize}
      \item Una lectura del procesador P sobre la posición X posterior a una escritura del procesador P sobre la posición X, sin escrituras intermedias de X por otro procesador, siempre devuelve el valor escrito por P.
    \end{itemize}

  \mode<presentation>{\vfill\pause}
  \item \textmark{Vista coherente de la memoria}:
    \begin{itemize}
      \item Una lectura de un procesador sobre la posición X posterior a una escritura por otro procesador sobre la posición X, devuelve el valor escrito si las dos operaciones están suficientemente separadas en el tiempo y no hay escrituras intermedias sobre X.
    \end{itemize}

  \mode<presentation>{\vfill\pause}
  \item \textmark{Serialización de escrituras}:
    \begin{itemize}
      \item Dos escrituras sobre la misma posición por dos procesadores son vistas en el mismo orden por todos los procesadores.
    \end{itemize}

\end{itemize}
\end{frame}

\begin{frame}[t]{Consistencia de memoria}
\begin{itemize}
  \item Define en qué momento un proceso que lee verá una escritura.

  \mode<presentation>{\vfill\pause}
  \item \textgood{Coherencia} y \textgood{consistencia} son complementarias:
    \begin{itemize}
      \item \textmark{Coherencia}: Comportamiento de lecturas y escrituras a una única posición de memoria.
      \item \textmark{Consistencia}: Comportamiento de lecturas y escrituras con respecto a accesos a otras posiciones de memoria.
    \end{itemize}

  \mode<presentation>{\vfill\pause}
  \item Existen distintos modelos de consistencia de memoria.
    \begin{itemize}
      \item \textbad{Dedicaremos una sesión específica a este problema}.
    \end{itemize}

\end{itemize}
\end{frame}
