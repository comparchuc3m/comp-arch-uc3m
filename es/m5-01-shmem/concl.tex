\section{Conclusión}

\begin{frame}[t]{Resumen}
\begin{itemize}
  \item \textemph{Multiprocesador} como computador con \textmark{procesadores altamente acoplados}
        con coordinación, uso y compartición de memoria.

  \mode<presentation>{\vfill\pause}
  \item \textemph{Multiprocesadores} clasificados en 
        \textmark{SMP} (Symetric multiprocessor) y
        \textmark{DSM} (Distributed Shared Memory).

  \mode<presentation>{\vfill\pause}
  \item Dos \textmark{aspectos} a considerar en la jerarquía de memoria: 
        \textgood{coherencia} y \textgood{consistencia}.

  \mode<presentation>{\vfill\pause}
  \item Dos alternativas en la \textmark{coherencia de caché}: 
        \textgood{directorio} y \textgood{espionaje} (\emph{snooping}).

  \mode<presentation>{\vfill\pause}
  \item Los \textemph{protocolos de espionaje} \textbad{no requieren} un 
        \textgood{elemento centralizado}.
    \begin{itemize}
      \item Pero generan más tráfico de bus.
    \end{itemize}
\end{itemize}
\end{frame}

\begin{frame}[t]{Referencias}
\begin{itemize}
  \item \bibhennessy
    \begin{itemize}
      \item Sección 5.1 -- \emph{Introduction}
      \item Sección 5.2 -- \emph{Centralized Shared-Memory Architectures}
      \item Sección 5.3 -- \emph{Performance of Symmetric Shared-Memory Multiprocessors}
            (solamente páginas 393--394).
    \end{itemize}

\end{itemize}
\end{frame}
