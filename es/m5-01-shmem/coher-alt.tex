\section{Alternativas de coherencia de caché}

\begin{frame}[t]{Multiprocesadores coherentes}
\begin{itemize}
  \item Un \textgood{multiprocesador coherente} ofrece:
    \begin{itemize}
      
      \mode<presentation>{\vfill\pause}
      \item \textmark{Migración} de datos compartidos.
        \begin{itemize}
          \item Un dato puede moverse a una caché local y usarse de forma transparente.
          \item Reduce latencia de acceso a dato remoto y demanda de ancho de banda a la memoria compartida.
        \end{itemize}
      
      \mode<presentation>{\vfill\pause}
      \item \textmark{Replicación} de datos compartidos leídos simultáneamente.
        \begin{itemize}
          \item Se realiza copia del dato en caché local.
          \item Reduce latencia de acceso y contención de las lecturas.
        \end{itemize}
    \end{itemize}

  \mode<presentation>{\vfill\pause}
  \item \textgood{Propiedades críticas para el rendimiento}:
    \begin{itemize}
      \item \textmark{Solución}: Protocolo hardware de mantenimiento de coherencia de caché.
    \end{itemize}
\end{itemize}
\end{frame}

\begin{frame}[t]{Clases de protocolos de coherencia de caché}
\begin{itemize}
  \item Basados en \textgood{directorio}:
    \begin{itemize}
      \item El estado de compartición se mantiene en un directorio.
      \item \textmark{SMP}: Directorio centralizado en memoria o en caché de más alto nivel.
      \item \textmark{DSM}: Para evitar cuello de botella se usa un directorio distribuido (más complejo).
    \end{itemize}

  \mode<presentation>{\vfill\pause}
  \item \textgood{Snooping} (espionaje):
    \begin{itemize}
      \item Cada caché mantiene el estado de compartición de cada bloque que tiene.
      \item Las cachés accesibles mediante medio de multidifusión (bus).
      \item Todas las cachés monitorizan el medio de multidifusión para determinar si tienen una copia del bloque.
    \end{itemize}
\end{itemize}
\end{frame}
