\section{Introducción a las arquitecturas multiprocesador}

\begin{frame}[t]{Creciente importancia de multiprocesadores}
\begin{itemize}
  \item Caída de la eficiencia en uso de silicio y energía al explotar mayor nivel de ILP.
    \begin{itemize}
      \item El coste de silicio y energía crece más rápidamente que el rendimiento.
    \end{itemize}

  \mode<presentation>{\vfill\pause}
  \item Interés creciente en servidores de alto rendimiento.
    \begin{itemize}
      \item Cloud computing, \emph{software as a service}, \ldots
    \end{itemize}

  \mode<presentation>{\vfill\pause}
  \item Crecimiento de aplicaciones intensivas en datos.
    \begin{itemize}
      \item Enormes cantidades de datos en Internet.
      \item \emph{Big data analytics}.
    \end{itemize}

\end{itemize}
\end{frame}

\begin{frame}[t]{TLP: Paralelismo a nivel de hilo}
\begin{itemize}
  \item TLP implica la existencia de múltiples contadores de programa.

    \begin{itemize}
      \mode<presentation>{\vfill}
      \item Asume MIMD.

      \mode<presentation>{\vfill}
      \item Uso generalizado de TLP fuera de computación científica relativamente reciente.

      \mode<presentation>{\vfill}
      \item Nuevas aplicaciones:
        \begin{itemize}
          \item Aplicaciones empotradas.
          \item Desktop.
          \item Servidores de alta gama.
        \end{itemize}
    \end{itemize}
\end{itemize}
\end{frame}

\begin{frame}[t]{Multiprocesadores}
\begin{itemize}
  \item Un \textgood{multiprocesador} es un computador formado por procesadores altamente acoplados con:
    \begin{itemize}
      \item \pause\textgood{Coordinación y uso} típicamente controlados 
            por un \textmark{sistema operativo único}.
      \item \textgood{Compartición de memoria} mediante un 
            \textmark{único espacio de direcciones compartido}.
    \end{itemize}

  \mode<presentation>{\vfill\pause}
  \item \textgood{Modelos de software}:
    \begin{itemize}
      \item \pause\textmark{Procesamiento paralelo}: Conjunto de hilos acoplados que cooperan.
      \item \pause\textmark{Procesamiento de peticiones}: Ejecución de procesos independientes originados por usuarios.
      \item \pause\textmark{Multiprogramación}: Ejecución independiente de múltiples aplicaciones.
    \end{itemize}

\end{itemize}
\end{frame}

\begin{frame}[t]
\begin{itemize}
  \item La aproximación más común:
    \begin{itemize}
      \item De 2 a decenas de procesadores.
      \item Memoria compartida.
        \begin{itemize}
          \item Implica memoria compartida.
          \item No necesariamente implica una única memoria física.
        \end{itemize}
    \end{itemize}

  \mode<presentation>{\vfill\pause}
  \item \textgood{Alternativas}:
    \begin{itemize}
      \item \textmark{CMP} (\emph{Chip MultiProcessors}) o \emph{multi-core}.
      \item Múltiples chips.
        \begin{itemize}
          \item Cada uno puede ser (o no) \emph{multi-core}.
        \end{itemize}
      \item \textmark{Multicomputador}: Procesadores débilmente acoplados que no comparten memoria.
        \begin{itemize}
          \item Usados en computación científica de gran escala.
        \end{itemize}
    \end{itemize}
\end{itemize}
\end{frame}

\begin{frame}[t]
\begin{itemize}
  \item \textgood{Aprovechamiento} de un multiprocesador:
    \begin{itemize}
      \item Con \textmark{n} procesadores se necesitan \textmark{n} hilos o procesos.
    \end{itemize}

  \mode<presentation>{\vfill\pause}
  \item \textgood{Identificación} de hilos:
    \begin{itemize}
      \item Identificados explícitamente por programador.
      \item Creados por el sistema operativo a partir de peticiones.
      \item Iteraciones de un bucle generadas por compilador paralelo (p. ej. OpenMP).
    \end{itemize}
\end{itemize}

\mode<presentation>{\vfill\pause}
\setbeamercolor{blockremark}{bg=cyan!60!black}
\begin{beamercolorbox}[sep=1em,wd=\textwidth]{blockremark}
Identificación realizada a alto nivel por el programador o software de sistema
con hilos con un número \textmark{suficiente} de instrucciones a ejecutar.
\end{beamercolorbox}

\end{frame}

\begin{frame}[t,shrink=10]{Multiprocesadores y memoria compartida}

\vspace{-1em}
\begin{columns}[T]

\column{.5\textwidth}

\begin{block}{SMP: Symmetric Multi-Processor}
\begin{itemize}
  \item Memoria compartida centralizada.
  \item Comparten una memoria centralizada única a la que todos acceder por igual.
  \item Todos los multi-core son SMP.
  \item \textmark{UMA}: Uniform Memory Access
    \begin{itemize}
      \item La latencia de memoria es uniforme.
      \item Todos los SMP son UMA.
    \end{itemize}
  \item textmark{NUCA}: Non-Uniform Cache Access.
    \begin{itemize}
      \item Acceso no uniforme a caché LL.
      \item Ejemplo: IBM Power.
    \end{itemize}
\end{itemize}
\end{block}

\pause
\column{.5\textwidth}

\begin{block}{DSM: Distributed Shared Memory}
\begin{itemize}
  \item Memoria compartida distribuida
  \item La memoria se distribuye entre los procesadores.
  \item Necesaria cuando hay muchos procesadores.
  \item \textmark{NUMA}: Non Uniform Memory Access.
    \begin{itemize}
      \item La latencia depende de la ubicación del dato accedido.
    \end{itemize}
\end{itemize}
\end{block}

\end{columns}

\mode<presentation>{\vfill\pause}

\begin{itemize}
  \item \textgood{Comunicación} mediante acceso a 
        \textmark{espacio compartido de direcciones}.
\end{itemize}
\end{frame}

\begin{frame}[t]{SMP: Symmetric MultiProcessor}
\makebox[\textwidth][c]{\input{es/m5-01-shmem/smp.tkz}}
\end{frame}

\begin{frame}[t]{DSM: Distributed Shared Memory}
\makebox[\textwidth][c]{\input{es/m5-01-shmem/dsm.tkz}}
\end{frame}

\begin{frame}[t]{Desafíos del procesamiento paralelo}
\begin{itemize}
  \item \textbad{Actividad}
    \begin{enumerate}
      \item \textemph{Lee} la \textmark{sección 5.1} -- Introduction.
            Challenges of Parallel Processing (pg. 373--376).
        \begin{itemize}
          \item \bibhennessy
        \end{itemize}

      \mode<presentation>{\vfill}
      \item \textemph{Presta atención} a los ejemplos numéricos.

      \mode<presentation>{\vfill}
      \item \textgood{Aspectos clave}:
        \begin{itemize}
          \item ¿Cuáles son los dos mayores desafíos de rendimiento para multiprocesadores?
        \end{itemize}
    \end{enumerate}
\end{itemize}
\end{frame}

