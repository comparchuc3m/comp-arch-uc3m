\section{Protocolos de espionaje}

\begin{frame}[t]{Mantenimiento de la coherencia}
\begin{itemize}
  \item \textgood{Invalidación de escrituras}:
    \begin{itemize}
      \item Garantiza que un procesador tiene \textmark{acceso exclusivo} 
            a un bloque antes de realizar una escritura.
      \item \textmark{Invalida} el resto de copias que puedan tener otros procesadores.
    \end{itemize}

  \mode<presentation>{\vfill\pause}
  \item \textgood{Actualización de escrituras} (difusión de escrituras):
    \begin{itemize}
      \item Difunde todas las escrituras a \textmark{todas las cachés} para modificar el bloque.
      \item Consume \textmark{más ancho de banda}.
    \end{itemize}

  \mode<presentation>{\vfill\pause}
  \item Estrategia más común $\Rightarrow$ \textmark{Invalidación}.
\end{itemize}
\end{frame}

\begin{frame}[t]{Uso del bus de memoria}
\begin{itemize}
  \item \textgood{Invalidación}.
    \begin{itemize}
      \item El procesador adquiere el bus y difunde la dirección a invalidar.
      \item Todos los procesadores espían el bus.
      \item Cada procesador comprueba si tienen en caché la dirección difundida y la invalidan.
    \end{itemize}

  \mode<presentation>{\vfill\pause}
  \item No puede haber dos escrituras simultáneas:
    \begin{itemize}
      \item El uso exclusivo del bus serializa las escrituras.
    \end{itemize}

  \mode<presentation>{\vfill\pause}
  \item \textgood{Fallos de caché}:
    \begin{itemize}
      \item \textmark{Escritura inmediata} (write through):
        \begin{itemize}
          \item La memoria tiene la última escritura realizada.
        \end{itemize}
      \item \textmark{Post-escritura} (write back):
        \begin{itemize}
          \item Si un procesador tiene una copia modificada, contesta al falló de caché del otro procesador.
        \end{itemize}
    \end{itemize}
\end{itemize}
\end{frame}

\begin{frame}[t]{Implementación}
\begin{itemize}
  \item \textgood{Invalidación}:
    \begin{itemize}
      \item Se aprovecha el \textmark{bit de validez} (V) asociado a cada bloque.
    \end{itemize}

  \mode<presentation>{\vfill\pause}
  \item \textgood{Escrituras}:
    \begin{itemize}
      \item Necesidad de saber si hay \textmark{otras copias} en caché.
        \begin{itemize}
          \item Si no hay otras copias no hay que \textmark{difundir escritura}.
        \end{itemize}

      \mode<presentation>{\vfill\pause}
      \item Se añade \textmark{bit de compartición} (S) asociado a cada bloque.

      \mode<presentation>{\vfill\pause}
      \item Cuando hay escritura:
        \begin{itemize}
          \item Se genera \textmark{invalidación} en bus.
          \item Se pasa de \textmark{estado compartido} a \textmark{estado exclusivo}.
          \item No hace falta enviar nuevas invalidaciones.
        \end{itemize}

      \mode<presentation>{\vfill\pause}
      \item Cuando hay \textmark{fallo de caché} en otro procesador:
        \begin{itemize}
          \item Se pasa de \textmark{estado exclusivo} a \textmark{estado compartido}.
        \end{itemize}
    \end{itemize}
\end{itemize}
\end{frame}

\begin{frame}[t]{Protocolo básico}
\begin{itemize}
  \item Basado en una \textgood{máquina de estados} para \textmark{cada bloque de caché}:
    \begin{itemize}
      \item \textgood{Cambios de estado} generados por:
        \begin{itemize}
          \item Peticiones del procesador.
          \item Peticiones del bus.
        \end{itemize}
      \item \textgood{Acciones}:
        \begin{itemize}
          \item Cambios de estado.
          \item Acciones sobre el bus.
        \end{itemize}
    \end{itemize}

  \mode<presentation>{\vfill\pause}
  \item Aproximación simple con \textgood{tres estados}:
    \begin{itemize}
      \item \textmark{M}: El bloque ha sido modificado.
      \item \textmark{S}: El bloque está compartido.
      \item \textmark{I}: El bloque ha sido invalidado.
    \end{itemize}
\end{itemize}
\end{frame}

\begin{frame}[t,shrink=5]{Acciones generadas por el procesador}

{\scriptsize
\begin{tabular}{l|l|l|l}

Petición & Estado & Acción & Descripción \\
\hline
\hline

\pause Acierto de lectura &
S $\rightarrow$ S &
Acierto &
Leer dato de caché local
\\
\hline

\pause Acierto de lectura &
M $\rightarrow$ M &
Acierto &
Leer dato de caché local
\\
\hline

\pause Fallo de lectura & 
I $\rightarrow$ S &
Fallo &
Difundir fallo de lectura en bus.
\\
\hline

\pause Fallo de lectura &
S $\rightarrow$ S &
Remplazo &
Fallo de conflicto de dirección. 
\\

& & &
Difundir fallo de lectura en bus. 
\\
\hline

\pause Fallo de lectura &
M $\rightarrow$ S &
Remplazo &
Fallo de conflicto de dirección. 
\\

& & &
Escribir bloque y difundir fallo de lectura.
\\
\hline

\pause Acierto de escritura &
M $\rightarrow$ M &
Acierto &
Escribir dato en caché local.
\\
\hline

\pause Acierto de escritura &
S $\rightarrow$ M &
Coherencia &
Invalidación en bus.
\\
\hline

\pause Fallo de escritura &
I $\rightarrow$ M &
Fallo &
Difundr fallo de escritura en bus.
\\
\hline

\pause Fallo de escritura &
S $\rightarrow$ M &
Remplazo &
Fallo de conflicto de dirección. 
\\

& & &
Difundir fallo de escritura en bus.
\\
\hline

\pause Fallo de escritura &
M $\rightarrow$ M &
Remplazo &
Fallo de conflicto de dirección. 
\\

& & &
Escribir bloque y difundir fallo de escritura.
\\
\hline

\end{tabular}
}
\end{frame}


\begin{frame}[t]{Protocolo MSI: Acciones de procesador}
\makebox[\textwidth][c]{
\input{es/m5-01-shmem/msi-proc.tkz}
}
\end{frame}


\begin{frame}[t,shrink=5]{Acciones generadas por el bus}

{\scriptsize
\begin{tabular}{l|l|l|l}

Petición & Estado & Acción & Descripción\\
\hline
\hline

\pause Fallo de lectura &
S $\rightarrow$ S &
-- &
Memoria compartida sirve el fallo
\\
\hline

\pause Fallo de lectura &
M $\rightarrow$ S &
Coherencia &
Intento de compartir dato.
\\

& & &
Se coloca bloque en bus.
\\
\hline

\pause Invalidar &
S $\rightarrow$ I &
Coherencia &
Intento de escribir bloque compartido.
\\

& & & 
Invalidar el bloque.
\\
\hline

\pause Fallo de escritura &
S $\rightarrow$ I &
Coherencia &
Intento de escribir bloque compartido.
\\

& & &
Invalidar el bloque.
\\
\hline

\pause Fallo de escritura &
M $\rightarrow$ I &
Coherencia &
Intento de escribir bloque que es.
\\

& & &
exclusivo en algún sitio.
\\

& & &
Se escribe (\emph{write-back}) el bloque.
\\
\hline

\end{tabular}
}
\end{frame}

\begin{frame}[t]{Protocolo MSI: Acciones de bus}
\makebox[\textwidth][c]{
\input{es/m5-01-shmem/msi-bus.tkz}
}
\end{frame}

\begin{frame}[t]{Complejidad del protocolo MSI}
\begin{itemize}
  \item El protocolo asume que las operaciones son \textmark{atómicas}.
    \begin{itemize}
      \item \textgood{Ejemplo}: Se asume que se puede detectar un fallo de escritura, 
            adquirir el bus y recibir una respuesta en una única acción sin interrupción.
    \end{itemize}

  \mode<presentation>{\vfill\pause}
  \item Si las operaciones \textbad{no son atómicas}:
    \begin{itemize}
      \item Posibilidad de deadlock y/o carreras.
    \end{itemize}

  \mode<presentation>{\vfill\pause}
  \item \textgood{Solución}:
    \begin{itemize}
      \item El procesador que envía invalidación mantiene 
            \textmark{propiedad del bus} hasta que la invalidación 
            llega al resto.
    \end{itemize}
\end{itemize}
\end{frame}

\begin{frame}[t]{Extensiones a MSI}
\begin{itemize}
  \pause
  \item \textgood{MESI}:
    \begin{itemize}
       \item Añade \textmark{estado exclusivo} (E) que indica 
             que el bloque reside en una única caché pero no está modificado.
       \item Escritura de un bloque E \textgood{no genera invalidaciones}.
    \end{itemize}

  \mode<presentation>{\vfill\pause}
  \item \textgood{MESIF}:
    \begin{itemize}
      \item Añade \textmark{estado forward} (F): Alternativa a S que indica qué nodo debe responder a una petición.
      \item Usado en Intel Core i7.
    \end{itemize}

  \mode<presentation>{\vfill\pause}
  \item \textgood{MOESI}:
    \begin{itemize}
      \item Añade \textmark{estado poseído} (O) que indica que el bloque está desactualizado en memoria.
      \item Evita escrituras a memoria.
      \item Usado en AMD Opteron.
    \end{itemize}
\end{itemize}
\end{frame}
