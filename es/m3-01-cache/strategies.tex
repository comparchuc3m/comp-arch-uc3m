\section{Políticas y estrategias}

\begin{frame}[t]{Cuatro preguntas sobre la jerarquía de memoria}
\begin{enumerate}
  \item ¿Dónde se ubica un bloque en el nivel superior?
    \begin{itemize}
      \item \textmark{Ubicación de bloque}.
    \end{itemize}

  \mode<presentation>{\vfill\pause}
  \item ¿Cómo se localiza un bloque en el nivel superior?
    \begin{itemize}
      \item \textmark{Identificación de bloque}.
    \end{itemize}

  \mode<presentation>{\vfill\pause}
  \item ¿Qué bloque debe remplazarse en caso de fallo?
    \begin{itemize}
      \item \textmark{Remplazo de bloque}.
    \end{itemize}


  \mode<presentation>{\vfill\pause}
  \item ¿Qué ocurre en caso de escritura?
    \begin{itemize}
      \item \textmark{Estrategia de escritura}.
    \end{itemize}

\end{enumerate}
\end{frame}

\begin{frame}[t,shrink=10]{P1: Ubicación de bloque}
\begin{itemize}
  \item \textgood{Correspondencia directa}.
    \begin{itemize}
      \item Caché organizada líneas de caché.
      \item Un bloque de memoria se ubica en una línea determinada por su dirección.
      \item Ubicación $\rightarrow$ $bloque \mod n_{bloques}$
    \end{itemize}

  \mode<presentation>{\vfill\pause}
  \item \textgood{Correspondencia totalmente asociativa}.
    \begin{itemize}
      \item Cada bloque de memoria tiene asociada una etiqueta única.
      \item Cada línea de la caché guarda la etiqueta del bloque que contiene.
      \item Un bloque de memoria se ubica en cualquier línea de la caché.
      \item Ubicación $\rightarrow$ Cualquier lugar.
    \end{itemize}

  \mode<presentation>{\vfill\pause}
  \item \textgood{Correspondencia asociativa por conjuntos}.
    \begin{itemize}
      \item Las líneas de la caché se organizan en conjuntos.
      \item Cada conjunto tiene varias líneas.
      \item Selección de conjunto determinada por dirección de bloque 
            $\rightarrow$ $bloque \mod n_{conjuntos}$
      \item Ubicación dentro de conjunto determinada por etiqueta
            $\rightarrow$ Cualquier lugar de conjunto.
    \end{itemize}
\end{itemize}
\end{frame}

\begin{frame}[t]{Razones para los fallos de caché}
\begin{itemize}
  \item Un fallo de caché puede deberse a alguna de las siguientes razones:

    \begin{itemize}
    
      \mode<presentation>{\vfill\pause}
      \item \textmark{Obligatorio} (\emph{Compulsory}):
        \begin{itemize}
          \item Primer acceso a una dirección de memoria de un bloque.
          \item Ocurrirían incluso con caché de tamaño infinito.
        \end{itemize}

      \mode<presentation>{\vfill\pause}
      \item \textmark{Capacidad} (\emph{Capacity}):
        \begin{itemize}
          \item Acceso a un bloque que ha sido expulsado de la caché.
          \item Podría no ocurrir en caché de mayor tamaño.
        \end{itemize}

      \mode<presentation>{\vfill\pause}
      \item \textmark{Conflicto} (\emph{Conflict}):
        \begin{itemize}
          \item Acceso a un bloque que ha sido expulsado porque le corresponde 
                el mismo conjunto/línea que otro.
          \item Solamente en caché que no sean totalmente asociativas.
        \end{itemize}
    \end{itemize}
\end{itemize}
\end{frame}

\begin{frame}[t]{P2: Identificación de bloque}
\begin{itemize}
  \item \textgood{Dirección de bloque}:
    \begin{itemize}
      \item \textmark{Etiqueta}: Identifica la dirección en entrada.
        \begin{itemize}
          \item Bit validez en cada entrada indica el contenido es válido.
        \end{itemize}
      \item \textmark{Índice}: Selecciona el conjunto.
    \end{itemize}

  \mode<presentation>{\vfill\pause}
  \item \textgood{Desplazamiento de bloque}:
    \begin{itemize}
      \item Selecciona datos dentro de un bloque.
    \end{itemize}

  \mode<presentation>{\vfill\pause}
  \item \textmark{Con mayor asociatividad}:
    \begin{itemize}
      \item Menos bit de \textmark{índice}.
      \item Más bits de \textmark{etiqueta}.
    \end{itemize}
\end{itemize}
\end{frame}

\begin{frame}[t]{P3: Remplazo de bloque}
\begin{itemize}
  \item \textgood{Relevante} para \textmark{asociativa} y 
        \textmark{asociativa por conjuntos}:
    \begin{itemize}
      \item \textmark{Aleatorio}.
        \begin{itemize}
          \item Fácil de implementar.
        \end{itemize}
      \item \textmark{LRU}: Menos recientemente usado.
        \begin{itemize}
          \item Complejidad creciente cuando aumenta asociatividad.
        \end{itemize}
      \item \textmark{FIFO}.
        \begin{itemize}
          \item Aproxima LRU con menor complejidad.
        \end{itemize}
    \end{itemize}
\end{itemize}

\end{frame}

\begin{frame}[t]{P4: Estrategia de escritura}
\begin{columns}[T]

\begin{column}{.5\textwidth}
\begin{block}{Escritura inmediata (\emph{write-through})}
\begin{itemize}
  \item Todas las escrituras van a bus y memoria.
  \item Fácil de implementar.
  \item Problemas de rendimiento en SMP’s
\end{itemize}
\end{block}
\end{column}

\pause
\begin{column}{.5\textwidth}
\begin{block}{Post-escritura (\emph{write-back})}
\begin{itemize}
  \item Muchas escrituras son aciertos.
  \item Aciertos de escritura \alert{no} van a bus y memoria.
  \item Problemas de propagación y serialización.
  \item Más complejo.
\end{itemize}
\end{block}
\end{column}

\end{columns}
\end{frame}

\begin{frame}[t,shrink=10]{P4: Estrategia de escritura}
\begin{itemize}
  \item \textgood{¿Donde se escribe?}:
    \begin{itemize}
      \item \textmark{write-trhough}: En bloque de caché y siguiente nivel de memoria.
      \item \textmark{write-back}: Solamente en bloque de caché.
    \end{itemize}

  \mode<presentation>{\vfill\pause}
  \item \textgood{¿Qué ocurre cuándo bloque sale de caché?}:
    \begin{itemize}
      \item \textmark{write-trhough}: Nada más.
      \item \textmark{write-back}: Se actualiza siguiente nivel de memoria.
    \end{itemize}

  \mode<presentation>{\vfill\pause}
  \item \textgood{Depuración}:
    \begin{itemize}
      \item \textmark{write-trhough}: Fácil.
      \item \textmark{write-back}: Difícil.
    \end{itemize}

  \mode<presentation>{\vfill\pause}
  \item \textgood{Fallo provoca escritura}:
    \begin{itemize}
      \item \textmark{write-trhough}: No.
      \item \textmark{write-back}: Si.
    \end{itemize}

  \mode<presentation>{\vfill\pause}
  \item \textgood{Escritura repetida va al siguiente nivel}:
    \begin{itemize}
      \item \textmark{write-trhough}: Si.
      \item \textmark{write-back}: No.
    \end{itemize}
\end{itemize}
\end{frame}

\begin{frame}[t]{Búfer de escritura}
\makebox[\textwidth][c]{
\input{es/m3-01-cache/write-buffer.tkz}
}

\begin{columns}[T]

\begin{column}{.5\textwidth}
\begin{itemize}
  \item \textmark{¿Por qué un búfer?}
    \begin{itemize}
      \item Para evitar detenciones de CPU.
    \end{itemize}
  \item \textmark{¿Por qué un búfer y no un registro?}
    \begin{itemize}
      \item Ráfagas de escritura son comunes.
    \end{itemize}
\end{itemize}
\end{column}

\begin{column}{.5\textwidth}
\begin{itemize}
  \item \textmark{¿Puede haber riesgos RAW?}
    \begin{itemize}
      \item Si.
    \end{itemize}
  \item \textgood{Alternativas}:
    \begin{itemize}
      \item \textmark{Write allocate}: Se asigna un bloque en caché.
      \item \textmark{No-write allocate}: Se modifica bloque en siguiente nivel.
    \end{itemize}
\end{itemize}
\end{column}

\end{columns}
\end{frame}

\begin{frame}[t]{Penalización de fallo}
\begin{itemize}
  \item \textgood{Penalización de fallo}:
    \begin{itemize}
      \item Latencia total del fallo.
      \item Latencia expuesta (la que obliga a detener CPU).
    \end{itemize}
\end{itemize}
\mode<presentation>{\vfill}
\begin{block}{Penalización de fallos}
\[
\frac{\text{ciclos\_detención}_{memoria}}{IC} =
\]
\[
\frac{fallos}{IC} \times \left( latencia_{total} - latencia_{solapada} \right)
\]
\end{block}
\end{frame}
