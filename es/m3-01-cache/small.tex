\subsection{Cachés pequeñas y simples}

\begin{frame}[t]{Cachés pequeñas}
\begin{itemize}
  \item Proceso de \textgood{búsqueda}:
    \begin{itemize}
      \item Selección de línea utilizando índice.
      \item Lectura de etiqueta de línea.
      \item Comparación con etiqueta de dirección.
    \end{itemize}

  \mode<presentation>{\vfill\pause}
  \item El tiempo de búsqueda se \textbad{incrementa} con 
        el tamaño de la caché.
  
  \mode<presentation>{\vfill}
  \item Una caché \textmark{más pequeña} permite:
    \begin{itemize}
      \item Hardware de búsqueda más simple.
      \item La caché cabe en el chip del procesador.
    \end{itemize}

  \mode<presentation>{\vfill\pause}
  \item \textgood{Una caché pequeña mejora el tiempo de búsqueda}.
\end{itemize}
\end{frame}

\begin{frame}[t]{Cachés simples}
\begin{itemize}
  \item Simplificación de la caché.
    \begin{itemize}
      \item Uso de mecanismos de correspondencia lo más \textgood{sencillo} posibles.
      \item \textgood{Correspondencia directa}:
        \begin{itemize}
          \item Se permite \textmark{paralelizar} la comparación de 
                la etiqueta con la transmisión del dato.
        \end{itemize}
    \end{itemize}

  \mode<presentation>{\vfill\pause}
  \item \textmark{Observación}:
        La mayoría de los procesadores modernos se centran más 
        en usar cachés pequeñas que en su simplificación.
\end{itemize}
\end{frame}

\begin{frame}[t]{Intel Core i7}
\begin{itemize}
  \item Caché L1 (1 por núcleo)
    \begin{itemize}
      \item 32 KB instrucciones
      \item 32 KB datos
      \item Latencia: 3 ciclos
      \item Asociativa 4(i), 8(d) vías. 
    \end{itemize}

  \mode<presentation>{\vfill}
  \item Caché L2 (1 por núcleo)
    \begin{itemize}
      \item 256 KB
      \item Latencia: 9 ciclos
      \item Asociativa 8 vías.
    \end{itemize}

  \mode<presentation>{\vfill}
  \item Caché L3 (compartida)
    \begin{itemize}
      \item 8 MB
      \item Latencia: 39 ciclos
      \item Asociativa 16 vías.
    \end{itemize}
\end{itemize}
\end{frame}
