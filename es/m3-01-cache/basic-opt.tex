\section{Optimizaciones básicas}

\begin{frame}[t]{Optimizaciones básicas de caché}
\begin{itemize}
  \item \textgood{Reducción} de \textmark{tasa de fallos}.
    \begin{itemize}
      \item Aumento de tamaño de bloques.
      \item Aumento de tamaño de caché.
      \item Incremento de asociatividad.
    \end{itemize}

  \mode<presentation>{\vfill}
  \item \textgood{Reducción} de \textmark{penalización de fallos}.
    \begin{itemize}
      \item Cachés multinivel.
      \item Prioridad de lecturas sobre escrituras.
    \end{itemize}

  \mode<presentation>{\vfill}
  \item \textgood{Reducción} de \textmark{tiempo de acierto}.
    \begin{itemize}
      \item Evitar traducción de direcciones en indexado en caché.
    \end{itemize}

\end{itemize}
\end{frame}

\subsection{Reducción de tasa de fallos}

\begin{frame}[t]{1: Aumento de tamaño de bloque}
\begin{itemize}
  \item Mayor tamaño de bloque $\rightarrow$ \textgood{menor la tasa de fallos}.
    \begin{itemize}
       \item Mejor aprovechamiento de localidad espacial.
    \end{itemize}

  \mode<presentation>{\vfill\pause}
  \item Mayor tamaño de bloque $\rightarrow$ \textbad{mayor penalización por fallo}.
    \begin{itemize}
      \item En caso de fallo hay que transferir bloques más grandes.
      \item Más fallos porque la caché tiene menos bloques.
    \end{itemize}

  \mode<presentation>{\vfill\pause}
  \textgood{Necesidad de equilibrio}:
    \begin{itemize}
      \item Memoria de alta latencia y alto ancho de banda:
        \begin{itemize}
          \item Incrementar tamaño de bloque.
        \end{itemize}
      \item Memoria con baja latencia y bajo ancho de banda:
        \begin{itemize}
          \item Tamaño de bloque reducido.
        \end{itemize}
    \end{itemize}
\end{itemize}
\end{frame}

\begin{frame}[t]{2: Aumento de tamaño de la caché}
\begin{itemize}
  \item Mayor tamaño de caché $\rightarrow$ \textgood{menor tasa de fallos}.
    \begin{itemize}
      \item Caben más datos en la caché.
    \end{itemize}

  \mode<presentation>{\vfill\pause}
  \item Puede \textbad{incrementar el tiempo de acierto}.
    \begin{itemize}
      \item Hace falta más tiempo para encontrar el bloque.
    \end{itemize}

  \mode<presentation>{\vfill\pause}
  \item \textbad{Mayor coste}.

  \mode<presentation>{\vfill\pause}
  \item \textbad{Mayor consumo de energía}.

  \mode<presentation>{\vfill\pause}
  \item \textgood{Necesidad de encontrar un equilibrio}.
    \begin{itemize}
      \item Especialmente en cachés on-chip.
    \end{itemize}
\end{itemize}
\end{frame}

\begin{frame}[t]{3: Incremento de asociatividad}
\begin{columns}[T]

\column{.55\textwidth}

\begin{itemize}
  \item Mayor asociatividad $\rightarrow$ \textgood{menor tasa de fallos}.
    \begin{itemize}
      \item Hay menos conflictos al poderse usar más vías en un mismo conjunto.
    \end{itemize}

  \mode<presentation>{\vfill\pause}
  \item Puede \textbad{incrementar tiempo de acierto}.
    \begin{itemize}
      \item Hace falta más tiempo para encontrar el bloque.
    \end{itemize}

  \mode<presentation>{\vfill\pause}
  \item \textgood{Consecuencia}:
    \begin{itemize}
      \item 8 vías $\approx$ Totalmente asociativa.
    \end{itemize}
\end{itemize}

\column{.45\textwidth}

\end{columns}
\end{frame}

\subsection{Reducción de penalización de fallos}

\begin{frame}[t]{4: Cachés multinivel}
\begin{itemize}
  \item \textmark{Objetivo}: \textgood{Reducción de la penalización de fallos}.

  \mode<presentation>{\vfill\pause}
  \item \textgood{Evolución}:
    \begin{itemize} 
      \item \textbad{Mayor distancia} entre rendimiento de DRAM y CPU a lo largo del tiempo.
      \item \textbad{Coste} de penalización de fallo \textbad{creciente}.
    \end{itemize}

  \mode<presentation>{\vfill\pause}
  \item \textgood{Alternativas}:
    \begin{itemize} 
      \item Hacer la caché \textgood{más rápida}.
      \item Hacer la caché \textgood{más grande}.
    \end{itemize}

  \mode<presentation>{\vfill\pause}
  \item \textgood{Solución}:
    \begin{itemize}
      \item \textgood{¡Las dos cosas!}
      \item \textgood{Varios niveles de caché}.
    \end{itemize}
\end{itemize}
\end{frame}

\begin{frame}[t]{Tasa de fallos global y local}
\begin{itemize}
  \item \textmark{Tasa de fallos local}:
    \begin{itemize}
      \item Fallos en un nivel de la caché con respecto a accesos a ese nivel de caché.
      \item Tasa de fallos en L1 $\rightarrow$ $m(L1)$
      \item Tasa de fallos en L2 $\rightarrow$ $m(L2)$
    \end{itemize}

  \mode<presentation>{\vfill\pause}
  \item \textmark{Tasa de fallos global}:
    \begin{itemize}
      \item Fallos en un nivel de caché con respecto a todos los accesos a memoria.
      \item Tasa de fallos en L1 $\rightarrow$ $m(L1)$
      \item Tasa de fallos en L2 $\rightarrow$ $m(L1) \times m(L2)$
    \end{itemize}

  \mode<presentation>{\vfill\pause}
  \item \textmark{Tiempo medio de acceso}:
\end{itemize}
\[
t_a(L1) + m(L1) \times t_f(L1) =
\]
\[
t_a(L1) + m(L1) \times \left( t_a(L2) + m(L2) \times t_f(L2) \right)
\]
\end{frame}

\begin{frame}[t]{5: Prioridad de fallos de lectura sobre escritura}
\begin{itemize}
  \item \textmark{Objetivo}: \textgood{Reduce la penalización de fallos}.
    \begin{itemize}
      \item Evita que un fallo de lectura tenga que esperar a que 
            una escritura termine.
    \end{itemize}

  \mode<presentation>{\vfill\pause}
  \item \textmark{Cachés de escritura inmediata}.
    \begin{itemize}
      \item El búfer de escritura \textbad{podría contener una modificación}
            del dato que se lee.
        \begin{enumerate}[a)]
          \item Esperar el vaciado del búfer de escritura.
          \item Comprobar los valores del búfer de escritura.
        \end{enumerate}
    \end{itemize}

  \mode<presentation>{\vfill\pause}
  \item \textmark{Cachés con post-escritura}.
    \begin{itemize}
      \item Un fallo de lectura \textbad{podría remplazar un bloque modificado}.
        \begin{itemize}
          \item Copiar bloque modificado a búfer, leer y volcar bloque a memoria.
          \item Aplicar opciones \textgood{a} o \textgood{b} al búfer.
        \end{itemize}
    \end{itemize}
\end{itemize}
\end{frame}

