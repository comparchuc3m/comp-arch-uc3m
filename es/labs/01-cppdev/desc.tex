\section{Descripción del laboratorio}

En este laboratorio adquirirás experiencia con el entorno de programación que se utiliza en varias partes
de la asignatura.

\begin{itemize}
  \item Sistema operativo: Linux
  \item Compiladores de C++: g++-10 o superior (recomendado g++-11, g++-12, g++-13).
  \item IDE: CLion.
  \item Gestor de construcción: CMake
  \item Herramientas de análisis dinámico: valgrind-memcheck y \emph{sanitizers} del compilador.
\end{itemize}

Para llevar a cabo este laboratorio tienes las siguientes opciones:
\begin{itemize}
  \item Utilizar tu porpio portátil con Ubuntu/Linux como sistema operativo 
        u otra distribución de GNU/Linux.
  \item Utilizar tu propio portátil con MS Windows y WSL2. 
        Ten en cuenta que en este caso puedes tener divergencias menores en el rendimiento.
  \item Utilizar tu propio portátil con MacOS.
        Ten en cuenta que en este caso puedes tener divergencias menores en el rendimiento.
  \item También puede utilizar una máquina virtual con un sistema operativo GNU/Linux
        en este laboratorio, pero tenga en cuenta que en otros laboratorios de la
        asignatura esta aproximación podría no ser válida.
\end{itemize}
