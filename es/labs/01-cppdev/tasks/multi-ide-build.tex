\subsection{Uso de varios archivos fuente}

En esta sección crearemos un ejecutable con varios archivos fuente:

\begin{enumerate}

\item
Crea una carpeta llamada \cppid{src/areas}.

\item
Añade un archivo llamado \cppid{src/areas/geom.hpp}

\begin{lstlisting}
#ifndef AREAS_GEOM_HPP
#define AREAS_GEOM_HPP

namespace geom {

  double area(double w, double h);

}

#endif
\end{lstlisting}

\item 
Añade un archivo llamado \cppid{src/areas/geom.cpp}:

\begin{lstlisting}
#include "geom.hpp"

namespace geom {
  double area(double w, double h) {
    return w * h;
  }
}
\end{lstlisting}

\item
Añade un archivo \cppid{src/areas/main.cpp}:

\begin{lstlisting}
#include "geom.hpp"

#include <print>

int main() {
  double x = 2.0;
  double y = 3.0;
  std::println("area({}, {})= {}", x, y, geom::area(x, y));
}
\end{lstlisting}

\item
Modifica tu archivo \cppid{CMakeLists.txt} para añadir un segundo 
objetivo de compilación

\begin{lstlisting}
...

add_executable(areas)
target_sources(areas PRIVATE src/areas/geom.cpp src/areas/main.cpp)
\end{lstlisting}

\item
Ejecuta el programa.

\end{enumerate}
