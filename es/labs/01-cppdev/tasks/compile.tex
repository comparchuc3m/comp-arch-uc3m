\subsection{Compilación de un programa sencillo}

Para este ejercicio de compilación utilizaremos el archivo \cppid{hello.cpp} 
con el siguiente contenido:

\begin{lstlisting}
#include <print>

int main() {
  std::println("Hello C++");
}
\end{lstlisting}

Utilizaremos diversas opciones para observar diferentes comportamientos de la
interfaz de línea de comando del compilador
\textmark{g++}.

\begin{itemize}
  \item \cppid{g++ hello.cpp -E -o hello.ii}: 
        Ejecuta solamente el preprocesador y obtiene la salida en el archivo 
        \cppid{hello.ii}.
    \begin{itemize}
      \item Intenta determinar cuantas líneas de código C++ se generan después 
            de procesar \cppid{hello.cpp}.
    \end{itemize}
\begin{lstlisting}[style=terminal,escapechar=@]
root@a4794809b16b:/workspace@\#@ g++ src/hello.cpp -E hello.ii | wc -l
\end{lstlisting}


  \item \cppid{g++ hello.cpp -S -o hello.s}:
        Genera código ensamblador para la unidad de traducción actual.
    \begin{itemize}
      \item Intenta determinar cuantas líneas de código ensamblador se generan 
            para \cppid{hello.cpp}.
    \end{itemize}
\begin{lstlisting}[style=terminal,escapechar=@]
root@\@@a4794809b16b:/workspace@\#@ g++ src/hello.cpp -S -o hello.s
root@\@@a4794809b16b:/workspace@\#@ ls -l hello.s
-rw-r--r-- 1 root root 1403 Sep  3 13:16 hello.s
root@\@@a4794809b16b:/workspace@\#@ wc -l hello.s 
63 hello.s
root@\@@a4794809b16b:/workspace@\#@ g++ src/hello.cpp -S -o hello.s && wc -l hello.s 
63 hello.s
\end{lstlisting}

  \item \cppid{g++ hello.cpp -c -o hello.o}:
        Genera un archivo objeto \cppid{hello.o} que puede enlazarse con 
        otros archivos objeto.
    \begin{itemize}
      \item Intenta determinar cuantos bytes se requieren para el archivo \cppid{hello.o}.
    \end{itemize}
\begin{lstlisting}[style=terminal,escapechar=@]
root@\@@a4794809b16b:/workspace@\#@ g++ src/hello.cpp -c -o hello.s
root@\@@a4794809b16b:/workspace@\#@ ls -l hello.o
-rw-r--r-- 1 root root 1896 Sep  3 13:22 hello.o
\end{lstlisting}

  \item \cppid{g++ hello.cpp -o hello}:
        Genera un archivo ejecutable \cppid{hello}.
    \begin{itemize}
      \item Intenta determinar cuantos bytes se requieren para el 
            ejecutable binario \cppid{hello}.
\begin{lstlisting}[style=terminal,escapechar=@]
root@\@@a4794809b16b:/workspace@\#@ g++ src/hello.cpp -o hello
root@\@@a4794809b16b:/workspace@\#@ ls -l hello
-rwxr-xr-x 1 root root 16352 Sep  3 13:23 hello
root@\@@a4794809b16b:/workspace@\#@ ./hello
Hello C++
\end{lstlisting}
    \end{itemize}
\end{itemize}

Hay otras herramientas relacionadas que te podrían ser útiles como desarrollador de sistemas, pero que no se cubren en este laboratorio.
Intenta encontrar información sobre las siguientes:

\begin{itemize}
  \item \cppid{nm}
  \item \cppid{readelf}
  \item \cppid{objdump}
  \item \cppid{c++filt}
\end{itemize}

