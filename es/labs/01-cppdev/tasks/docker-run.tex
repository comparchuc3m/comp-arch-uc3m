\subsection{Ejecución de contenedor}

\subsubsection{Ejecución de un contenedor}

Puedes ejecutar un contenedor a partir de una imagen con el mandato 
\textemph{docker run}. Al hacerlo puedes indicar el mandato que quieres
ejecutar dentro del contenedor.

Ejecuta el siguiente mandato:

\begin{lstlisting}[style=terminal]
docker run --rm cpp-intro-lab g++ --version
\end{lstlisting}

Este mandato inicia un contenedor a partir de la image \textgood{cpp-intro-lab}
y ejecuta el mandato \textemph{g++ -{}-version}, para obtener por la salida del
terminal la versión del mandato \textemph{g++} (el compilador de GCC para el
lenguaje C++). La opción \textemph{-{}-rm} indica que una vez finalizado el
contenedor se debe eliminar.

\subsubsection{Ejecución interactiva de un contenedor}

También puedes ejecutar un intérprete de mandatos interactivo (opción
\textemph{-{}-it}) dentro del contenedor. Esto te permitirá ejecutar varios
mandatos sin abandonar el contenedor.

\begin{lstlisting}[style=terminal,escapechar=@]
@\$@ docker run --rm -it cpp-intro-lab
root@\@@b8521933590a:/workspace# g++ --version
g++ (Ubuntu 14.2.0-4ubuntu2~24.04) 14.2.0
Copyright (C) 2024 Free Software Foundation, Inc.
This is free software; see the source for copying conditions.  There is NO
warranty; not even for MERCHANTABILITY or FITNESS FOR A PARTICULAR PURPOSE.

root@\@@b8521933590a:/workspace# clang++ --version
Ubuntu clang version 20.1.8 (++20250804090239+87f0227cb601-1~exp1~20250804210352.139)
Target: x86_64-pc-linux-gnu
Thread model: posix
InstalledDir: /usr/lib/llvm-20/bin
root@\@@b8521933590a:/workspace# cmake --version
cmake version 4.1.1

CMake suite maintained and supported by Kitware (kitware.com/cmake).
root@\@@b8521933590a:/workspace# exit
exit
@\$@
\end{lstlisting}

\subsubsection{Compartición de archivos}

Cuando se ejecuta un contenedor no se comparte ningún archivo entre la máquina
anfitriona y el contenedor. Para hacerlo es necesario montar un volumen (opción
\textemph{-v}) al ejecutar el contenedor e indicar el directorio de trabajo.

\begin{lstlisting}[style=terminal,escapechar=@]
docker run --rm -it -v @\$@(pwd):/workspace cpp-intro-lab
\end{lstlisting}

Ten en cuenta que el directorio \textmark{/workspace} se define como directorio
de trabajo en el archivo \cppid{Dockerfile}.
