\subsection{Uso del entorno de desarrollo \textbf{VS Code}}

\begin{enumerate}

\item Desde el directorio raíz del proyecto inicia el entorno de desarrollo
\textmark{VS Code}.

\begin{lstlisting}[style=terminal,escapechar=@]
@\$@ code .
\end{lstlisting}

  \begin{itemize}
    \item Puede que aparezca un mensaje preguntando si confías en los
          autores de los archivos de la carpeta. Indica que sí confías.
  \end{itemize}

\item
Selecciona en el panel de la izquierda la opción extensiones (o utiliza el menú
\textmark{View | Extensions}). Asegúrate que instalas las siguientes extensiones:

  \begin{itemize}
    \item \textmark{Dev Containers}.
  \end{itemize}

\item
Selecciona la paleta de mandatos (menú \textmark{View | Command Palette}) y
selecciona la opción \textgood{Dev Containers: Rebuild and Reopen in Container}.

\item
Añade al espacio de trabajo un archivo denominado \textgood{CMakeLists.txt}. El
archivo debe contener las siguientes instrucciones:

\begin{lstlisting}
cmake_minimum_required(VERSION 4.0)
project(hello LANGUAGES CXX)

add_executable(hello)
target_sources(hello PRIVATE src/hello.cpp)
\end{lstlisting}

  \begin{itemize}
    \item Este archivo establece lo siguiente:
      \begin{itemize}
        \item Se requiere al menos la versión \textgood{4.0} de \textemph{cmake}.
        \item El proyecto se llama \textgood{hello} y está en \textgood{C++}.
        \item El proyecto genera el ejecutable \textemph{hello}.
        \item El ejecutable \textemph{hello} se genera a partir del 
              a partir del archivo \cppid{src/hello.cpp}.
      \end{itemize}
  \end{itemize}

\item
Añade al espacio de trabajo un archivo denominado \textgood{CMakePresets.json}. El
archivo debe contener las siguientes instrucciones:

\begin{lstlisting}
{
  "version": 9,
  "cmakeMinimumRequired": {
    "major": 3,
    "minor": 30,
    "patch": 0
  },
  "configurePresets": [
    {
      "name": "default",
      "description": "Default configuration",
      "generator": "Ninja Multi-Config",
      "binaryDir": "${sourceDir}/out/build/${presetName}",
      "installDir": "${sourceDir}/out/bin/${presetName}",
      "cacheVariables": {
        "CMAKE_CONFIGURATION_TYPES": "Release;Debug",
        "CMAKE_CXX_STANDARD": "23",
        "CMAKE_CXX_STANDARD_REQUIRED": "ON",
        "CMAKE_CXX_EXTENSIONS": "OFF",
        "CMAKE_CXX_SCAN_FOR_MODULES": "OFF",
        "CMAKE_EXPORT_COMPILE_COMMANDS": "YES",
        "CMAKE_CXX_FLAGS": "-Wall -Wextra -Werror -pedantic -pedantic-errors -Wconversion -Wsign-conversion"
      }
    }
  ],
  "buildPresets": [
    {
      "name": "debug",
      "configurePreset": "default",
      "configuration": "Debug"
    },
    {
      "name": "release",
      "configurePreset": "default",
      "configuration": "Release"
    }
  ]
}
\end{lstlisting}

  \begin{itemize}
    \item Este archivo establece lo siguiente:
      \begin{itemize}
        \item Se generarán dos configuraciones: 
              \textmark{Debug} y \textmark{Release}.
        \item Se compila en modo \textmark{C++23} sin extensiones.
        \item Se activan todos los flags de advertencias.
      \end{itemize}
  \end{itemize}

\item En la parte inferior de la pantalla puedes seleccionar los siguiente
      \emph{presets}:
  \begin{itemize}
    \item Configure: \textmark{default}.
    \item Build: \textmark{debug} o \textmark{release}.
  \end{itemize}

\item
Construye el programa mediante botón \textmark{Build}.

\item
Ejecuta el programa con el botón \textmark{Launch}. 

\end{enumerate}

