\subsection{Análisis dinámico de código con \emph{sanitizers}}

Hay varios \textmark{sanitizers} disponibles en algunos compiladores:

\begin{itemize}
  \item \textmark{Address Sanitizer}: 
        Detecta errores de memoria
  \item \textmark{Leak Sanitizer}: 
        Subconjunto de \emph{AddressSanitizer} solamente para goteos de memoria.
  \item \textmark{Thread Sanitizer}: 
        Detecta errores de concurrencia.
  \item \textmark{Undefined Behavior Sanitizer}: 
        Detecta algunos comportamientos no definidos.
  \item \textmark{Memory Sanitizer} (solo con \cppid{clang++}): 
        Detecta el uso de memoria sin iniciar.
\end{itemize}

Puedes utilizarlos añadiendo la correspondiente configuración a
\textemph{CMakePresets.json}:

\begin{enumerate}

\item Añade a la sección de \textgood{configurePresets} una configuración para
activar las opciones de \textmark{Address Sanitizer}:

\begin{lstlisting}
    ,
    {
      "name": "asan",
      "inherits": "default",
      "cacheVariables": {
        "CMAKE_CXX_FLAGS": "-fsanitize=address",
        "CMAKE_EXE_LINKER_FLAGS": "-fsanitize=address"
      }
    }
\end{lstlisting}

\item Añade a la sección de \textgood{buildPresets} una configruación para
compilar con la configuración \textgood{asan}:

\begin{lstlisting}
    ,
    {
      "name": "debug-asan",
      "configurePreset": "asan",
      "configuration": "Debug"
    }
\end{lstlisting}

\item
Compila el ejecutable con \textmark{debug-asan}.

\item
Ejectua el programa y analiza la salida generada.

\end{enumerate}
