\clearpage
\subsection{Tarea 2: Estructuras y arrays}

En esta tarea se pretende analizar dos programas: \cppid{soa.cpp} y
\cppid{aos.cpp}. Ambos programas implementan la misma funcionalidad: suma de las
coordenadas de dos conjuntos (\cppid{a} y \cppid{b}) de puntos con coordenadas en el plano.
Uno de ellos hace uso de tres arrays de estructuras que representan puntos y el otro hace uso de tres estructuras con dos arrays. El programa no imprime resultado.

\lstinputlisting[caption={soa.cpp},frame=single,numbers=left,basicstyle=\small]{lab/03-cache/soa.cpp}
\lstinputlisting[caption={aos.cpp},frame=single,numbers=left,basicstyle=\small]{lab/03-cache/aos.cpp}

Se pide: 

\begin{enumerate}

\item Ejecute \cppid{soa} y \cppid{aos} con el programa \textmark{valgrind} y la herramienta \textmark{cachegrind} para las siguientes configuraciones:

\begin{itemize}
\item Caché de último nivel fijada a 256 KiB.
\item Evalúe con tamaños de caché L1D de 8 KiB, 16 KiB y 32 KiB.
\end{itemize}

\item Observe los resultados obtenidos e inspeccione el código con la
herramienta \textmark{cg\_annotate}. Anote los resultados globales y observe
los resultados prestando especial la cuerpo de los bucles.

\item Compare ambos resultados.
Discuta en su informe los resultados para Dr, D1mr, DLmr, Dw, D1mw y DLmw

\end{enumerate}

