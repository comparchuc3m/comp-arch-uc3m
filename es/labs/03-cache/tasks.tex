\section{Tareas}

Para la realización de este laboratorio se suministra código fuente para un
conjunto de pequeños programas que se deben evaluar (ver siguientes secciones).
También se suministra un archivo \textmark{CMakeLists.txt}.

En la raíz de su directorio de trabajo en \cppid{avignon}, cree un
directorio con el nombre \cppid{lab3}.

\begin{lstlisting}[style=terminal,aboveskip=1em,belowskip=1em]
usuario@avignon-frontend:~$ mkdir lab3
usuario@avignon-frontend:~$
\end{lstlisting}

Desde su máquina, transfiera todos los archivos de soporte al laboratorio
que habrá descargado anteriormente:

\begin{lstlisting}[style=terminal,aboveskip=1em,belowskip=1em]
usuario@mimaquina:~/Descargas/lab3$ scp * usuario@avignon.lab.inf.uc3m.es:~/lab3/
usuario@avignon.lab.inf.uc3m.es password: 
aos.cpp                                                            100%  237    20.6KB/s   00:00    
CMakeLists.txt                                                     100%  369    35.7KB/s   00:00    
loop_merge.cpp                                                     100%  254    22.7KB/s   00:00    
loop_merge_opt.cpp                                                 100%  216    17.2KB/s   00:00    
soa.cpp                                                            100%  249    24.1KB/s   00:00    
usuario@mimaquina:~/Descargas/lab3$ 
\end{lstlisting}

Para compilar todos los programas a evaluar, se pueden usar los siguiente mandatos
que puede colocar en un \emph{script}:

\begin{lstlisting}[style=terminal,aboveskip=1em,belowskip=1em]
cmake -S . -B debug -DCMAKE_BUILD_TYPE=Debug
cmake --build debug
\end{lstlisting}

\textbad{NOTA:} 
A cada grupo de laboratorio se le asignará una de las siguientes configuraciones:
\begin{itemize}
\item Configuración 1:
Tamaño de línea de 32 B y todas las cachés son asociativas de 2 vías.
\item Configuración 2:
Tamaño de línea de 32 B y todas las cachés son asociativas de 4 vías.
\item Configuración 3:
Tamaño de línea de 64 B y todas las cachés son asociativas de 4 vías.
\item Configuración 4:
Tamaño de línea de 64 B y todas las cachés son asociativas de 8 vías.
\end{itemize}

\clearpage
\subsection{Tarea 1: Fusión de bucles}

En esta tarea se pretende analizar dos programas: \cppid{loop\_merge.cpp} y \cppid{loop\_merge\_opt.cpp}.

\lstinputlisting[caption={loop\_merge.cpp},frame=single,numbers=left,basicstyle=\small]{lab/03-cache/loop_merge.cpp}
\lstinputlisting[caption={loop\_merge\_opt.cpp},frame=single,numbers=left,basicstyle=\small]{lab/03-cache/loop_merge_opt.cpp}

Ambos programas implementan la misma funcionalidad:
dados dos vectores $\vec{z}$ y $\vec{t}$, calculan otros dos vectores $\vec{u}$ y $\vec{v}$:

\[
\vec{u} = \vec{z} + \vec{t}
\]
\[
\vec{v} = \vec{u} + \vec{t}
\]

Para ello, los programas hacen uso de 4 arrays de tamaño fijo.
El programa no imprime ningún resultado.

Se pide: 

\begin{enumerate}

\item Ejecute \cppid{loop\_merge} y \cppid{loop\_merge\_opt} con el programa
\textmark{valgrind} y la herramienta \textmark{cachegrind} para las
siguientes configuraciones:

\begin{itemize}
\item Caché de último nivel fijada a 128 KiB.
\item Evalúe con tamaños de caché L1D de 16 KiB, 32 KiB y 64 KiB.
\end{itemize}

\item Observe los resultados obtenidos e inspeccione el código con la
herramienta \textmark{cg\_annotate}. Anote los resultados globales y observe los
resultados prestando especial atención al cuerpo de los bucles.

\item Compare ambos resultados.
Estudie los resultados para Dr, D1mr, DLmr, Dw,
D1mw y DLmw.
 
\end{enumerate}


\clearpage
\subsection{Tarea 2: Estructuras y arrays}

En esta tarea se pretende analizar dos programas: \cppid{soa.cpp} y
\cppid{aos.cpp}. Ambos programas implementan la misma funcionalidad: suma de las
coordenadas de dos conjuntos (\cppid{a} y \cppid{b}) de puntos con coordenadas en el plano.
Uno de ellos hace uso de tres arrays de estructuras que representan puntos y el otro hace uso de tres estructuras con dos arrays. El programa no imprime resultado.

\lstinputlisting[caption={soa.cpp},frame=single,numbers=left,basicstyle=\small]{lab/03-cache/soa.cpp}
\lstinputlisting[caption={aos.cpp},frame=single,numbers=left,basicstyle=\small]{lab/03-cache/aos.cpp}

Se pide: 

\begin{enumerate}

\item Ejecute \cppid{soa} y \cppid{aos} con el programa \textmark{valgrind} y la herramienta \textmark{cachegrind} para las siguientes configuraciones:

\begin{itemize}
\item Caché de último nivel fijada a 256 KiB.
\item Evalúe con tamaños de caché L1D de 8 KiB, 16 KiB y 32 KiB.
\end{itemize}

\item Observe los resultados obtenidos e inspeccione el código con la
herramienta \textmark{cg\_annotate}. Anote los resultados globales y observe
los resultados prestando especial la cuerpo de los bucles.

\item Compare ambos resultados.
Discuta en su informe los resultados para Dr, D1mr, DLmr, Dw, D1mw y DLmw

\end{enumerate}


