\clearpage
\section{Entrega}

La fecha límite para la entrega de los resultados de este laboratorio 
se anunciará a través de Aula Global.

Se seguirán las siguientes reglas:

\begin{itemize}

  \item Todas las entregas se realizarán a través de aula global.

  \item Se deberá entregar un archivo en formato zip que contendrá los siguientes archivos:
     \begin{itemize}
       \item \cppid{counter\_mutex.cpp}.
       \item \cppid{counter\_atomic.cpp}.
       \item \cppid{counter\_spin\_seq.cpp}.
       \item \cppid{counter\_spin\_seq\_opt.cpp}.
       \item \cppid{counter\_spin\_ra.cpp}.
       \item \cppid{counter\_spin\_ra\_opt.cpp}.
     \end{itemize}

  \item El contenido del informe deberá incluir.
  \begin{enumerate}
    \item Datos identificativos:
      \begin{itemize}
        \item Título.
        \item Nombre y NIA de los estudiantes.
        \item Grupo reducido al que pertenecen los estudiantes.
        \item Número de grupo de laboratorio.
      \end{itemize}

    \item Tabla de resultados de las evaluaciones.

    \item Breves conclusiones.

  \end{enumerate}
  
\end{itemize}

En cuanto a los resultados de las evaluaciones deberá entregarse exclusivamente
información numérica en formato tabla. Indique claramente al principio de la
sección cuantas veces ha ejecutado cada experimento.  Cada tabla deberá tener
una fila por cada uno de los programas y una columna por cada versión en cuanto
a número de hilos (2, 4, 8, 16).

\textbad{IMPORTANTE}: Deberá ejecutar un número suficiente de veces cada
experimento como para asegurar una desviación típica aceptable.

Se deberán incluir en el informe las siguientes tablas:

\begin{enumerate}
  \item Tabla de tiempos medios de ejecución.
  \item Tabla de desviaciones típicas.
\end{enumerate}
