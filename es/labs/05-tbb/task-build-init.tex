\subsection{Construcción de la versión inicial}

Tu primera tarea es construir la versión original de los ejercicios. Deberás
construir solamente la versión de \emph{release}. Para hacerlo, utiliza un
\emph{script} como el que se presenta en el listado~\ref{lst:build-normal}.

Podrás ejecutar este script en un nodo normal del clúster haciendo:

\begin{lstlisting}[style=terminal,language=bash]
sbatch build-normal.sh
\end{lstlisting}

Para ejecutar cualquiera de los programas, puedes usar el
\emph{script} como \cppid{run-invoke.sh} (ver listado~\ref{lst:run-invoke-normal}).

\lstinputlisting[
language=bash,style=terminal,
caption={Script de ejecución para invoke en un nodo normal},
label={lst:run-invoke-normal}
]{lab/05-tbb/run-invoke-normal.sh}

Observa que en este script se definen dos variables del sistema de colas:
\cppid{job-name} y {output}. La primera fija el nombre del trabajo en el sistema
de colas y la segunda define el nombre del archivo de salida del trabajo.
Recuerda que el directorio logs, debe existir previamente.

\begin{lstlisting}[style=terminal,language=bash]
mkdir logs
sbatch run-invoke-normal.sh
\end{lstlisting}

La salida de la ejecución del programa podrás encontrarla en el archivo
\textemph{logs/invoke-normal.txt}.

Ahora puedes hacer lo mismo ejecutando para el resto de programas
lazando scripts similares. En el material de apoyo encontrarás los siguientes
scripts:
\textemph{run-invoke-normal.sh},
\textemph{run-matrixmult-normal.sh},
\textemph{run-picalc-normal.sh},
\textemph{run-pimontecarlo-normal.sh} y
\textemph{run-matrixmult-part-normal.sh}.

