\clearpage
\subsection{Partición avanzada de tareas y producto de matrices}

Todo el código de este ejercicio se encuentra en la carpeta
\textemph{05-matrixmult-part}.

En este ejercicio vamos a valuar de los distintos particionadores de tareas que
ofrece TBB.

\subsubsection{Partición de tareas}

En TBB cualquier algoritmo paralelo genera un conjunto de tareas. Éstas tareas
se pueden dividir en otras más pequeñas. De esta manera, cada hilo mantiene una
lista de tareas pendientes de realizar de distinto tamaño. Cuando la lista de
tareas de un hilo queda vacía, este hilo roba una tarea a otro hilo para
progresar. Este mecanismo es muy efectivo para mantener el equilibrio de carga
entre los distintos hilos.

Un elemento fundamental en el reparto de trabajo es el \textmark{partitioner}.
Es decir, el objeto encargado de la partición de tareas en otras de menor
tamaño.

La biblioteca ofrece las siguientes estrategias:

\begin{itemize}

\item \cppid{simple\_partitioner}:
El rango se divide hasta que se alcanza un tamaño de grano asociado al rango.
Si no se especifica ningún valor el tamaño del grano es la unidad.

\item \cppid{auto\_partitioner}:
El rango se divide dinámicamente y puede pararse la división aunque no se
alcance el tamaño de grano.
Es el particionador por defecto en la mayoría de los algoritmos.

\item \cppid{static\_partitioner}:
El rango se divide uniformemente pero no hay posibilidad de equilibrio de carga.
Es el particionador con menos sobrecarga, pero puede no ser adecuado si hay
desequilibrio entre las tareas.

\item \cppid{affinity\_partitioner}:
Combina características de \cppid{auto\_partitioner} y
\cppid{static\_partitioner} mejorando el uso de la caché si el mismo objeto
particionador se pasa repetidamente al mismo algoritmo. Comienza generando una
partición estática, pero posteriormente la adapta dinámicamente.

\end{itemize}

\subsection{Evaluación de los particionadores}

Este ejercicio incluye código para evaluar distintos algoritmos de producto de
matrices con distintos particionadores. En cada evaluación cada algoritmo se
ejecuta 10 veces y se toman valores promedios.

En este ejercicio no tienes que modificar el código. Analiza los resultados con
los distintos particionadores e intenta encontrar explicaciones para los
resultados.
