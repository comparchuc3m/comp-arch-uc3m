\section{Descripción del laboratorio}

En este laboratorio te familiarizarás con la programación paralela utilizando
la biblioteca TBB (\emph{Threading Building Blocks}).

En total practicarás con los siguientes ejercicios:

\begin{itemize}

\item \textmark{invoke}:
En este ejercicio practicarás la ejecución de tareas en paralelo.

\item \textmark{matrixmult}:
En este ejercicio partiremos de un producto de matrices clásico y veremos
distintas alternativas para optimizar la ejecución del producto de matrices.

\item \textmark{picalc}:
En este ejercicio veremos cómo acelerar un método numérico para calcular una
integral. Aplicaremos estas técnicas al cálculo del número $\pi$.

\item \textmark{pimontecarlo}:
En este ejercicio veremos cómo se puede paralelizar un algoritmo que depende de
un generador de números aleatorios. Aplicaremos esta técnica a una simulación
de Monte Carlo para estimar el valor del número $\pi$.

\item \textmark{matrixmult-part}:
En este ejercicio veremos los distintas formas de repartir el trabajo entre los
hilos de un programa paralelo.

\end{itemize}

\subsection{Construcción de los programas}

Por favor, recuerda que para compilar la versión optimizada del programa 
deberás generar una configuración de \emph{release}.

\begin{tcolorbox}
[colframe=red!50!black,
colback=red!10!white, 
title={\textbf{IMPORTANTE}}]
Para ejecutar trabajos en el clúster \cppid{avignon}, deberás crear un
\emph{script} para lanzar la compilación.
\end{tcolorbox}

Por ejemplo, el script~\ref{lst:build-normal} permite compilar en un nodo normal del
cluster Avignon. En primer lugar, se genera la configuración de compilación en el
directorio \textemph{build/normal}. A continuación se realiza la compilación
generando los ejecutables correspondientes utilizando la configuración de
compilación \textemph{gcc-release} y la configuración \textemph{Release}.

\lstinputlisting[
caption={Script de compilación para nodo normal del clúster Avignon},
label=lst:build-normal,
language=bash,style=terminal]
{lab/05-tbb/build-normal.sh}

\subsection{Estructura del código fuente}

El código fuente que se suministra tiene la siguiente estructura:

\begin{itemize}

\item \textgood{01-invoke}: 
Ejercicio de invocación de tareas paralelas.

\item \textgood{02-matrixmult}:
Ejercicio de producto de matrices.

\item \textgood{03-picalc}:
Ejercicio de cálculo del número $\pi$.

\item \textgood{04-pimontecarlo}:
Ejercicio del estimación del número $\pi$ con el método de Monte Carlo.

\item \textgood{05-matrixmult-part}:
Ejercicio de uso de esquemas de particionamiento de tareas.

\end{itemize}

Ten en cuenta que durante este laboratorio, tendrás que modificar los ejemplos
para completarlos antes de ejecutar.
