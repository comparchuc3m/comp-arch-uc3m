\subsection{Características de nodo NUMA}

A través del \emph{front-end} \textmark{avignon} puedes tener acceso a un
servidor de cómputo (llamado \textmark{stan}).

Prepara un \emph{script} para ver las características hardware de
\textmark{stan} llamado \cppid{lscpu.sh}. El listado~\ref{lst:lscpu} presenta
un posible script.

\lstinputlisting[
language=bash,style=terminal,
caption={Script para obtención de características del hardware},
label={lst:lscpu},
]{lab/05-tbb/lscpu.sh}

Revisa las principales características del hardware y analiza las diferencias
con los nodos normales de cómputo y el nodo \textmark{stan}.

\subsection{Construcción de la versión NUMA}

En el material de apoyo encontrarás \emph{scripts} con el sufijo
\textmark{stan}. Estos están diseñados para usarlos en este nodo de cómputo.

Podrás compilar en este nodo haciendo:

\begin{lstlisting}[language=bash,style=terminal]
sbatch -p stan build-stan.sh
\end{lstlisting}

Para ejecutar, puedes usar un script como \textgood{run-invoke-stan.sh} (ver
listado~\ref{lst:run-invoke-stan}):

\lstinputlisting[
language=bash,style=terminal,
caption={Script de ejecución para invoke en el nodo stan},
label={lst:run-invoke-stan}
]{lab/05-tbb/run-invoke-stan.sh}

\textbad{IMPORTANTE}: 
Para ejecutar un \emph{script} en \textmark{stan} debes especificar el nombre de
la cola de ejecución.

\begin{lstlisting}[style=terminal]
sbatch -p stan run-invoke-stan.sh
\end{lstlisting}

