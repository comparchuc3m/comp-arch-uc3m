\subsection{Estructura del proyecto inicial}

El código que se suministra contiene la estructura general del proyecto con un
directorio para cada uno de los ejercicios que vamos a realizar.

El archivo \textemph{CMakePresets.json} contiene la configuración general del
proyecto con los \textmark{presets} de configuración y de construcción.

Para poder trabajar con TBB necesitarás que se encuentre instalada esta
biblioteca. La biblioteca ya está instalada en todos los nodos de cómputo del
sistema \textmark{avignon}. También está incluida en la configuración del
correspondiente \textmark{Dockerfile} por si quieres probar los ejemplos en tu
máquina. El paquete para poder instalar esta biblioteca en una máquina GNU/Linux
es \textemph{libtbb-dev}.

Si te fijas en el archivo \textemph{CMakeLists.txt} lo único relevante está en
la siguiente línea:

\begin{lstlisting}
find_package(TBB REQUIRED)
\end{lstlisting}

Esto indica a CMake que debe encontrar instalado en el sistema la biblioteca TBB
y que en otro caso debe generar un error de compilación.

Dentro del proyecto encontrarás una carpeta con el contenido inicial de cada
ejercicio. Todo el código se encuentra siempre en un único archivo con el nombre
\cppid{main.cpp}. En el correspondiente archivo \textemph{CMakeLists.txt}
encontrarás la definición del ejecutable y el establecimiento de la dependencia
con respecto a TBB.

\begin{lstlisting}
add_executable(invoke)
target_sources(invoke PRIVATE
    main.cpp
)
target_link_libraries(invoke PRIVATE
    TBB::tbb
)
\end{lstlisting}
