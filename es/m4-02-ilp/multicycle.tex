\section{Operaciones multiciclo}

\begin{frame}[t]{Operaciones de coma flotante}
\begin{itemize}
  \item ¿Operaciones de coma flotante en un ciclo?
    \begin{itemize}
      \item Tener un ciclo de reloj extremadamente largo.
        \begin{itemize}
          \item Impacto en rendimiento global.
        \end{itemize}
      \item Lógica de control de FPU muy compleja.
        \begin{itemize}
          \item Consumo excesivo de recursos de diseño.
        \end{itemize}
    \end{itemize}

  \mode<presentation>{\vfill\pause}
  \item \textgood{Alternativa}: Segmentación de coma flotante.
    \begin{itemize}
      \item La etapa de ejecución se puede repetir varias veces.
      \item Múltiples unidades funcionales en EX.
        \begin{itemize}
          \item Ejemplo: Unidad entera, multiplicador FP y entero, sumador FP, divisor FP y entero.
        \end{itemize}
    \end{itemize}
\end{itemize}
\end{frame}

\begin{frame}[t]{Segmentación con coma flotante}
\begin{itemize}
  \item La etapa EX pasa a tener una duración mayor que 1 ciclo de reloj.
\end{itemize}
\makebox[\textwidth][c]{
\section{Operaciones multiciclo}

\begin{frame}[t]{Operaciones de coma flotante}
\begin{itemize}
  \item ¿Operaciones de coma flotante en un ciclo?
    \begin{itemize}
      \item Tener un ciclo de reloj extremadamente largo.
        \begin{itemize}
          \item Impacto en rendimiento global.
        \end{itemize}
      \item Lógica de control de FPU muy compleja.
        \begin{itemize}
          \item Consumo excesivo de recursos de diseño.
        \end{itemize}
    \end{itemize}

  \mode<presentation>{\vfill\pause}
  \item \textgood{Alternativa}: Segmentación de coma flotante.
    \begin{itemize}
      \item La etapa de ejecución se puede repetir varias veces.
      \item Múltiples unidades funcionales en EX.
        \begin{itemize}
          \item Ejemplo: Unidad entera, multiplicador FP y entero, sumador FP, divisor FP y entero.
        \end{itemize}
    \end{itemize}
\end{itemize}
\end{frame}

\begin{frame}[t]{Segmentación con coma flotante}
\begin{itemize}
  \item La etapa EX pasa a tener una duración mayor que 1 ciclo de reloj.
\end{itemize}
\makebox[\textwidth][c]{
\section{Operaciones multiciclo}

\begin{frame}[t]{Operaciones de coma flotante}
\begin{itemize}
  \item ¿Operaciones de coma flotante en un ciclo?
    \begin{itemize}
      \item Tener un ciclo de reloj extremadamente largo.
        \begin{itemize}
          \item Impacto en rendimiento global.
        \end{itemize}
      \item Lógica de control de FPU muy compleja.
        \begin{itemize}
          \item Consumo excesivo de recursos de diseño.
        \end{itemize}
    \end{itemize}

  \mode<presentation>{\vfill\pause}
  \item \textgood{Alternativa}: Segmentación de coma flotante.
    \begin{itemize}
      \item La etapa de ejecución se puede repetir varias veces.
      \item Múltiples unidades funcionales en EX.
        \begin{itemize}
          \item Ejemplo: Unidad entera, multiplicador FP y entero, sumador FP, divisor FP y entero.
        \end{itemize}
    \end{itemize}
\end{itemize}
\end{frame}

\begin{frame}[t]{Segmentación con coma flotante}
\begin{itemize}
  \item La etapa EX pasa a tener una duración mayor que 1 ciclo de reloj.
\end{itemize}
\makebox[\textwidth][c]{
\section{Operaciones multiciclo}

\begin{frame}[t]{Operaciones de coma flotante}
\begin{itemize}
  \item ¿Operaciones de coma flotante en un ciclo?
    \begin{itemize}
      \item Tener un ciclo de reloj extremadamente largo.
        \begin{itemize}
          \item Impacto en rendimiento global.
        \end{itemize}
      \item Lógica de control de FPU muy compleja.
        \begin{itemize}
          \item Consumo excesivo de recursos de diseño.
        \end{itemize}
    \end{itemize}

  \mode<presentation>{\vfill\pause}
  \item \textgood{Alternativa}: Segmentación de coma flotante.
    \begin{itemize}
      \item La etapa de ejecución se puede repetir varias veces.
      \item Múltiples unidades funcionales en EX.
        \begin{itemize}
          \item Ejemplo: Unidad entera, multiplicador FP y entero, sumador FP, divisor FP y entero.
        \end{itemize}
    \end{itemize}
\end{itemize}
\end{frame}

\begin{frame}[t]{Segmentación con coma flotante}
\begin{itemize}
  \item La etapa EX pasa a tener una duración mayor que 1 ciclo de reloj.
\end{itemize}
\makebox[\textwidth][c]{
\input{es/m4-02-ilp/multicycle.tkz}
}

\end{frame}

\begin{frame}[t]{Latencia e intervalo de iniciación}
\begin{itemize}
  \item \textgood{Latencia}: Número de ciclos entre la instrucción que produce un resultado y la instrucción que usa ese resultado.
  \item \textgood{Intervalo de iniciación}: Número de ciclos entre la emisión de dos instrucciones que usan la misma unidad funcional. 
\end{itemize}

\mode<presentation>{\vfill}
{
\begin{tabular}[c]{*{3}{|p{.3\textwidth}}|}
\hline
Operación & Latencia & Intervalo iniciación 
\\
\hline
\hline

ALU entera  &
0  &
1
\\
\hline

Instrucciones \emph{load}&
1 &
1
\\
\hline

Suma FP &
3 &
1
\\
\hline

Multiplicación FP &
6 &
1
\\
\hline

División FP &
24 &
25
\\
\hline

\end{tabular}
}

\end{frame} 

}

\end{frame}

\begin{frame}[t]{Latencia e intervalo de iniciación}
\begin{itemize}
  \item \textgood{Latencia}: Número de ciclos entre la instrucción que produce un resultado y la instrucción que usa ese resultado.
  \item \textgood{Intervalo de iniciación}: Número de ciclos entre la emisión de dos instrucciones que usan la misma unidad funcional. 
\end{itemize}

\mode<presentation>{\vfill}
{
\begin{tabular}[c]{*{3}{|p{.3\textwidth}}|}
\hline
Operación & Latencia & Intervalo iniciación 
\\
\hline
\hline

ALU entera  &
0  &
1
\\
\hline

Instrucciones \emph{load}&
1 &
1
\\
\hline

Suma FP &
3 &
1
\\
\hline

Multiplicación FP &
6 &
1
\\
\hline

División FP &
24 &
25
\\
\hline

\end{tabular}
}

\end{frame} 

}

\end{frame}

\begin{frame}[t]{Latencia e intervalo de iniciación}
\begin{itemize}
  \item \textgood{Latencia}: Número de ciclos entre la instrucción que produce un resultado y la instrucción que usa ese resultado.
  \item \textgood{Intervalo de iniciación}: Número de ciclos entre la emisión de dos instrucciones que usan la misma unidad funcional. 
\end{itemize}

\mode<presentation>{\vfill}
{
\begin{tabular}[c]{*{3}{|p{.3\textwidth}}|}
\hline
Operación & Latencia & Intervalo iniciación 
\\
\hline
\hline

ALU entera  &
0  &
1
\\
\hline

Instrucciones \emph{load}&
1 &
1
\\
\hline

Suma FP &
3 &
1
\\
\hline

Multiplicación FP &
6 &
1
\\
\hline

División FP &
24 &
25
\\
\hline

\end{tabular}
}

\end{frame} 

}

\end{frame}

\begin{frame}[t]{Latencia e intervalo de iniciación}
\begin{itemize}
  \item \textgood{Latencia}: Número de ciclos entre la instrucción que produce un resultado y la instrucción que usa ese resultado.
  \item \textgood{Intervalo de iniciación}: Número de ciclos entre la emisión de dos instrucciones que usan la misma unidad funcional. 
\end{itemize}

\mode<presentation>{\vfill}
{
\begin{tabular}[c]{*{3}{|p{.3\textwidth}}|}
\hline
Operación & Latencia & Intervalo iniciación 
\\
\hline
\hline

ALU entera  &
0  &
1
\\
\hline

Instrucciones \emph{load}&
1 &
1
\\
\hline

Suma FP &
3 &
1
\\
\hline

Multiplicación FP &
6 &
1
\\
\hline

División FP &
24 &
25
\\
\hline

\end{tabular}
}

\end{frame} 
