\section{Introducción}

\begin{frame}[t]{¿Por qué SIMD?}
\begin{itemize}
  \item Muchas aplicaciones exhiben un \textmark{Paralelismo a Nivel de Datos}
        (DLP -- \emph{Data Level Parallelism}) significativo.
    \begin{itemize}
      \item Operaciones matriciales (álgebra lineal).
      \item Procesamiento multimedia de imagen y sonido.
      \item Algoritmos de aprendizaje automático.
    \end{itemize}

  \mode<presentation>{\vfill\pause}
  \item Eficiencia energética:
    \begin{itemize}
      \item Una única instrucción puede lanzar múltiples operaciones de datos.
    \end{itemize}

  \mode<presentation>{\vfill\pause}
  \item Modelo mental secuencial:
    \begin{itemize}
      \item Pero la aceleración se deriva de la operaciones de datos paralelas.
    \end{itemize}
\end{itemize}
\end{frame}

\begin{frame}[t]{Variantes SIMD}
\begin{itemize}
  \item \textmark{Arquitecturas vectoriales}:
    \begin{itemize}
      \item Extiend la ejecución en \emph{pipeline} de muchas operaciones de datos.
      \item Consideradas costosas hasta hace poco.
    \end{itemize}

  \mode<presentation>{\vfill\pause}
  \item \textmark{Extensiones SIMD a juegos de instrucciones}:
    \begin{itemize}
      \item Operaciones simultáneas con paralelismo de datos.
      \item MMX $\rightarrow$ MultMedia eXtensions.
      \item SSE $\rightarrow$ Streaming Simd Extensions.
      \item AVX $\rightarrow$ Advanced Vector eXtensions.
    \end{itemize}

  \mode<presentation>{\vfill\pause}
  \item \textmark{GPUs}:
    \begin{itemize}
      \item Uso de aceleradores gráficos como aceleradores cómputo.
      \item Memoria de anfitrión y memoria de GPU $\rightarrow$ Necesidad de transferencias.
      \item Computación heterogénea.
    \end{itemize}

\end{itemize}
\end{frame}
