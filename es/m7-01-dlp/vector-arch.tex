\section{Arquitecturas vectoriales}

\begin{frame}[t]{Enfoque de arquitectura vectorial}
\begin{itemize}
  \item Realizan cálculos sobre vectores de datos:
    \begin{enumerate}
      \item Lee datos de memoria y los ubican en grandes bancos de registros secuenciales.
      \item Operan sobre datos en el banco de registros.
      \item Almacenan datos en memoria leyéndolos del banco de registros.
    \end{enumerate}

  \mode<presentation>{\vfill\pause}
  \item El banco de registros se comporta como un búfer controlado por el compilador.
    \begin{itemize}
      \item Oculta la latencia de memoria.
      \item El acceso a memoria profundamente segmentado.
      \item Gran latencia de memoria solamente una vez por \emph{load}/\emph{store}.
    \end{itemize}
\end{itemize}
\end{frame}

\begin{frame}[t]{Ejemplo: RV64V}
\begin{itemize}
  \item \textmark{Registros vectoriales}: 
    \begin{itemize}
      \item 32 registros vectoriales.
      \item Cada elemento del vector de 64 bits de ancho.
    \end{itemize}

  \mode<presentation>{\vfill\pause}
  \item \textmark{Unidades funcionales vectoriales}: 
    \begin{itemize}
      \item Cada unidad está totalmente segmentada.
    \end{itemize}

  \mode<presentation>{\vfill\pause}
  \item \textmark{Unidad vectorial carga/almacenamiento}: 
    \begin{itemize}
      \item Transfiere vector de/a memoria.
      \item Totalmente segmentada.
    \end{itemize}

  \mode<presentation>{\vfill\pause}
  \item \textmark{Registros escalares}:
    \begin{itemize}
      \item Entrada de datos escalar a unidades funcionales vectoriales.
      \item Cálculo de direcciones para carga/almacenamiento vectorial.
    \end{itemize}
\end{itemize}
\end{frame}

\begin{frame}[t]{Tipos dinámicos de registros}
\begin{itemize}
  \item Las instrucciones vectoriales omiten tipo de datos y tamaño.
    \begin{itemize}
      \item Asocia tipo de datos y tamaño con cada registro vectorial.
      \item Instrucciones para configurar registros vectoriales.
      \item Simplifica el número de instrucciones vectoriales.
    \end{itemize}

  \mode<presentation>{\vfill\pause}
  \item Memoria usada por registros vectoriales activados.
    \begin{itemize}
      \item 4 vectores de \cppkey{doubles}:
        \begin{itemize}
          \item \asminst{vsetdcfg} \asmreg{4*FP64} 
        \end{itemize}
      \item 1 vector de \cppkey{floats} y 3 vectores de \cppkey{doubles}.
        \begin{itemize}
          \item \asminst{vsetdcfg} \asmreg{1*FP32,3*FP64}
        \end{itemize}
    \end{itemize}

  \mode<presentation>{\vfill\pause}
  \item Longitud máxima de vector:
    \begin{itemize}
      \item Establecida por procesador y derivada del número de vectores activados.
      \item 1024 bytes con 4 vectores de \cppkey{doubles} $\Rightarrow$ 
            256 / 8 = 32 elementos.
    \end{itemize}
\end{itemize}
\end{frame}

\begin{frame}[t,fragile]{Ejemplo: DAXPY}
\begin{columns}[T]

\column{.5\textwidth}
\begin{itemize}
  \item Transformación lineal de un vector:
\[
\vec{y} = a \cdot \vec{x} + \vec{y}
\]

  \item Asume longitud de vector encaja con tamaño de array.
    \begin{itemize}
      \item Arrays de 32 elementos.
    \end{itemize}
\end{itemize}

\pause
\column{.5\textwidth}
\begin{block}{Código escalar}
\begin{lstlisting}[language=generalasm2]
      fld f0, a
      addi t2,t0,#256
Loop: fld f1,0(t0)
      fmul.d f1,f1,f0
      fld f2,0(t1)
      fadd.d f2,f2,f1
      fsd f2,0(t1)
      addi t0,t0,#8
      addi t1,t1,#8
      bne t2,t0,Loop
\end{lstlisting}
\end{block}

\end{columns}
\end{frame}

\begin{frame}[t,fragile]{DAXPY con instruccioens vectoriales}
\begin{columns}[T]

\column{.65\textwidth}
\begin{block}{DAXPY vectorizado}
\begin{lstlisting}[language=generalasm2]
vsetdcfg 4*FP64  # 4 vectors double precision
fld f0, a        # Load scalar a in f0
vld v0, t0       # Load vector v0 from address t0
vmul v1, v0, f0  # Multiply vector by scalar
vld v2, t1       # Load vector v2 from address t1
vadd v3, v1, v2  # Add vectors
vst v3, t1       # Store vector v3 to address t1
vdisable         # Disable vector registers
\end{lstlisting}
\end{block}

\column{.35\textwidth}
\begin{itemize}
  \item Reducción den el número de instrucciones.
    \begin{itemize}
      \item 258 $\rightarrow$ 8.
    \end{itemize}

  \item La vectorización requiere iteraciones independientes.

  \item Reducción de detenciones gracias a encadenamiento de instrucciones.
\end{itemize}

\end{columns}
\end{frame}
