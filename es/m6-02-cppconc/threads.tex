\subsection{Hilos}

\begin{frame}[fragile]{Lanzamiento de hilos}
\begin{itemize}
  \item Un \textgood{hilo} representado por la clase \cppid{std::thread}.
    \begin{itemize}
      \item Normalmente representa un hilo del SO.
    \end{itemize}
\end{itemize}

\mode<presentation>{\vfill}
\begin{block}{Lanzando un hilo a través de una función}
\begin{lstlisting}
void f1();
void f2();

void g() {
  std::thread t1{f1}; // Lanza un hilo que ejecuta f1()
  std::thread t2{f2}; // Lanza un hilo que ejecuta f2()

  t1.join(); // Espera a que t1 termine.
  t2.join(); // Espera a que t2 termine.
}
\end{lstlisting}
\end{block}
\end{frame}


\begin{frame}[fragile]{Objetos compartidos}
\begin{itemize}
  \item Dos hilos pueden acceder a un \textmark{objeto compartido}.
  \item Posibilidad de \textbad{carreras de datos}.
\end{itemize}

\mode<presentation>{\vfill}
\begin{block}{Acceso a variables compartidas}
\begin{lstlisting}
int x = 42;

void f() { ++x; }
void g() { x=0; }
void h() { std::cout << "Hola" << "\n"; }
void i() { std::cout << "Adios" << "\n"; }

void carrera() {
  std::thread t1{f}; std::thread t2{g};
  t1.join(); t2.join();

  std::thread t3{h}; std::thread t4{i};
  t3.join(); t4.join();
}
\end{lstlisting}
\end{block}
\end{frame}

\begin{frame}[fragile]{Paso de argumentos}
\begin{itemize}
  \item \textgood{Paso de argumentos simplificado} sin necesidad de \textmark{casts}.
\end{itemize}

\mode<presentation>{\vfill}
\begin{block}{Paso de argumentos}
\begin{lstlisting}
void f1(int x);
void f2(double x, double y);

void g() {
  std::thread t1{f1, 10}; // Ejecuta f1(10)
  std::thread t2{f1}; // Error
  std::thread t3{f2, 1.0} // Error
  std::thread t4{f2, 1.0, 1.0}; // Ejecuta f2(1.0,1.0)
  //...
  // Joins de hilos
\end{lstlisting}
\end{block}
\end{frame}

\begin{frame}[fragile]{Hilos y objetos función}
\begin{itemize}
  \item \textgood{Objeto función}: Objeto invocable como función.
  \item Sobrecarga/redefinición del \textmark{operador} \cppid{()}
\end{itemize}

\mode<presentation>{\vfill}
\begin{block}{Objeto función en un hilo}
\begin{lstlisting}
struct mifunc {
  mifunc(int val) : x{val} {}
  void operator()() { haz_algo(x); }
  int x;
};

void g() {
  mifunc f1{10}; // Construye objeto f1
  f1(); // Invoca operador de llamada f1.operator()
  std::thread t1{f1}; // Ejecuta f1() en un hilo
  std::thread t2{mifunc{20}}; // Construye temporal e invoca mifunc{20}.operator()
  // ...
  // Joins de hilos
\end{lstlisting}
\end{block}
\end{frame}

