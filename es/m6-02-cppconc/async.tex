\subsection{Tareas asíncronas}

\begin{frame}[t]{Tareas asíncronas y futuros}
\begin{itemize}
  \item Una tarea asíncrona permite el lanzamiento simple de la ejecución de una tarea:
    \begin{itemize}
      \item En otro hilo de ejecución.
      \item Como una tarea diferida.
    \end{itemize}

  \mode<presentation>{\vfill\pause}
  \item Un \textgood{futuro} es un objeto que permite que un hilo pueda devolver un valor a la 
    sección de código que lo invocó.
\end{itemize}
\end{frame}

\begin{frame}[t,fragile]
\begin{block}{Invocación de tareas asíncronas}
\begin{lstlisting}
#include <future>
#include <iostream>

int main() {
  std::future<int> r = std::async(tarea, 1, 10); // Ejecuta tarea(1,10)
  otra_tarea();
  std::cout << "Resultado= " << r.get() << "\n";
  return 0;
}
\end{lstlisting}
\end{block}
\end{frame}

\begin{frame}[t]{Uso de futuros}
\begin{itemize}
  \item \textgood{Idea general}:
    \begin{itemize}
      \item Cuando un hilo necesita pasar un valor a otro hilo pone el valor en una \textgood{promesa}.
      \item La implementación hace que el valor esté disponible en el correspondiente \textgood{futuro}.
    \end{itemize}

  \mode<presentation>{\vfill\pause}
  \item Acceso al \textgood{futuro} mediante \cppid{f.get()}:
    \begin{itemize}
      \item Si se ha asignado un valor $\rightarrow$ obtiene el valor.
      \item En otro caso $\rightarrow$ el hilo llamante se bloquea hasta que esté disponible.
      \item Permite la transferencia transparente de excepciones entre hilos.
    \end{itemize}
\end{itemize}
\end{frame}
