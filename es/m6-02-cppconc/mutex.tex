\section{Objetos mutex y variables condición}

\subsection{Objetos mutex}

\begin{frame}[t]{Clasificación de \emph{mutex}}
\begin{itemize}
  \item Representan el acceso exclusivo a un recurso.
    \begin{itemize}
      \item \cppid{mutex}: \emph{Mutex} básico no recursivo.
      \item \cppid{recursive\_mutex}: Un \emph{mutex} que puede ser adquirido más de una vez por un mismo hilo.
      \item \cppid{timed\_mutex}: \emph{Mutex} no recursivo con operaciones con tiempo límite.
      \item \cppid{recursive\_timed\_mutex}: \emph{Mutex} recursivo con operaciones con tiempo límite.
    \end{itemize}

  \mode<presentation>{\vfill\pause}
  \item Solamente un hilo puede poseer un \emph{mutex} en un momento dado.
    \begin{itemize}
      \item Adquirir un \emph{mutex} $\rightarrow$ obtener acceso exclusivo al objeto.
        \begin{itemize}
          \item Operación bloqueante.
        \end{itemize}
      \item Liberar un \emph{mutex} $\rightarrow$ Liberar el acceso exclusivo al objeto.
        \begin{itemize}
          \item Permite que otro hilo obtenga el acceso.
        \end{itemize}
    \end{itemize}
\end{itemize}
\end{frame}

\begin{frame}[t]{Operaciones}
\begin{itemize}
  \item Construcción y destrucción:
    \begin{itemize}
      \item Se puede construir por defecto.
      \item No se pueden copiar ni mover.
      \item El destructor puede dar comportamiento no definido si el \emph{mutex} no está libre.
    \end{itemize}
  \vfill
  \item Adquisición y liberación:
    \begin{itemize}
      \item \cppid{m.lock()}: Adquiere el \emph{mutex} de forma bloqueante.
      \item \cppid{m.unlock()}: Libera el \emph{mutex}.
      \item \cppid{r = m.try\_lock()}: Intenta adquirir el \emph{mutex}, devolviendo indicación de éxito.
    \end{itemize}
  \vfill
  \item Otras:
    \begin{itemize}
      \item \cppid{h = m.native\_handle()}: Devuelve el identificador dependiente de la plataforma de tipo
            \cppid{native\_handle\_type}.
    \end{itemize}
\end{itemize}
\end{frame}

\begin{frame}[t,fragile]{Ejemplo}
\begin{columns}

\begin{column}{.5\textwidth}
\begin{block}{Acceso exclusivo}
\begin{lstlisting}
mutex mutex_salida;

void imprime(int x) {
  std::mutex_salida.lock();
  std::cout << x << "\n";
  std::mutex_salida.unlock();
}

void imprime(double x) {
  std::mutex_salida.lock();
  std::cout << x << "\n";
  std::mutex_salida.unlock();
}
\end{lstlisting}
\end{block}
\end{column}

\begin{column}{.5\textwidth}
\begin{block}{Lanzamiento de hilos}
\begin{lstlisting}
void f() {
  std::thread t1{imprime, 10};
  std::thread t2(imprime, 5.5};
  std::thread t3(imprime, 3);
 
  t1.join(); 
  t2.join(); 
  t3.join();
}
\end{lstlisting}
\end{block}
\end{column}

\end{columns}
\end{frame}

\begin{frame}[t,fragile]{Errores en exclusión mutua}
\begin{itemize}
  \item En caso de error se lanza la excepción \cppid{system\_error}.
  \item Códigos de error:
    \begin{itemize}
      \item \cppid{resource\_deadlock\_would\_occur}.
      \item \cppid{resource\_unavailabe\_try\_again}.
      \item \cppid{operation\_not\_permitted}.
      \item \cppid{device\_or\_resource\_busy}.
      \item \cppid{invalid\_argument}.
    \end{itemize}
\begin{lstlisting}
std::mutex m;
try {
  m.lock();
  // 
  m.lock();
}
catch (std::system_error & e) {
  std::cerr << e.what() << '\n' << e.code() << '\n';
}
\end{lstlisting}
\end{itemize}
\end{frame}

\begin{frame}[t]{Tiempo límite}
\begin{itemize}
  \item Operaciones soportadas por \cppid{timed\_mutex} y \cppid{recursive\_timed\_mutex}.
  \vfill
  \item Añaden operaciones de adquisión con especificación de tiempo límite.
    \begin{itemize}
      \item \cppid{r = m.try\_lock\_for(d)}: Intenta adquirir el \emph{mutex} durante una duración
            \cppid{d}, devolviendo indicación de éxito.
      \item \cppid{r = m.try\_lock\_until(t)}: Intenta adquirir el \emph{mutex} hasta un momento
            en el tiempo, devolviendo indicación de éxito.
    \end{itemize}
\end{itemize}
\end{frame}

\subsection{Variables condición}

\begin{frame}[t]{Variables condición}
\begin{itemize}
  \item Sincronización de operaciones entre hilos.
  \item Optimizada para la clase \cppid{mutex} (alternativa \cppid{condition\_variable\_any})).
  \item Construcción y destrucción:
    \begin{itemize}
      \item \cppid{condition\_variable c\{\}}: Crea una variable condición.
        \begin{itemize}
          \item Puede lanzar \cppid{system\_error}.
        \end{itemize}
      \item Destructor: Destruye la variable condición.
        \begin{itemize}
         \item Requiere que no haya ningún hilo esperando en condición.
        \end{itemize}
      \item No se pueden copiar ni mover.
      \item Antes de destruirla se debe despertar a todos los hilos bloqueados en la variable.
        \begin{itemize}
          \item O se podrían quedar bloquedos para siempre.
        \end{itemize}
    \end{itemize}
\end{itemize}
\end{frame}

\begin{frame}[t]{Operaciones de notificación/espera}
\begin{itemize}
  \item Notificación:
    \begin{itemize}
      \item \cppid{c.notify\_one()}: Despierta a uno de los hilos en espera.
      \item \cppid{c.notify\_all()}: Despierta a todos los hilso en espera.
    \end{itemize}
  \item Espera incondicional (\cppid{l} de tipo \cppid{unique\_lock<mutex>}):
    \begin{itemize}
      \item \cppid{c.wait(l)}: Se bloquea hasta que consigue adquirir el cerrojo \cppid{l}.
      \item \cppid{c.wait\_until(l,t)}: Se bloquea hasta que consigue adquirir el cerrojo \cppid{l} o se llega al tiempo \cppid{t}.
      \item \cppid{c.wait\_for(l,t)}: Se bloquea hasta que consigue adquirir el cerrojo \cppid{l} o pasa la duración \cppid{d}.
    \end{itemize}
  \item Espera con predicados.
    \begin{itemize}
      \item Admiten como argumento adicional un predicado que debe satisfacerse.
    \end{itemize}
\end{itemize}
\end{frame}

\begin{frame}[t,fragile]{Revisando el producto consumidor}
\begin{block}{Inyectando el predicado en wait}
\begin{lstlisting}
void consumidor() {
  for (;;) {
    std::unique_lock<std::mutex> l{m};

    cv.wait(l, [this]{return !cola.empty();});

    auto p = cola.front();
    cola.pop();
    l.unlock();
   
    procesa(p);
  };
}
\end{lstlisting}
\end{block}
\end{frame}

