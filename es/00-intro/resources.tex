\section{Recursos}

\begin{frame}[t]{Bibliografía}
\begin{itemize}
  \item Bibliografía básica:
    \begin{itemize}
      \item \textbf{Computer Architecture: A quantiative approach, 6th Edition}. 
            Hennessy, JL and Patterson, DA. 
            Morgan Kaufmann, 2017.
    \end{itemize}
  \vspace{1em}
  \pause
  \item Bibliografía complementaria:
    \begin{itemize}
      \item \emph{Computer Organization and Design, MIPS Edition: The Hardware Software Interface}. 
            Patterson, DA; Hennessy, JL. 
            Morgan Kaufmann, 2020.
      \item \emph{The OpenMP Common Core: Making OpenMP Simple Again}.
            Mattson, TG; He, Y.; Koniges, A.E.
            MIT Press, 2019. 
      \item \emph{C++ Concurrency in Action. Practical Multithreading, 2nd Edition} 
            Williams, A.  
            Manning. 2018.
    \end{itemize}
\end{itemize}
\end{frame}

\ocwexclude{
\begin{frame}[t]{Otro material}
\begin{itemize}
  \item Los materiales usados en clase se publicarán a través de \emph{Aula Global}.
  \vspace{1em}
  \item \alert{AVISO MUY IMPORTANTE}:
    \begin{itemize}
      \item Las transparencias y otros materiales publicados mediante \emph{Aula Global} solamente son un \textbf{guión de clase}.
        \begin{itemize}
          \item \alert{No son los materiales del curso}.
        \end{itemize}
      \item El conocimiento de los contenidos de dichos guiones es insuficiente para alcanzar los objetivos de la asignatura.
        \begin{itemize}
          \item \alert{Es muy probable que suspendas si no haces más}.
        \end{itemize}
      \item Es altamente recomendable usar, estudiar y trabajar con la bibliografía básica y complementaria.
        \begin{itemize}
          \item \alert{p. ej.: resolución individual de ejercicios y libros}.
        \end{itemize}
    \end{itemize}
\end{itemize}
\end{frame}
}

\ocwonly{
\begin{frame}{Materiales del curso}
\begin{itemize}
  \item Materiales de clase:
    \begin{itemize}
      \item Estructurados en 6 bloques.
      \item Un total de 15 lecciones.
      \item Al final de cada lección:
        \begin{itemize}
          \item Recomendaciones de lecturas adicionales.
          \item Recomendaciones sobre ejercicios.
        \end{itemize}
    \end{itemize}

  \vfill
  \item Lecturas obligatorias:
    \begin{itemize}
      \item Complementan el material de clase.
    \end{itemize}

  \vfill
  \item Ejercicios:
    \begin{itemize}
      \item Colecciones para cada bloque temático.
    \end{itemize}

  \vfill
  \item Prácticas:
    \begin{itemize}
      \item Programación paralela y concurrente.
    \end{itemize}

  \vfill
  \item Exámenes.

\end{itemize}
\end{frame}
}
