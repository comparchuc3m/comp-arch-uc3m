\section{Evaluación}

\begin{frame}[t]{Sistema de evaluación}
\begin{itemize}
  \item \textmark{Resumen}:

  \mode<presentation>{\vfill}
  \begin{itemize}
    \item \textgood{Examen final}: 40\% de la calificación final.
      \begin{itemize}
        \item Incluye todos los contenidos (teoría, prácticas, y proyectos).
      \end{itemize}

    \mode<presentation>{\vfill}
    \item \textgood{Evaluación continua}: 60\% de la calificación final.
      \begin{itemize}
        \item \textmark{Evaluaciones durante el curso}: 25\% de la calificación final.
        \item \textmark{Prácticas}: 35\% de la calificación final.
      \end{itemize}
  \end{itemize}

  \mode<presentation>{\vfill}
  \begin{itemize}
    \item \textemph{Convocatorias}:
      \begin{itemize}
        \item \textmark{Ordinaria}: Enero.
        \item \textmark{Extraordinaria}: Junio.
      \end{itemize}
  \end{itemize}
\end{itemize}
\end{frame}

\begin{frame}[t]{Evaluación continua}
\begin{itemize}
  \item Obtener buen resultado en la evaluación continua es \textemph{clave} para superar 
        la asignatura.
  
  \mode<presentation>{\vfill}
  \item \textgood{Elementos}:
    \begin{itemize}
      \item \textmark{Evaluaciones durante el curso}: 25\% de la calificación final.
      \item \textmark{Prácticas y laboratorios}: 35\% de la calificación final.
    \end{itemize}
  
  \mode<presentation>{\vfill}
  \item \textbad{No has seguido} la evaluación continua si:
    \begin{itemize}
      \item obtienes menos de 2.0 en una práctica/laboratorio.
    \end{itemize}
\end{itemize}
\end{frame}

\begin{frame}[t]{Convocatoria ordinaria: Evaluación continua}
\begin{itemize}
  \item Si sigues el proceso de \textmark{evaluación continua}:
    \begin{itemize}
    \item \textgood{Examen final}: 40\%.
      \begin{itemize}
        \item Mínimo necesario: \textbad{3.5}.
      \end{itemize}
    \item \textgood{Evaluaciones durante el curso}: 25\%
      \begin{itemize}
        \item Mínimo necesario: \alert{No hay mínimo}.
      \end{itemize}
    \item \textgood{Prácticas y Laboratorios}: 35\%.
      \begin{itemize}
        \item Mínimo en cada práctica: \textbad{2.0}.
      \end{itemize}
    \item Si no logras algún mínimo, la media no se calcula y serás calificado como suspenso.
  \end{itemize}

  \mode<presentation>{\vfill}
  \item \textemph{Bonus}:
    \begin{itemize}
      \item Se añadirá un punto adicional a la calificación final si:
        \begin{itemize}
          \item Obtienes al menos 7.0 puntos en la evaluación continua, y además
          \item obtienes al menos 6.0 puntos en el examen final.
        \end{itemize}
    \end{itemize}
\end{itemize}
\end{frame}

\begin{frame}[t]{Evaluaciones durante el curso}
\begin{itemize}
  \item Cuestionarios realizados \textmark{durante clase}.

  \mode<presentation>{\pause\vfill}
  \item Cada cuestionario \textmark{puede incluir}:
    \begin{itemize}
      \item Preguntas teóricas.
      \item Preguntas prácticas.
      \item Resolución de problemas.
    \end{itemize}

  \mode<presentation>{\pause\vfill}
  \item Las fechas de cada prueba están \textemph{ya publicadas}.

  \mode<presentation>{\pause\vfill}
  \item La calificación de esta parte será el promedio de las 
        \textemph{4 mejores calificaciones}
        de los cuestionarios. 
    \begin{itemize}
      \item No presentado $\Rightarrow$ 0.
    \end{itemize}
\end{itemize}
\end{frame}

\begin{frame}[t]{Calendario de pruebas de evaluación}
\begin{itemize}
  \item Todas las pruebas de evaluación se realizan en el grupo reducido
        en el que el estudiante está \textbad{oficialmente matriculado}.

  \vfill
  \item \textmark{Calendario}:
  \begin{itemize}
    \item Semana 4 (26/09 -- 30/09): Introducción, fundamentos y rendimiento.
    \item Semana 6 (10/10 -- 14/10): Jerarquía de memoria.
    \item Semana 10 (7/11 -- 11/11): Paralelismo a nivel de instrucción.
    \item Semana 13 (28/11 -- 2/12): Multiprocesadores.
    \item Semana 14 (28/11 -- 13/12): Programación paralela y concurrente.
  \end{itemize}
\end{itemize}
\end{frame}


\begin{frame}[t]{Prácticas/Laboratorios}
\begin{itemize}
  \item Cuatro laboratorios.
    \begin{itemize}
{\scriptsize
      \item Realización durante la clase práctica en grupos de 2 personas.
      \item Entrega de cuestionario o memoria durante un plazo
            establecido.
      \item \textgood{Peso de cada laboratorio}: 3\% de la nota final.
      \item \textbad{Calificación mínima}: 2 puntos sobre 10.
}
    \end{itemize}

  \mode<presentation>{\vfill\pause}
  \item Una proyecto de programación orientada al rendimiento.
    \begin{itemize}
{\scriptsize
      \item Realizada usando C++ y OpenMP.
      \item Dos entregas.
        \begin{itemize}
          {\scriptsize
          \item Semana 8 (24/10): Optimizaciones secuenciales.
          \item Semana 11 (14/11): Optimizaciones paralelas.
          }
        \end{itemize}
      \item Grupos de 4 personas.
      \item El \textmark{rendimiento} será un criterio clave.
      \item Se valorará la \textmark{calidad del código} y las 
            \textmark{pruebas y evaluaciones} realizadas.
      \item Muy importante la \textmark{calidad de la memoria} elaborada.
      \item \textgood{Peso de la práctica}: 23\% de la nota final.
      \item \textbad{Calificación mínima}: 2 puntos sobre 10.
}
    \end{itemize}
\end{itemize}
\end{frame}

\begin{frame}[t]{Sesiones de laboratorio}
\begin{itemize}
  \item 6 sesiones de laboratorio.

  \mode<presentation>{\vfill}
  \item \textmark{Calendario}:
    \begin{enumerate}
      \item Semana 2 (12/09 -- 16/09): Laboratorio C++ básico.
        \begin{itemize}
          \item Sin entrega asociada.
          \item Preparación de proyecto.
        \end{itemize}
      \item Semana 3 (19/09 -- 23/09): Laboratorio de memoria caché.
        \begin{itemize}
          \item Entrega en semana 4 (26/09).
        \end{itemize}
      \item Semana 5 (3/10 -- 7/10): Laboratorio de memoria caché.
        \begin{itemize}
          \item Entrega en semana 6 (10/10).
        \end{itemize}
      \item Semana 8 (24/10 -- 28/10): Laboratorio de OpenMP.
        \begin{itemize}
          \item Sin entrega asociada.
          \item Preparación de proyecto.
        \end{itemize}
      \item Semana 9 (31/10 -- 4/11): Laboratorio de ILP.
        \begin{itemize}
          \item Entrega en semana 10 (7/11).
        \end{itemize}
      \item Semana 12 (21/11 -- 25/11): Lab. consistencia memoria.
        \begin{itemize}
          \item Entrega en semana 13 (28/11).
        \end{itemize}
    \end{enumerate}
\end{itemize}
\end{frame}

\begin{frame}[t]{Convocatoria ordinaria: Evaluación NO-continua}
\begin{itemize}
  \item Si \textbad{no has seguido} el proceso de \textgood{evaluación continua}:
    \begin{itemize}
      \item El \textgood{examen final} tiene un valor del 
            \textbad{60\%} de la calificación final.
      \item Necesitarás \textbad{8.33} en el \textgood{examen final} 
            para superar la asignatura.
    \end{itemize}

  \mode<presentation>{\vfill}
  \item \textbad{CONSEJO}:
    \begin{itemize}
      \item Pon esfuerzo en seguir el proceso de evaluación continua.
    \end{itemize}
\end{itemize}
\end{frame}


\begin{frame}[t]{Convocatoria extraordinaria}
\begin{itemize}
  \item \textgood{Examen extraordinario} en el mes de junio.

  \mode<presentation>{\vfill}
  \item \textmark{Normas}:
    \begin{enumerate}
      \item Estudiantes que \textgood{han completado} el proceso de evaluación continua:
        \begin{itemize}
          \item El examen extraordinario vale el \textbad{40\%} 
                y la evaluación continua el otro \textbad{60\%}.
          \item \textbad{Solamente se aplica} si la calificación en el examen es de 
                al menos \textbad{3.5}.
        \end{itemize}
      \item Estudiantes que \textbad{no han completado} el proceso de evaluación continua:
        \begin{itemize}
          \item El examen vale el \textbad{100\%}.
        \end{itemize}
    \end{enumerate}

    \mode<presentation>{\vfill}
    \begin{itemize}
      \item Para los estudiantes que hayan completado el proceso de evaluación continua 
            se tomará \textgood{la opción más favorable}.
    \end{itemize}
\end{itemize}
\end{frame}

\begin{frame}[t]{Pruebas de evaluación}
\begin{itemize}
  \item Todas las pruebas de evaluación \textbad{deberán realizarse} 
        en el grupo en que el estudiante se encuentra \textmark{oficialmente matriculado}.
    \begin{itemize}
      \item No se admitirán cambios de grupo que no se realicen oficialmente.
    \end{itemize}

  \mode<presentation>{\vfill}
  \item \textbad{MUY IMPORTANTE}:   
\begin{itemize}
      \item La no asistencia al examen final 
            implica la calificación como \textbad{NO-PRESENTADO}, 
            independientemente de cualquier otra calificación.
    \end{itemize}
\end{itemize}
\end{frame}
