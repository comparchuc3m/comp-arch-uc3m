\section{Secciones y planificación}

\begin{frame}[t,fragile]{Secciones}
\begin{columns}[T]

\column{.6\textwidth}
\begin{itemize}
  \item Define un conjunto de secciones de código.
  \item Cada sección se pasa a un hilo distinto.
  \item Por defecto hay una barrera al final del bloque \cppkey{sections}
\end{itemize}

\mode<presentation>{\vfill\pause}
\column{.4\textwidth}
\begin{block}{Ejemplo}
\begin{lstlisting}
#pragma omp parallel
{
  #pragma omp sections
  {
    #pragma omp section
    f();
    #pragma omp section
    g();
    #pragma omp section
    h();
  }
}
\end{lstlisting}
\end{block}

\end{columns}
\end{frame}

\begin{frame}[t]{Planificación de bucles}
\begin{itemize}
  \item \cppkey{schedule(static)} | \cppkey{schedule(static,n)}:
    \begin{itemize}
      \item Planifica bloques de iteraciones (de tamaño $n$) para cada hilo.
    \end{itemize}
  
  \mode<presentation>{\vfill\pause}
  \item \cppkey{schedule(dynamic)} | \cppkey{schedule(dynamic,n)}:
    \begin{itemize}
      \item Cada hilo toma un bloque de $n$ iteraciones de una cola hasta que se han procesado todas.
    \end{itemize}
  
  \mode<presentation>{\vfill\pause}
  \item \cppkey{schedule(guided)} | \cppkey{schedule(guided,n)}:
    \begin{itemize}
      \item Cada hilo toma un bloque iteraciones hasta que se han procesado todas. Se 
            comienza con un tamaño de bloque grande y se va reduciendo hasta llegar a
            un tamaño $n$.
    \end{itemize}
  
  \mode<presentation>{\vfill\pause}
  \item \cppkey{schedule(runtime)} | \cppkey{schedule(runtime,n)}:
    \begin{itemize}
      \item Se usa lo indicado en \cppid{OMP\_SCHEDULE} o por la biblioteca
            en tiempo de ejecución.
    \end{itemize}
  
\end{itemize}
\end{frame}
