\section{Protocolo basado en directorio}

\begin{frame}[t]{Transición de estados}
\begin{itemize}
  \item \textgood{En chips multicore}:
    \begin{itemize}
      \item La \textmark{coherencia interna} se mantiene mediante \textgood{directorio centralizado}.
      \item El \textmark{mismo directorio} puede actuar como \textmark{directorio local} en \textgood{DSM}.
    \end{itemize}

  \mode<presentation>{\vfill\pause}
  \item \textgood{Implementación del protocolo}:
    \begin{itemize}
      \item Transición de estados de \textmark{caché local}.
        \begin{itemize}
          \item Envían peticiones a \textmark{directorio local}.
        \end{itemize}
      \item Transición de estados del \textmark{directorio}.
    \end{itemize}
\end{itemize}
\end{frame}

\begin{frame}[t]{Transición de estados de caché individual}
\makebox[\textwidth][c]{
\input{es/m5-02-shmem/state-ind.tkz}
}
\end{frame}

\begin{frame}[t]{Entrada no cacheada}
\begin{itemize}
  \item El valor de memoria está actualizado.

  \mode<presentation>{\vfill\pause}
  \item \textgood{Peticiones}:
    \begin{itemize}
      \mode<presentation>{\vfill}
      \item \textmark{Fallo de lectura}:
        \begin{itemize}
          \item Se envía dato de memoria a nodo peticionario.
          \item Nodo peticionario es el único en estado compartido.
          \item Estado pasa a compartido.
        \end{itemize}
      \mode<presentation>{\vfill}
      \item \textmark{Fallo de escritura}:
        \begin{itemize}
          \item Se envía dato de memoria a nodo peticionario.
          \item El bloque se pasa a estado exclusivo.
          \item Nodo peticionario es el propietario.
        \end{itemize}
    \end{itemize}
\end{itemize}
\end{frame}

\begin{frame}[t]{Entrada compartida}
\begin{itemize}
  \item El valor de memoria está actualizado.

  \mode<presentation>{\vfill\pause}
  \item \textgood{Peticiones}:
    \begin{itemize}

      \mode<presentation>{\vfill}
      \item \textmark{Fallo de lectura}:
        \begin{itemize}
          \item Se envía el dato de memoria al nodo peticionario.
          \item El nodo peticionario se añade al conjunto de nodos de la entrada.
        \end{itemize}

      \mode<presentation>{\vfill}
      \item \textmark{Fallo de escritura}:
        \begin{itemize}
          \item Se envía el dato de memoria al nodo peticionario.
          \item Se envían mensajes de invalidación al conjunto de nodos de la entrada.
          \item Se activa en el conjunto solamente el nodo peticionario.
          \item Se pasa a estado exclusivo.
        \end{itemize}
    \end{itemize}
\end{itemize}
\end{frame}

\begin{frame}[t]{Entrada exclusiva}
\begin{itemize}
  \item El valor del bloque se encuentra en caché en el nodo identificado por el conjunto (nodo porpietario).

  \mode<presentation>{\vfill\pause}
  \item \textgood{Peticiones}:
    \begin{itemize}
      \item \textmark{Fallo de lectura}:
        \begin{itemize}
          \item Se envía mensaje de captación a propietario.
          \item Se escribe dato en memoria.
          \item Se envía dato a nodo peticionario.
          \item Se añade nodo peticionario a conjunto de nodos.
        \end{itemize}
    \end{itemize}
\end{itemize}
\end{frame}

\begin{frame}[t]{Entrada exclusiva}
\begin{itemize}
  \item \textgood{Peticiones}:
    \begin{itemize}
      \item \textmark{Post-escritura}:
        \begin{itemize}
          \item Ocurre cuando el propietario hace post-escritura del bloque.
          \item El bloque pasa a estado no cacheado.
          \item Se vacía el conjunto de la entrada.
        \end{itemize}
      \item \textmark{Fallo de escritura}:
        \begin{itemize}
          \item El bloque tiene nuevo propietario.
          \item Se invalida bloque en antiguo propietario y se obtiene valor.
          \item Se envía valor a nodo peticionario.
          \item Se activa en el conjunto solamente el nuevo peticionario.
        \end{itemize}
    \end{itemize}
\end{itemize}
\end{frame}

\begin{frame}[t]{Transición de estados del directorio}
\makebox[\textwidth][c]{
\input{es/m5-02-shmem/state-dir.tkz}
}
\end{frame}

