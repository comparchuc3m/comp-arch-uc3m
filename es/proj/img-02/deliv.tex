\section{Calificación}

La puntuación final obtenida en este proyecto se obtiene teniendo en cuenta el siguiente reparto: 

\begin{itemize}
  \item Rendimiento alcanzado: 20\%.
  \item Consumo energético alcanzado: 20\%.
  \item Código modificado y explicación: 15\%.
  \item Evaluación del rendimiento y la energía: 20\%.
  \item Organización del trabajo: 10\%.
  \item Conclusiones: 15\%.
\end{itemize}

\textgood{Advertencias}:

\begin{itemize}
  \item Si el código entregado no compila, la nota final de la práctica será de 0. 
  \item Si se ignora injustificadamente alguna norma de calidad de código,
        la nota final de la práctica será de 0.
  \item En caso de copia todos los grupos implicados obtendrán una nota de 0.
        Además se notificará a la dirección de la EPS para la correspondiente
        apertura de expediente disciplinario.
\end{itemize}

\section{Procedimiento de entrega}

La entrega del proyecto se realizará a través de Aula Global.

Para ello se habilitarán 2 entregadores separados:

\begin{itemize}

\item \textgood{Entregador de memoria}. Contendrá la memoria del proyecto, que será
un archivo en formato pdf con el nombre \textmark{memoria.pdf}.

\item \textgood{Entregador de código}: Contendrá todo el código fuente
necesario para compilar la versión paralela.
\begin{itemize}
  \item Debe ser un archivo comprimido (formato zip) con el nombre
        \textmark{image.zip}.
\end{itemize}

\end{itemize}

La memoria no deberá exceder de 15 páginas con una fuente mínima de 10 puntos
incluyendo la portada y todas las secciones.  no se tendrá en cuenta en la
corrección los contenidos a partir de la página 16 si fuese el caso.  deberá
contener, al menos, las siguientes secciones: 

\begin{itemize}

\item \textmark{Página de título}: contendrá los siguientes datos:
  \begin{itemize}
    \item Nombre de la práctica.
    \item Nombre del grupo reducido en el que están matriculados los estudiantes.
    \item Número de equipo asignado.
    \item Nombre y NIA de todos los autores.
  \end{itemize}

\item \textmark{Paralelización}.
      Debe incluir una explicación de las modificaciones realizados sobre
      el código suministrado. 

\item \textmark{Evaluación de rendimiento y energía}: 
      Deberá incluir las evaluaciones de rendimiento y energía llevadas a cabo.
      Se deberá evaluar el rendimiento para 1, 2, 4, 8 y 16 hilos.
      Se deberá evaluar el impacto de las distintas políticas de planificación 
      disponibles en OpenMP.

\item \textmark{Organización del trabajo}:
      Deberá describir la organización del trabajo entre los miembros del equipo
      haciendo explícita las tareas llevadas a cabo por cada persona.
      Debe incluir el tiempo dedicado por cada persona a cada tarea.
      No se podrá asignar más de una persona a una tarea.

\item \textmark{Conclusiones}.
      Se valorará especialmente las derivadas de los resultados de la evaluación
      del rendimiento, así como las que relacionen el trabajo realizado con el contenido
      de la asignatura.

\end{itemize}
