\section{Tareas}

\subsection{Desarrollo de software}

El proyecto consiste en la paralelización de funcionalidades concretas del proyecto
suministrado.

En particular se deben paralelizar las siguientes funcionalidades:

\begin{itemize}
  \item Transformación a escala de grises:
    \begin{itemize}
      \item Función \cppid{bitmap\_aos::to\_gray()}.
      \item Función \cppid{bitmap\_soa::to\_gray()}.
    \end{itemize}

  \item Generación de histograma:
    \begin{itemize}
      \item Función \cppid{bitmap\_aos::generate\_histogram()}.
      \item Función \cppid{bitmap\_soa::generate\_histogram()}
    \end{itemize}

  \item Difusión mediante filtro de Gauss:
    \begin{itemize}
      \item Función \cppid{bitmap\_aos::gauss()}.
      \item Función \cppid{bitmap\_soa::gauss()}
    \end{itemize}
\end{itemize}

Tenga también en cuenta que deberá realizar todas las evaluaciones con las
optimizaciones del compilador activadas con la opción de CMake
\cppid{-DCMAKE\_BUILD\_TYPE=Release}.


\subsubsection{Normas de calidad del código}

Se deberán seguir manteniendo todas las normas aplicables al proyecto anterior.

\textbad{MUY IMPORTANTE}: El código no deberá depender de ninguna forma del 
número de hilos ni de la política de planificación concreta. Estos parámetros
se controlarán mediante las variables de entorno correspondiente externas
al programa.

\subsubsection{Pruebas unitarias}

Se suministran un conjunto de pruebas unitarias. Los equipos deberán comprobar
que al modificar el código todas las pruebas unitarias siguen teniendo éxito.

\subsection{Evaluación del rendimiento y energía}

Esta tarea consiste en realizar una evaluación comparativa del rendimiento
de las dos estrategias de implementación \cppid{image-aos} e \cppid{image-soa}.

En cada caso se deben realizar evaluaciones variando los siguientes parámetros:
\begin{itemize}
  \item \textmark{Número de hilos}: Se evaluará el impacto del número de hilos
        considerando los valores 1, 2, 4, 8 y 16. Se puede controlar el número de hilos
        mediante la variable de entorno \cppid{OMP\_NUM\_THREADS}.

  \item \textmark{Política de planificación}: Se considerarán todas las
        políticas de planificación disponibles en OpenMP. Se puede controlar
        la política de planificación mediante la variable de entorno 
        \cppid{OMP\_SCHEDULE}.
      
\end{itemize}

Para evaluar el rendimiento debe medir el tiempo de ejecución de la
aplicación. También debe medirse el consumo energético de la aplicación
y derivar el uso de potencia.

Todas las evaluaciones de rendimiento se realizarán en un nodo del
clúster \cppid{avignon}.

Represente en una gráfica todos los tiempos totales de ejecución, uso de energía y potencia
obtenidos para la imagen suministrada \cppkey{sabatini.bmp}.

\textbf{Incluya en la memoria de esta práctica las conclusiones que pueda inferir de los resultados.}
No se limite simplemente a describir los datos. Debe buscar también una
explicación convincente de los resultados.
