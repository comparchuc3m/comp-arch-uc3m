\section{Descripción del proyecto}

En este proyecto se construirá una aplicación de procesamiento de imágenes.

En las siguientes secciones se presentan los requisitos que deben cumplir
con los programas desarrollados.

\subsection{Visión general}

Se desarrollará una aplicación que aplica diversas operaciones
sobre una imagen en formato BMP.

Para cada imagen se implementarán las siguientes operaciones:

\begin{itemize}
\item Copia de imágenes.
\item Obtención del histograma.
\item Conversión a escala de grises.
\item Difusión gaussiana (\emph{Gaussian blur}).
\end{itemize}

La aplicación tomará los siguientes parámetros:

\begin{itemize}

\item Directorio con imágenes de entrada.
\item Directorio en el que dejar las imágenes de salida.
\item Operación a ejecutar.

\end{itemize}

La operación a ejecutar será una de las siguientes:

\begin{itemize}

\item \textmark{copy}: No realiza ninguna transformación. Solamente
copia los archivos.
\item \textmark{histo}: Genera un archivo de texto con el histograma de cada canal (RGB).
\item \textmark{mono}: Convierte a tonos de grises.
\item \textmark{gauss}: Aplica el filtro de suavizado gaussiano.

\end{itemize}

\textbad{NOTA}: Se implementarán dos programas distintos \cppid{image-aos}
e \cppid{image-soa}. No obstante, por simplicidad en el resto de esta
sección se utiliza el nombre \cppid{image} para hacer referencia
indistintamente a los dos programas.
Para más detalles sobre estos dos programas vaya a la sección~\ref{sec:tasks:versions}.

En cualquiera de los casos, la aplicación leerá todos los archivos
del directorio de entrada, aplicará las transformaciones
correspondientes y escribirá los archivos correspondientes con el
mismo nombre en el directorio de salida.

El programa deberá comprobar que el número de parámetros es 3:

\begin{lstlisting}[style=terminal]
$image
Wrong format:
  image in_path out_path oper
    operation: copy, histo, mono, gauss
$
\end{lstlisting}

Si el tercer argumento no toma un valor adecuado (\cppid{copy}, \cppid{histo}, \cppid{mono} o \cppid{gauss}), 
se presentará un mensaje de error y se terminará:

\begin{lstlisting}[style=terminal]
$image indir outdir gaus
Unexpected operation:gaus
  image in_path out_path oper
    operation: copy, histo, mono, gauss
$
\end{lstlisting}

Si el directorio de entrada no existe o no puede abrirse, se presentará un mensaje de error y se terminará:

\begin{lstlisting}[style=terminal]
$image inx outdir copy
Input path: inx
Output path: out
Cannot open directory [inx]
  image in_path out_path oper
    operation: copy, histo, mono, gauss
$
\end{lstlisting}

Si el directorio de salida no existe, se presentará un mensaje de error y se terminará:

\begin{lstlisting}[style=terminal]
$image-seq sobel indir outx
Input path: indir
Output path: outx
Output directory [outx] does not exist
  image in_path out_path oper
    operation: copy, histo, mono, gauss
$
\end{lstlisting}

En cualquier otro caso, se procesarán todos los archivos del directorio de entrada y se dejarán los resultados en el directorio de salida. Para cada archivo procesado se imprimirá la información del tiempo dedicado a cada tarea. Todos los valores de tiempo se expresarán en microsegundos:

\begin{lstlisting}[style=terminal]
$image-seq indir outdir gauss
Input path: indir
Output path: outdir
File: "indir/62096.bmp"(time: 43994)
  Load time: 3442
  Gauss time: 38909
  Store time: 1642
File: "indir/169012.bmp"(time: 37209)
  Load time: 2602
  Gauss time: 32950
  Store time: 1655
$
\end{lstlisting}


\subsection{El formato BMP}

Se utilizará el formato de imágenes BMP. Los ficheros de este tipo
contienen una cabecera de al menos 54 bytes con la  información
especificada en el cuadro~\ref{tab:bmp}.

\begin{table}[!htb]
\centering
\begin{tabular}{|l|l|l|}
\hline
\textbf{Comienzo} & \textbf{Longitud} & \textbf{Descripción}\\
\hline
\hline
0 & 2 & Caracteres \cppid{'B'} y \cppid{'M'}.\\
\hline
2 & 4 & Tamaño del archivo.\\
\hline
6 & 4 & Reservado.\\
\hline
10 & 4 & Inicio de datos de imagen.\\
\hline
14 & 4 & Tamaño de la cabecera de bitmap.\\
\hline
18 & 4 & Anchura en pixeles.\\
\hline
22 & 4 & Altura en pixeles.\\
\hline
26 & 2 & Número de planos.\\
\hline 
28 & 2 & Tamaño de punto en bits.\\
\hline
30 & 4 & Compresión.\\
\hline
34 & 4 & Tamaño de la imagen.\\
\hline
38 & 4 & Resolución horizontal.\\
\hline
42 & 4 & Resolución vertical.\\
\hline
46 & 4 & Tamaño de la tabla de color.\\
\hline
50 & 4 & Contador de colores importantes.\\
\hline
\end{tabular}
\caption{Campos de cabecera para el formato BMP.}
\label{tab:bmp}
\end{table}

Se tendrán en cuenta las siguientes restricciones para que un fichero
se pueda considerar válido.

\begin{itemize}
\item El número de planos debe ser \cppid{1}.
\item El tamaño de cada punto debe ser de \cppid{24} bits.
\item El valor de compresión debe ser \cppid{0}.
\end{itemize}

Obsérvese que no será necesario utilizar ciertos campos.

Los pixeles se almacenan a partir de la posición indicada por el campo
\emph{Inicio de datos de imagen}. Cada pixel se representa por 3 bytes
consecutivos que representan el nivel de azul, verde y rojo para ese
pixel. Dichos valores estarán siempre en el rango de \cppid{0} a
\cppid{255}.

Al finalizar los píxeles de una fila se dejarán bytes de relleno, de forma
que la siguiente fila de píxeles comience al principio de una palabra.

\subsection{Histograma de una imagen}

Por cada imagen se generará un archivo de texto con extensión \cppid{.hst}
que contendrá consecutivamente la información de los tres histogramas.
Esto es, se generará un histograma para el canal R, otro histograma para
el canal G, y un tercer histograma para el canal B.

Un histograma representa la frecuencia absoluta (número de apariciones) de 
cada valor numérico (de $0$ a $255$).

El archivo de salida será un archivo de texto que contendrá 
un total de 768 ($256 \times 3$) líneas.
Las primeras 256 líneas contendrán los valores de frecuencia absoluta de cada
valor para la componente R. Las siguientes 256 líneas serán para la componente
G. Las últimas 256 líneas serán para la componente B.

\subsection{Conversión a escala de grises}

Una imagen en escala de grises tiene el mismo valor para los tres valores RGB del pixel.

La transformación se realiza en cuatro pasos:

\begin{enumerate}

\item Normalización: Se normalizan los valores a una escala real de \cppid{0} to \cppid{1}.

\[
n_r = \frac{R}{255}
\]
\[
n_g = \frac{G}{255}
\]
\[
n_b = \frac{B}{255}
\]

\item Transformación a intensidad lineal: Para cada color normalizado
$(n_r, n_g, n_b)$ se obtiene un nuevo color $(c_r, c_g, c_b)$.
Para ellos se aplica la misma transformación en los tres canales:

\[
c_i = \frac{n_i}{12.92} \quad\quad \text{si} \quad n_i \leq 0.04045
\]
\[
c_i = \left(\frac{n_i + 0.055}{1.055}\right)^{2.4} \quad\quad \text{si} \quad n_i > 0.04045
\]

\item Transformación lineal: Se aplica una transformación lineal a cada color.
\[
c_l = 0.2126 \cdot c_r + 0.7152 \cdot c_g + 0.0722 \cdot c_b
\]

\item Corrección gamma: 
\[
c_g = 12.92 \cdot c_l \quad\quad \text{si} \quad c_l \leq 0.0031308
\]
\[
c_g = 1.055 \cdot c_l^{\frac{1}{2.4}} - 0.055 \quad\quad \text{si} \quad c_l > 0.0031308
\]

\item Desnormalización: Se transforma de la escala $[0,1]$ a la escala $[0,255]$
\[
g = c_g \cdot 255
\]

\end{enumerate}

\textbad{NOTA}:
Tenga en cuenta que debe almacenarse el valor obtenido $g$, para las tres componentes
(R, G y B) del pixel.

\subsection{Difusión gaussiana}

La difusión gaussiana tiene por objeto disminuir la nitidez de una
imagen y se utiliza como paso previo a la detección de bordes.

En general, se utiliza una matriz cuadrada que actúa como máscara.
Para una máscara de $5 \times 5$ la imagen resultado $res$, se obtiene
a partir de la máscara $m$ y de la imagen original $im$ aplicando 
la siguiente expresión:

\[
res(i,j) = \frac{1}{w} \sum_{s=-2}^2 \sum_{t=-2}^2 m(s+3,t+3) \cdot
im(i+s,j+t)
\]

donde la matriz de la máscara $m$ y el peso $w$ son:

\[
m =
\begin{pmatrix}
1 & 4 & 7 & 4 & 1\\ 
4 & 16 & 26 & 16 & 4\\
7 & 26 & 41 & 26 & 7\\
4 & 16 & 26 & 16 & 4\\
1 & 4 & 7 & 4 & 1\\
\end{pmatrix}
\qquad
\qquad
w = 273
\]


\textbf{Importante}: Tenga en cuenta que al aplicar la máscara en los 
bordes de la imagen, se considerará que la matriz de la imagen está
rodeada de pixeles imaginarios de valor \cppid{0}.

