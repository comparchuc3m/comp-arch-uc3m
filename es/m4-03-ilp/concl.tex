\section{Conclusión}

\begin{frame}[t]{Resumen}
\begin{itemize}[<+->]
  \mode<presentation>{\vfill}
  \item La planificación dinámica gestiona detenciones desconocidas en tiempo de compilación.
  
  \mode<presentation>{\vfill}
  \item Las técnicas especulativas se apoyan de la predicción de saltos y la planificación dinámica.
  
  \mode<presentation>{\vfill}
  \item La emisión múltiple en ILP queda limitada de forma práctica de 3 a 6.
  
  \mode<presentation>{\vfill}
  \item SMT como aproximación a TLP dentro un núcleo.
\end{itemize}
\end{frame}


\begin{frame}[t]{Referencias}
\begin{itemize}
  \item \bibhennessy
  Secciones 3.3, 3.4, 3.6, 3.7, 3.10, 3.12.

\end{itemize}
\end{frame}
