\section{Técnicas de emisión múltiple}

\begin{frame}[t]{CPI $<$ 1}
\begin{itemize}
  \item CPI $\geq$ 1 $\rightarrow$ Emisión de una instrucción por ciclo.

  \mode<presentation>{\vfill\pause}
  \item Procesadores de múltiple emisión 
        (\textmark{CPI} $<$ 1 $\rightarrow$ \textmark{IPC} $>$ 1):
    \begin{itemize}
      \mode<presentation>{\vfill}
      \item Procesadores superescalares planificados \textgood{estáticamente}.
        \begin{itemize}
          \item Ejecución en orden.
          \item Número variable de instrucciones por ciclo.
        \end{itemize}
      \mode<presentation>{\vfill}
      \item Procesadores superescalares planificados \textgood{dinámicamente}.
        \begin{itemize}
          \item Ejecución fuera de orden.
          \item Número variable de instrucciones por ciclo.
        \end{itemize}
      \mode<presentation>{\vfill}
      \item Procesadores \textgood{VLIW} (\emph{Very Long Instruction Word}).
        \begin{itemize}
          \item Varias instrucciones empaquetadas en paquete.
          \item Planificación estática.
          \item Paralelismo explícito a nivel de instrucción por el compilador.
        \end{itemize}
    \end{itemize}
\end{itemize}
\end{frame}

\begin{frame}[t]{Enfoques de la emisión múltiple}
\begin{itemize}
  \item Varios enfoques posibles con emisión múltiple.
    \begin{itemize}
      \item \textmark{Superescalar estático}.
      \item \textmark{Superescalar dinámico}.
      \item \textmark{Superescalar especulativo}.
      \item \textmark{VLIW/LIW}.
      \item \textmark{EPIC}.
    \end{itemize}
\end{itemize}
\end{frame}

\begin{frame}[t]{Superescalar estático}
\begin{itemize}
  \item \textgood{Emisión}: Dinámica.
  \item \textgood{Detección de riesgos}: Hardware.
  \item \textgood{Planificación}: Estática.
  \item \textgood{Propiedad discriminante}:
    \begin{itemize}
      \item Ejecución en orden.
    \end{itemize}
  
  \mode<presentation>{\vfill}
  \item \textmark{Ejemplos}:
    \begin{itemize}
      \item MIPS.
      \item ARM Cortex-A7.
    \end{itemize}
\end{itemize}
\end{frame}

\begin{frame}[t]{Superescalar dinámico}
\begin{itemize}
  \item \textgood{Emisión}: Dinámica.
  \item \textgood{Detección de riesgos}: Hardware.
  \item \textgood{Planificación}: Dinámica.
  \item \textgood{Propiedad discriminante}:
    \begin{itemize}
      \item Ejecución fuera de orden sin especulación.
    \end{itemize}
  
  \mode<presentation>{\vfill}
  \item \textmark{Ejemplos}: Ninguno.
\end{itemize}
\end{frame}

\begin{frame}[t]{Superescalar especulativo}
\begin{itemize}
  \item \textgood{Emisión}: Dinámica.
  \item \textgood{Detección de riesgos}: Hardware.
  \item \textgood{Planificación}: Dinámica con especulación.
  \item \textgood{Propiedad discriminante}:
    \begin{itemize}
      \item Ejecución fuera de orden con especulación.
    \end{itemize}
  
  \mode<presentation>{\vfill}
  \item \textmark{Ejemplos}: 
    \begin{itemize}
      \item Intel Core i3, i5, i7. 
      \item AMD Phenom. 
      \item IBM Power 7
    \end{itemize}
\end{itemize}
\end{frame}

\begin{frame}[t]{VLIW}
\begin{itemize}
  \item Empaquetado de varias operaciones en una instrucción.

  \mode<presentation>{\vfill}
  \item Instrucción ejemplo en ISA VLIW:
    \begin{itemize}
      \item Una instrucción entera o una bifurcación.
      \item Dos operaciones de coma flotante independientes.
      \item Dos referencias a memoria independientes.
    \end{itemize}

  \mode<presentation>{\vfill}
  \item \textmark{IMPORTANTE}: El código debe presentar suficiente paralelismo.
\end{itemize}
\end{frame}

\begin{frame}[t]{VLIW / LIW}
\begin{itemize}
  \item \textgood{Emisión}: Estática.
  \item \textgood{Detección de riesgos}: Principalmente software.
  \item \textgood{Planificación}: Estática.
  \item \textgood{Propiedad discriminante}:
    \begin{itemize}
      \item Todos los riesgos determinados e indicados por el compilador.
    \end{itemize}
  
  \mode<presentation>{\vfill}
  \item \textmark{Ejemplos}: 
    \begin{itemize}
      \item DSPs (p. ej. TI C6x).
    \end{itemize}
\end{itemize}
\end{frame}

\begin{frame}[t]{Problemas de VLIW}
\begin{itemize}
  \item \textbad{Desventajas} del modelo original \textmark{VLIW}:
    \begin{itemize}
      \item Complejidad de encontrar paralelismo estáticamente.
      \item Tamaño del código generado.
      \item No tienen hardware de detección de riesgos.
      \item Problemas de compatibilidad binaria mayores que en superescalares.
    \end{itemize}

  \mode<presentation>{\vfill}
  \item \textgood{EPIC} intenta resolver la mayoría de estos problemas.
\end{itemize}
\end{frame}

\begin{frame}[t]{EPIC}
\begin{itemize}
  \item \textgood{Emisión}: Principalmente estática.
  \item \textgood{Detección de riesgos}: Principalmente software.
  \item \textgood{Planificación}: Sobre todo estática.
  \item \textgood{Propiedad discriminante}:
    \begin{itemize}
      \item Todos los riesgos determinados e indicados por el compilador.
    \end{itemize}
  
  \mode<presentation>{\vfill}
  \item \textmark{Ejemplos}: 
    \begin{itemize}
      \item Itanium.
    \end{itemize}
\end{itemize}
\end{frame}

