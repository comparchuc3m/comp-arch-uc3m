\subsection{Destructor}

\begin{frame}[t,fragile]{Destrucción de objetos}
\begin{columns}[T]

\column{.4\textwidth}
\begin{itemize}
  \item Un \textmark{destructor} es una función miembro especial que se
        ejecuta \textgood{de forma automática} cuando un objeto sale de
        alcance.
    \begin{itemize}
      \item No tiene tipo de retorno.
      \item No toma parámetros.
      \item Nombre de clase precedido de carácter \textbf{\cppid{\~}}.
    \end{itemize}
\end{itemize}


\column{.6\textwidth}
\begin{block}{Definición}
\begin{lstlisting}
class numvector {
public:
  //...

  numvector(int n) : size{n}, vec{new double[size]}
    {}

  ~numvector() { delete []vec; }

private:
  int size;
  double * vec;
};
\end{lstlisting}
\end{block}

\end{columns}
\end{frame}

\begin{frame}[t,fragile]{Invocación de destructor}
\begin{itemize}
  \item El destructor se invoca de forma automática.
\end{itemize}

\begin{block}{Invocación automática}
\begin{lstlisting}
void f() {
  numvector v(100);
  for (int i=0; i<100; ++i) {
    v[i] = i;
  }
  // ...
  for (int i=0; i<100; ++i) {
    std::println("{}", v[i]);
  }
} // Invocación de destructor
\end{lstlisting}
\end{block}
\end{frame}
