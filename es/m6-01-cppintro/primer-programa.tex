\subsection{Un primer programa}

\begin{frame}{Hola}
\begin{columns}[T]

\column{ .4\textwidth}
\begin{block}{hola.cpp}
\lstinputlisting{int/m6-01-cppintro/hello/hello.cpp}
\end{block}

\column{.6\textwidth}
  \begin{itemize}
    \item Archivo de cabecera: \mode<presentation>{\cppid{iostream}.}
      \mode<article>{
        \begin{itemize}
          \item Se sigue el paradigma de sustitución de texto.
        \end{itemize}
      }
      \mode<article>{
        \begin{itemize}
          \item Alcance que contiene elementos de una biblioteca para evitar conflictos de nombrado.
          \item Es distinto de los modelos de \emph{paquetes} de otros lenguajes.
        \end{itemize}
      }
    \item Importación de espacio de nombres: \mode<presentation>{\cppid{std}.}
    \item Programa principal: \mode<presentation>{\cppid{main}.}
      \begin{itemize}
        \item Es el punto de entrada al programa.
      \end{itemize}
      \mode<article>{
        \begin{itemize}
          \item Hay cosas que ocurren antes.
          \item Hay cosas que ocurren después.
        \end{itemize}
      }
    \item Flujo de salida estándar: \mode<presentation>{\cppid{cout}.}
        \begin{itemize}
          \item Es una variable global.
        \end{itemize}
    \item Operador de salida: \mode<presentation>{\cppkey{<{}<}.}
        \begin{itemize}
          \item Envia datos a la salida estándar.
          \item Definido para la mayoría de los tipos.
        \end{itemize}
    \item Salto de línea: \cppid{endl} como \cppstr{"\textbackslash n"}.
    \item Código de salida: \cppid{0} (devuelto a SO).
  \end{itemize}

\end{columns}
\end{frame}
