\subsection{Funciones y paso de parámetros}

\begin{frame}[t,fragile]{Funciones}
\begin{itemize}
  \item \textgood{Declaración}: Incluye parámetros y tipo de retorno.
    \begin{itemize}
      \item Dos sintaxis alternativas.
    \end{itemize}
\begin{lstlisting}
double area(double ancho, double alto);
auto area(double ancho, double alto) -> double;
\end{lstlisting}

  \vfill
  \item \textgood{Definición}: Permite deducción automática de tipo de retorno
\begin{lstlisting}
auto area(double ancho, double alto) {
  return ancho * alto;
}
\end{lstlisting}
\end{itemize}
\end{frame}

\begin{frame}[t,fragile]{Paso por valor}
\begin{itemize}
  \item Único paso de parámetros válido en C.
  \item Se pasa a la función una copia del argumento especificado en la llamada.
\begin{lstlisting}
int incrementa(int n) {
  ++n;
  return n;
}

void f() {
  int x = 5;
  int a = incrementa(x);
  int b = incrementa(x);
  int c = incrementa(42);
}
\end{lstlisting}
\end{itemize}
\end{frame}

\begin{frame}[t,fragile]{Paso por referencia constante}
\begin{itemize}
\item Pasa la dirección del objeto pero impide su 
      alteración dentro de la función.
  \begin{itemize}
    \item Conceptualmente equivale a paso por valor.
    \item Físicamente equivalente a paso de un puntero.
  \end{itemize}
\end{itemize}

\begin{columns}[T]

\column{.6\textwidth}
\begin{lstlisting}
double maxref(std::vector<double> const & v) {
  double res = std::numeric_limits<double>::min();
  for (auto x : v) {
    if (x>res) {
      res = x;
    }
  }
  return res;
}
\end{lstlisting}

\column{.4\textwidth}
\begin{lstlisting}
void f() {
  std::vector<double> vec(1000000);
  //...
  std::println("Max : {}", maxref(vec));
}
\end{lstlisting}

\end{columns}
\end{frame}

\begin{frame}[t,fragile]{Paso por referencia}
\begin{itemize}
  \item Elimina la restricción de no modificar el parámetro dentro de la función.
\begin{lstlisting}
void rellena(std::vector <int> & v, int n) {
  for (int i=0; i<n; ++i) {
    v.push_back(i);
  } 
}

void f() {
  using namespace std;
  vector<int> v;   // v.size() == 0
  rellena(v, 100); // v.size() == 100
}
\end{lstlisting}

  \vfill
  \item No se pasa una copia.
    \begin{itemize}
      \item Se tiene acceso al propio objeto.
    \end{itemize}
\end{itemize}
\end{frame}
