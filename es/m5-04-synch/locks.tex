\section{Cerrojos}

\begin{frame}[t]{Cerrojo}
\begin{itemize}
  \item Un \textgood{cerrojo} es un mecanismo que asegura la \textmark{exclusión mutua}.

  \mode<presentation>{\vfill\pause}
  \item \textgood{Dos funciones de sincronización}:
    \begin{itemize}

      \mode<presentation>{\vfill\pause}
      \item \textmark{Lock(k)}:
        \begin{itemize}
          \item Adquiere el cerrojo.
          \item Si varios intentan adquirir el cerrojo, n-1 pasan a espera.
          \item Si llegan más procesos, pasan a espera.
        \end{itemize}

      \mode<presentation>{\vfill\pause}
      \item \textmark{Unlock(k)}:
        \begin{itemize}
          \item Libera el cerrojo.
          \item Permite que uno de los procesos en espera adquiera el cerrojo.
        \end{itemize}
    \end{itemize}
\end{itemize}
\end{frame}

\begin{frame}[t]{Mecanismos de espera}
\begin{itemize}
  \item \textgood{Dos alternativas}: \textmark{espera activa} y \textmark{bloqueo}.

  \mode<presentation>{\vfill\pause}
  \item \textmark{Espera activa}:
    \begin{itemize}
      \item El proceso queda en \textbad{bucle} que \textmark{constantemente} 
            consulta valor de variable de control de espera.
      \item \textmark{Spin-lock}.
    \end{itemize}

  \mode<presentation>{\vfill\pause}
  \item \textmark{Bloqueo}:
    \begin{itemize}
      \item El proceso queda suspendido y cede el procesador a otro proceso.
      \item Si un proceso ejecuta \textmark{unlock} y hay procesos 
            \textbad{bloqueados} se libera a uno de ellos.
      \item Requiere soporte de un planificador (típicamente SO o \emph{runtime}).
    \end{itemize}

  \mode<presentation>{\vfill\pause}
  \item \textbad{La selección de la alternativa depende del coste}.
\end{itemize}
\end{frame}

\begin{frame}[t]{Componentes}
\begin{itemize}
  \item Tres \textgood{elementos de diseño} en un mecanismo de cerrojos:
        \textmark{adquisición}, \textmark{espera} y \textmark{liberación}.

  \mode<presentation>{\vfill\pause}
  \item \textmark{Método de adquisición}:
    \begin{itemize}
      \item Usado para intentar adquirir el cerrojo.
    \end{itemize}

  \mode<presentation>{\vfill\pause}
  \item \textmark{Método de espera}:
    \begin{itemize}
      \item Mecanismo para esperar hasta que se pueda adquirir el cerrojo.
    \end{itemize}

  \mode<presentation>{\vfill\pause}
  \item \textmark{Método de liberación}:
    \begin{itemize}
      \item Mecanismo para liberar un o varios procesos en espera.
    \end{itemize}
\end{itemize}
\end{frame}

\begin{frame}[t]{Cerrojos simples}
\begin{itemize}
  \item Variable \textmark{compartida} \cppid{k} con dos valores.
    \begin{itemize}
      \item \textgood{0} $\rightarrow$ \textmark{abierto}.
      \item \textgood{1} $\rightarrow$ \textmark{cerrado}.
    \end{itemize}

  \mode<presentation>{\vfill\pause}
  \item \textmark{Lock(k)}
    \begin{itemize}
      \item Si \cppid{k=1} $\rightarrow$ \textmark{Espera activa} mientras \cppid{k=1}.
      \item Si \cppid{k=0} $\rightarrow$ \cppid{k=1}.
      \item \textbad{No se debe permitir} que 2 procesos \textmark{adquieran cerrojo simultáneamente}.
        \begin{itemize}
          \item Usar lectura-modificación-escritura para cerrar.
        \end{itemize}
    \end{itemize}
\end{itemize}
\end{frame}

\begin{frame}[t,fragile]{Implementaciones simples}
\begin{columns}[T]

\column{.5\textwidth}
\begin{block}{Test and set}
\begin{lstlisting}
void lock(atomic_flag & k) {
  while (k.test_and_set())
    {}
}

void unlock(atomic_flag & k) {
  k.clear();
}
\end{lstlisting}
\end{block}

\pause
\column{.5\textwidth}
\begin{block}{Captación y operación}
\begin{lstlisting}
void lock(atomic<int> & k) {
  while (k.fetch_or(1) == 1)
    {}
}

void unlock(atomic<int> & k) {
  k.store(0);
}
\end{lstlisting}
\end{block}

\end{columns}


\end{frame}


\begin{frame}[t,fragile]{Implementaciones simples}

\begin{columns}[T]

\column{.5\textwidth}

\begin{block}{Intercambio IA-32}
\begin{lstlisting}[language={[x86masm]Assembler}]
do_lock:   mov eax, 1
repeat:    xchg eax, _k
           cmp eax, 1
           jz repeat
\end{lstlisting}
\end{block}

\end{columns}

\end{frame}

\begin{frame}[t,fragile]{Retardo exponencial}
\begin{itemize}
  \item \textgood{Objetivo}:
    \begin{itemize}
      \item \textmark{Reducir} accesos a memoria.
      \item \textmark{Limitar} consumo de energía.
    \end{itemize}
\end{itemize}

\mode<presentation>{\vfill\pause}

\begin{columns}

\column{.6\textwidth}
\begin{block}{Cerrojo con retardo exponencial}
\begin{lstlisting}
void lock(atomic_flag & k) {
  while (k.test_and_set())
  {
    perform_pause(delay);
    delay *= 2;
  }
}
\end{lstlisting}
\end{block}

\column{.4\textwidth}

\begin{itemize}
  \item Se incrementa \textmark{exponencialmente} tiempo entre invocaciones a
        \cppid{text\_and\_set()}.
\end{itemize}

\end{columns}

\end{frame}


\begin{frame}[t,fragile]{Sincronización y modificación}
\begin{itemize}
  \item Se pueden \textmark{mejorar prestaciones} si se usa 
        la \textgood{misma variable para sincronizar y comunicar}.
    \begin{itemize}
      \item Se evita usar \textbad{variables compartidas} solamente para sincronizar.
    \end{itemize}
\end{itemize}

\mode<presentation>{\vfill\pause}

\begin{columns}

\column{.5\textwidth}
\begin{block}{Suma de un vector}
\begin{lstlisting}
double parcial = 0;
for (int i=iproc; i<max; i+=nproc) {
  parcial += v[i];
}
suma.fetch_add(parcial);
\end{lstlisting}
\end{block}

\end{columns}
\end{frame}

\begin{frame}[t]{Cerrojos y orden de llegada}
\begin{itemize}
  \item \textbad{Problema}:
    \begin{itemize}
      \item Implementaciones simples no fijan orden de adquisición de cerrojo.
      \item Se podría dar inanición.
    \end{itemize}

  \mode<presentation>{\vfill\pause}
  \item \textgood{Solución}:
    \begin{itemize}
      \item Hacer que el cerrojo se adquiera por \textmark{antigüedad} en la solicitud.
      \item Garantizar orden FIFO.
    \end{itemize}

\end{itemize}
\end{frame}

\begin{frame}[t]{Cerrojos con etiquetas}
\begin{itemize}
  \item \textgood{Dos contadores}:
    \begin{itemize}
      \item \textmark{Contador de adquisición}: Número de procesos que han solicitado el cerrojo.
      \item \textmark{Contador de liberación}: Número de veces que se ha liberado el cerrojo.
    \end{itemize}

  \mode<presentation>{\vfill\pause}
  \textmark{Lock}:
    \begin{itemize}
      \item \textgood{Etiqueta} $\rightarrow$ Valor de contador de adquisición
      \item Se incrementa contador de adquisición.
      \item Proceso queda esperando hasta que contador de liberación coincida con etiqueta.
    \end{itemize}

  \mode<presentation>{\vfill\pause}
  \textmark{Unlock}:
    \begin{itemize}
      \item Incrementa el contador de liberación.
    \end{itemize}
\end{itemize}
\end{frame}


\begin{frame}[t]{Cerrojos basados en colas}
\begin{itemize}
  \item Mantener una \textgood{cola} con \textmark{procesos que esperan} 
        para entrar en \textbad{sección crítica}.

  \mode<presentation>{\vfill\pause}
  \item \textmark{Lock}:
    \begin{itemize}
      \item Comprobar si cola está vacía.
      \item Sin un proceso se une a la cola hace espera activa en una variable.
      \begin{itemize}
        \item Cada proceso hace espera activa en una variable distinta.
      \end{itemize}
    \end{itemize}

  \mode<presentation>{\vfill\pause}
  \item \textmark{Unlock}:
    \begin{itemize}
      \item Eliminar proceso de cola.
      \item Modificar variable de espera de proceso.
    \end{itemize}

\end{itemize}
\end{frame}
