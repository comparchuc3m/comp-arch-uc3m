\section{Conclusión}

\begin{frame}[t]{Resumen}
\begin{itemize}[<+->]
  \item Necesidad de sincronización de accesos en memoria compartida:
    \begin{itemize}
      \item Comunicación individual (1-1) y colectiva (1-N).
    \end{itemize}
  \item Diversidad de primitivas hardware para sincronización.
  \item Cerrojos como mecanismo de exclusión mutua.
    \begin{itemize}
      \item Espera activa frente a bloqueo.
      \item Tres elementos de diseño: adquisición, espera y liberación.
    \end{itemize}
  \item Los cerrojos pueden tener problemas al no fijar orden (inanición).
    \begin{itemize}
      \item Soluciones basadas en etiquetas o colas.
    \end{itemize}
  \item Las barreras ofrecen mecanismos para estructurar programas en fases.
\end{itemize}
\end{frame}


\begin{frame}[t]{Referencias}
\begin{itemize}
  \item \bibhennessy
    \begin{itemize}
      \item Section 5.5 --
    \end{itemize}

\end{itemize}
\end{frame}
