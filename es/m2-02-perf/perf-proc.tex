\section{Rendimiento del procesador}

\begin{frame}[t]{Tiempo de ejecución}
\begin{itemize}
  \item Un procesador ejecuta cada instrucción en varios ciclos de reloj.
\end{itemize}
\begin{block}{Tiempo consumido por CPU}
\begin{displaymath}
tiempo_{CPU} = 
\frac{ciclos_{CPU}}{\text{frecuencia de reloj}}
\end{displaymath}
\end{block}
\end{frame}

\begin{frame}[t]{CPI: Ciclos por instrucción}
  \begin{itemize}
    \item Se puede expresar la velocidad media como ciclos por instrucción (CPI) a partir de:
      \begin{itemize}
         \item El número total de ciclos consumidos y,
         \item el número de instrucciones ejecutadas (IC).
      \end{itemize}
  \end{itemize}
\begin{block}{CPI}
\begin{displaymath}
CPI =
\frac{ciclos_{CPU}}{IC}
\end{displaymath}
\end{block}
\end{frame}

\begin{frame}[t]{Factores en tiempo de ejecución}
\begin{block}{CPI y tiempo de CPU}
\begin{displaymath}
CPI=\frac{ciclos_{CPU}}{IC}
\end{displaymath}
\pause
\begin{displaymath}
tiempo_{CPU}=
\frac{ciclos_{CPU}}{f} \pause =
\frac{CPI \times IC}{f} \pause =
CPI \times IC \times T
\end{displaymath}
\end{block}
\mode<article>{
\begin{itemize}
  \item Aclaraciones de símbolos:
    \begin{itemize}
      \item \textbf{f}: Frecuencia de reloj.
      \item \textbf{T}: Periodo de reloj.
    \end{itemize}
\end{itemize}
}
\mode<presentation>{\pause\vfill}
\begin{itemize}
  \item Si se reducen un 10\% cualquiera de los 3 factores 
        se reduce un 10\% el tiempo de ejecución.
    \begin{itemize}
      \item \alert{Pero los 3 factores están interrelacionados}.
    \end{itemize}
\end{itemize}
\end{frame}

\begin{frame}[t]{Clases de instrucciones}
\begin{itemize}
  \item Distintas clases de instrucciones tienen distinto IC y CPI.
\end{itemize}
\begin{columns}
\begin{column}{.7\textwidth}
\begin{block}{CPI global}
\mode<presentation>{\begin{small}}
\begin{displaymath}
ciclos_{CPU}=\sum_{i=1}^n IC_i \times CPI_i
\end{displaymath}
\pause
\begin{displaymath}
tiempo_{CPU} =
\big( \sum_{i=1}^n IC_i \times CPI_i \big) \times T
\end{displaymath}
\pause
\begin{displaymath}
CPI_{global} = 
\frac
{\sum_{i=1}^n IC_i \times CPI_i}
{IC}
\pause =
\sum_{i=1}^n \frac{IC_i}{IC} \times CPI_i
\end{displaymath}
\mode<presentation>{\end{small}}
\end{block}
\end{column}
\begin{column}{.3\textwidth}
\pause
\alert{Impacto de la frecuencia relativa de instrucciones en ejecución de programa.}
\end{column}
\end{columns}
\end{frame}

\begin{frame}[t]{Ejemplo}
\begin{itemize}
  \item En la ejecución de un programa se ha visto que:
    \begin{itemize}
      \item \textmark{Operaciones coma flotante}: 25\% (4.0 CPI en promedio).
      \item \textmark{Operación FPSQR} (\emph{raíz cuadrada}): 2\% (20 CPI).
        \begin{itemize}
          \item \alert{Incluida en coma flotante.}
        \end{itemize}
      \item \textmark{Resto de instrucciones}: 1.33 CPI.
    \end{itemize}
\mode<presentation>{\pause\vfill}
  \item Elegir entre alternativas de diseño:
    \begin{enumerate}[a]
      \item Reducir CPI de FPSQR a 2.
      \item Reducir CPI de todas las operaciones de coma flotante a 2.5.
    \end{enumerate}
\end{itemize}
\end{frame}

\begin{frame}[t]{Solución}
\begin{math}
CPI = 0.25 \times 4 + 1.33 \times 0.75 = \textbf{1.9975}
\end{math}
\mode<presentation>{\pause\vfill}
\begin{math}
0.25 \times CPI_{FP} = 0.23 \times CPI_{otrasFP} + 0.02 \times CPI_{FPSQR}
\end{math}
\pause
\begin{math}
0.25 \times 4 = 0.23 \times CPI_{otrasFP} + 0.02 \times 20
\end{math}

\mode<presentation>{\pause\vfill}
\framebox{
\begin{math}
CPI_{otrasFP} = \frac{0.25 \times 4 - 0.02 \times 20}{0.23} = 2.6087
\end{math}
}
\mode<presentation>{\pause\vfill}
\begin{math}
CPI_{nuevoFPSQR} = 0.23 \times 2.6087 + 0.02 \times \alert{2} + 0.75 \times 1.33 = \textbf{1.6375}
\end{math}
\pause
\begin{math}
CPI_{nuevoFP} = 0.25 \times \alert{2.5} + 0.75 \times 1.33 = \textbf{1.6225}
\end{math}
\end{frame}
