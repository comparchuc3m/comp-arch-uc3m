\section{Métricas de rendimiento}

\begin{frame}[t]{Velocidad de ejecución}
\begin{itemize}
  \item ¿Qué significa que el computador \textmark{A} es más rápido
        que el computador \textmark{B}?
    \mode<presentation>{\pause\vfill}
    \begin{itemize}
      \item Desktop.
        \begin{itemize}
          \item Mi programa se ejecuta en menos tiempo.
          \item Quiero reducir el tiempo de ejecución.
        \end{itemize}
    \mode<presentation>{\pause\vfill}
      \item Administrador de sitio Web.
        \begin{itemize}
          \item Puedo procesar más transacciones por hora.
          \item Quiero aumentar la tasa de procesamiento.
        \end{itemize}
    \end{itemize}
\end{itemize}
\end{frame}

\begin{frame}[t]{Rendimiento y tiempo de ejecución}
\begin{itemize}
  \item El rendimiento $R(x)$ es una métrica inversa al tiempo de 
        ejecución $T(x)$.
\end{itemize}
\begin{columns}
\begin{column}{.4\textwidth}
\begin{block}{Rendimiento}
\begin{math}
R(x) = \frac{1}{T(x)}
\end{math}
\end{block}
\end{column}
\begin{column}{.6\textwidth}
\begin{itemize}
  \item Alto rendimiento $\rightarrow$ Bajo tiempo de ejecución.
\end{itemize}
\end{column}
\end{columns}
\mode<presentation>{\pause\vfill}
\begin{itemize}
  \item X se ejecuta n veces más rápido que Y.
\end{itemize}
\begin{block}{Aceleración}
\begin{math}
n=\frac{T(x)}{T(y)}=
\frac{
\frac{1}{R(x)}
}{
\frac{1}{R(y)}
}
=
\frac{R(y)}{R(x)}
\end{math}
\end{block}
\end{frame}

\begin{frame}[t]{Métricas}
\begin{itemize}
  \item La \textbad{única} métrica \textemph{fiable} para comparar el rendimiento de computadores 
        es la ejecución de \textmark{programas reales}.
    \begin{itemize}
      \item Cualquier otra métrica conduce a errores.
      \item Cualquier alternativa a programas reales conduce a errores.
    \end{itemize}
  \mode<presentation>{\pause\vfill}
  \item \textgood{Tiempo de ejecución}.
    \begin{itemize}
      \item \textmark{Tiempo de respuesta}: 
            Tiempo total transcurrido.
        \begin{itemize}
          \item Tiempo de reloj (\emph{wall-clock time}) o tiempo transcurrido (\emph{elapsed time}).
        \end{itemize}
      \mode<presentation>{\vfill}
      \item \textmark{Tiempo de CPU}: 
            Tiempo que la CPU ha estado ocupada.
        \begin{itemize}
          \item Con \textgood{un solo procesador} usado \textemph{suele ser menor} 
                que el \textmark{tiempo de respuesta}.
          \item Con \textgood{varios procesadores} usados \textemph{puede ser mayor}
                que el \textmark{tiempo de respuesta}.
        \end{itemize}
    \end{itemize}
\end{itemize}
\end{frame}

