\section{Conclusión}

\begin{frame}[t]{Resumen}
\begin{itemize}
  \item El único enfoque \textemph{fiable} para \textgood{comparar el rendimiento}
        es la \textmark{ejecución de programas reales}.

  \mode<presentation>{\vfill\pause}
  \item La \textgood{Ley de Amdahl} establece un \textemph{límite superior} 
        sobre la \textmark{mejora del rendimiento}
        con diversas aplicaciones.

  \mode<presentation>{\vfill\pause}
  \item La \textmark{frecuencia relativa} de las instrucciones tiene un \textemph{fuerte impacto} 
        sobre la \textgood{velocidad} de ejecución de los programas.
\end{itemize}
\end{frame}

\begin{frame}[t]{Referencias}
\begin{itemize}
  \item \credithennessy
    \begin{itemize}
      \item Sección 1.8 -- Measuring, Reporting, and Summarizing Performance.
      \item Sección 1.9 -- Quantitative Principles of Computer Design.
    \end{itemize}
\end{itemize}
\end{frame}
