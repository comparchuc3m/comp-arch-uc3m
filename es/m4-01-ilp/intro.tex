\section{Introducción a la segmentación}

\begin{frame}[t]{Segmentación}
\begin{itemize}
  \item Técnica de implementación en la que una la ejecución de múltiples 
        instrucciones se solapan en el tiempo.
    \begin{itemize}
      \mode<presentation>{\vfill}
      \item Se divide una operación \textmark{costosa} en varias 
            sub-operaciones simples.
      \mode<presentation>{\vfill}
      \item Ejecución de las sub-operaciones por etapas.
    \end{itemize}
  \mode<presentation>{\pause\vfill}
  \item \textgood{Efectos}:
    \begin{itemize}
      \mode<presentation>{\vfill}
      \item \textgood{Aumenta} el \textmark{\emph{throughput}}.
      \mode<presentation>{\vfill}
      \item \alert{No disminuye} la \textmark{latencia}.
    \end{itemize}
\end{itemize}
\end{frame}

\begin{frame}[t,fragile]{Segmentación}
\begin{columns}[T]
\begin{column}{0.6\textwidth}
\input{es/m4-01-ilp/intro-pipeline.tkz}
\end{column}

\pause
\begin{column}[t]{.4\textwidth}
\begin{itemize}
  \item \textgood{Latencia}:
    \begin{itemize}
      \item Una instrucción requiere 5 etapas.
      \item 5 ciclos.
    \end{itemize}
  \item \textgood{Throughput}:
    \begin{itemize}
      \item Termina una instrucción en cada ciclo (con pipeline lleno).
      \item 1 instrucción por ciclo.
    \end{itemize}
\end{itemize}
\end{column}
\end{columns}
\end{frame}

\begin{frame}[t]{Etapas del pipeline}
\begin{itemize}
  \item Modelo simplificado:
    \begin{itemize}
      \item 5 etapas.
      \item \alert{Modelos más realistas requieren muchas más etapas}.
    \end{itemize}
  \mode<presentation>{\pause\vfill}
  \item \textgood{Etapas}:
    \begin{itemize}
      \item \textmark{Captación} (\emph{Instruction Fetch}).
      \item \textmark{Decodificación} (\emph{Instruction Decode}).
      \item \textmark{Ejecución} (\emph{Execution}).
      \item \textmark{Memoria} (\emph{Memory}).
      \item \textmark{Post-escritura} (\emph{Write-back}).
    \end{itemize}
\end{itemize}
\end{frame}

\begin{frame}[t]{Captación}
\begin{columns}
\begin{column}{.5\textwidth}
\input{int/m4-01-ilp/fetch-design.tkz}
\end{column}
\begin{column}{.5\textwidth}
\begin{itemize}
  \item Envío de PC a memoria.
  \item Lectura de instrucción.
  \item Actualización de PC.
\end{itemize}
\end{column}
\end{columns}
\end{frame}

\begin{frame}[t]{Decodificación}
\begin{columns}
\begin{column}{.5\textwidth}
\input{int/m4-01-ilp/decode-design.tkz}
\end{column}
\begin{column}{.5\textwidth}
\begin{itemize}
  \item Decodificación de instrucción.
  \item Lectura de registros.
  \item Extensión de signo de desplazamientos.
  \item Cálculo de posible dirección de salto.
\end{itemize}
\end{column}
\end{columns}
\end{frame}

\begin{frame}[t]{Ejecución}
\begin{columns}
\begin{column}{.5\textwidth}
\input{int/m4-01-ilp/exec-design.tkz}
\end{column}
\begin{column}{.5\textwidth}
\begin{itemize}
  \item Operación de ALU sobre registros.
  \item Alternativamente, cálculo de dirección efectiva.
\end{itemize}
\end{column}
\end{columns}
\end{frame}

\begin{frame}[t]{Memoria}
\begin{columns}
\begin{column}{.5\textwidth}
\begin{center}
\input{int/m4-01-ilp/mem-design.tkz}
\end{center}
\end{column}
\begin{column}{.5\textwidth}
\begin{itemize}
  \item Lectura o escritura en memoria.
\end{itemize}
\end{column}
\end{columns}
\end{frame}

\begin{frame}[t]{Post-escritura}
\begin{columns}
\begin{column}{.5\textwidth}
\input{int/m4-01-ilp/wb-design.tkz}
\end{column}
\begin{column}{.5\textwidth}
\begin{itemize}
  \item Escritura de resultado en banco de registros.
\end{itemize}
\end{column}
\end{columns}
\end{frame}

\begin{frame}[t]{Arquitectura general}
\input{int/m4-01-ilp/general-design.tkz}
\end{frame}

\begin{frame}[t]{Efectos del pipeline}
\begin{itemize}
  \item Un pipeline de \textmark{profundidad} \alert{n}, 
  multiplica por \alert{n} el \textmark{ancho de banda} 
  necesario de la versión sin pipeline con la misma frecuencia de reloj.
    \begin{itemize}
      \item Cachés, cachés, ...
    \end{itemize}
  \mode<presentation>{\vfill}
  \item La separación de \textgood{caché} de \textmark{datos} e 
        \textmark{instrucciones} elimina algunos \alert{conflictos} 
        de memoria.
  \mode<presentation>{\vfill}
  \item Las instrucciones que están en el pipeline \alert{no deberían}
        intentar usar el mismo recurso en el mismo momento.
    \begin{itemize}
      \item Introducción de registros de pipeline entre etapas sucesivas.
    \end{itemize}
\end{itemize}
\end{frame}

\begin{frame}[t,shrink=2]{Comunicación entre etapas}
\input{int/m4-01-ilp/general-reg-design.tkz}
\end{frame}

\begin{frame}[t]{El pipeline a lo largo del tiempo}
\input{es/m4-01-ilp/pipeline-over-time.tkz}
\begin{itemize}
  \item Lectura registros segunda mitad de ciclo.
  \item Escritura registros primera mitad de ciclo.
\end{itemize}
\end{frame}

\begin{frame}[t]{Ejemplo}
\begin{itemize}
  \item Procesador no segmentado.
    \begin{itemize}
      \item Ciclo de reloj: 1 ns.
      \item 40\% operaciones ALU $\rightarrow$ 4 ciclos.
      \item 20\% operaciones de bifurcación $\rightarrow$ 4 ciclos.
      \item 40\% operaciones de memoria $\rightarrow$ 5 ciclos.
      \item Sobrecoste de segmentación $\rightarrow$ 0.2 ns.
    \end{itemize}
  \mode<presentation>{\pause}
  \item ¿Qué aceleración se obtiene con la segmentación?
  \mode<presentation>{\pause}
\end{itemize}
\[
t_{orig} = ciclo_{reloj} \times CPI = 1 ns \times (0.6 \times 4 + 0.4 \times 5) = 4.4 ns
\]
\[
t_{nuevo} = 1 ns + 0.2 ns = 1.2 ns
\]
\[
S = \frac{4.4 ns}{1.2 ns} = 3.7
\]
\end{frame}
