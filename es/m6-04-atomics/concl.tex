\section{Conclusión}
\begin{frame}[t]{Resumen}
\begin{itemize}
  \item El modelo de memoria de C++ define las reglas de acceso a memoria
        de un programa correcto.
    \begin{itemize}
      \item Permite programación portable de estructuras de datos libres de cerrojos.
    \end{itemize}
  \item Los tipos atómicos permite realizar operaciones de memoria especificando un ordenamiento.
    \begin{itemize}
      \item El ordenamiento por defecto es consistencia secuencial.
    \end{itemize}
  \item Las relaciones \emph{sincroniza-con} y \emph{ocurre-antes} definen restricciones sobre los ordenamientos
        de operaciones.
  \item Las barreras permiten forzar ordenamientos sin modificar datos.
\end{itemize}
\end{frame}

\begin{frame}[t]{Referencias}
\begin{itemize}
  \item \emph{C++ Concurrency in Action. Practical multithreading.}\\
  Anthony Williams.\\
  Capítulo 5.
  
\end{itemize}
\end{frame}
