\section{Mutex objects and condition variables}

\subsection{Mutex objects}

\begin{frame}[t]{\emph{mutex} classification}
\begin{itemize}
  \item Represent exclusive access to a resource.
    \begin{itemize}
      \item \cppid{mutex}: Basic non-recursive \emph{mutex}.
      \item \cppid{recursive\_mutex}: A \emph{mutex} that can be acquired more than once from the same thread.
      \item \cppid{timed\_mutex}: Non-recursive \emph{mutex} with timed operations.
      \item \cppid{recursive\_timed\_mutex}: Recursive \emph{mutex} with timed operations.
    \end{itemize}

  \mode<presentation>{\vfill\pause}
  \item Only a thread can own a \emph{mutex} at a given time.
    \begin{itemize}
      \item Acquire a \emph{mutex} $\rightarrow$ Get exclusive access to object.
        \begin{itemize}
          \item Blocking operation.
        \end{itemize}
      \item Release a \emph{mutex} $\rightarrow$ Release exclusive access to object.
        \begin{itemize}
          \item Allows another thread to get access.
        \end{itemize}
    \end{itemize}
\end{itemize}
\end{frame}

\begin{frame}[t]{Operations}
\begin{itemize}
  \item Construction and destruction:
    \begin{itemize}
      \item Can be default constructed.
      \item Cannot be neither copied nor moved.
      \item Destructor may lead to undefined behavior if \emph{mutex} is not free.
    \end{itemize}
  \vfill
  \item Acquire and release:
    \begin{itemize}
      \item \cppid{m.lock()}: Acquires \emph{mutex} in a blocking mode.
      \item \cppid{m.unlock()}: Releases \emph{mutex}.
      \item \cppid{r = m.try\_lock()}: Tries to acquire \emph{mutex}, returning success indication.
    \end{itemize}
  \vfill
  \item Others:
    \begin{itemize}
      \item \cppid{h = m.native\_handle()}: Returns platform dependent identifier of type \cppid{native\_handle\_type}.
    \end{itemize}
\end{itemize}
\end{frame}

\begin{frame}[t,fragile]{Example}
\begin{columns}

\begin{column}{.5\textwidth}
\begin{block}{Exclusive access}
\begin{lstlisting}
mutex mutex_output;

void print(int x) {
  std::mutex_output.lock();
  std::cout << x << "\n";
  std::mutex_output.unlock();
}

void print(double x) {
  std::mutex_output.lock();
  std::cout << x << "\n";
  std::mutex_output.unlock();
}
\end{lstlisting}
\end{block}
\end{column}

\begin{column}{.5\textwidth}
\begin{block}{Threads launch}
\begin{lstlisting}
void f() {
  std::thread t1{print, 10};
  std::thread t2(print, 5.5};
  std::thread t3(print, 3);
 
  t1.join(); 
  t2.join(); 
  t3.join();
}
\end{lstlisting}
\end{block}
\end{column}

\end{columns}
\end{frame}

\begin{frame}[t,fragile]{Errors in mutual exclusion}
\begin{itemize}
  \item In case of error exception \cppid{system\_error} is thrown.
  \item Error codes:
    \begin{itemize}
      \item \cppid{resource\_deadlock\_would\_occur}.
      \item \cppid{resource\_unavailable\_try\_again}.
      \item \cppid{operation\_not\_permitted}.
      \item \cppid{device\_or\_resource\_busy}.
      \item \cppid{invalid\_argument}.
    \end{itemize}
\begin{lstlisting}
std::mutex m;
try {
  m.lock();
  // 
  m.lock();
}
catch (std::system_error & e) {
  std::cerr << e.what() << '\n' << e.code() << '\n';
}
\end{lstlisting}
\end{itemize}
\end{frame}

\begin{frame}[t]{Deadlines}
\begin{itemize}
  \item Operations supported by \cppid{timed\_mutex} and \cppid{recursive\_timed\_mutex}.
  \vfill
  \item Add acquire operations with indication of deadlines.
    \begin{itemize}
      \item \cppid{r = m.try\_lock\_for(d)}: Try to acquire \emph{mutex} for a duration
            \cppid{d}, returning success indication.
      \item \cppid{r = m.try\_lock\_until(t)}: Try to acquire \emph{mutex} until a point in time
            returning success indication.
    \end{itemize}
\end{itemize}
\end{frame}

\subsection{Condition variables}

\begin{frame}[t]{Condition variables}
\begin{itemize}
  \item Synchronizing operations among threads.
  \item Optimized for class \cppid{mutex} (alternative \cppid{condition\_variable\_any})).
  \item Construction and destruction:
    \begin{itemize}
      \item \cppid{condition\_variable c\{\}}: Creates a condition variable
        \begin{itemize}
          \item Might throw \cppid{system\_error}.
        \end{itemize}
      \item Destructor: Destroys condition variable.
        \begin{itemize}
         \item Requires no thread is waiting on condition.
        \end{itemize}
      \item Cannot be neither copied nor moved.
      \item Before destruction all threads blocked in variable need to be notified.
        \begin{itemize}
          \item Or they could be blocked forever.
        \end{itemize}
    \end{itemize}
\end{itemize}
\end{frame}

\begin{frame}[t]{Notification/waiting operations}
\begin{itemize}
  \item Notification:
    \begin{itemize}
      \item \cppid{c.notify\_one()}: Wakes up one of waiting threads.
      \item \cppid{c.notify\_all()}: Wakes up all waiting threads.

    \end{itemize}
  \item Unconditional waiting (\cppid{l} of type \cppid{unique\_lock<mutex>}):
    \begin{itemize}
      \item \cppid{c.wait(l)}: Blocks until it gets to acquire lock \cppid{l}.
      \item \cppid{c.wait\_until(l,t)}: Blocks until it gets to acquire lock \cppid{l} or time \cppid{t} is reached.
      \item \cppid{c.wait\_for(l,t)}: Blocks until it gets to acquire lock \cppid{l} or duration \cppid{d} elapses.
    \end{itemize}
  \item Waiting with predicates.
    \begin{itemize}
      \item Takes as additional arguments a predicate that must be satisfied.
    \end{itemize}
\end{itemize}
\end{frame}

\begin{frame}[t,fragile]{Revisiting producer/consumer}
\begin{block}{Predicate injection in wait}
\begin{lstlisting}
void consumer() {
  for (;;) {
    std::unique_lock l{m};

    cv.wait(l, [this]{return !q.empty();});

    auto r = q.front();
    q.pop();
    l.unlock();
   
    process(r);
  };
}
\end{lstlisting}
\end{block}
\end{frame}

