\section{Locks}

\begin{frame}[t]{Locks}
\begin{itemize}
  \item A \textgood{lock} is a mechanism to ensure \textmark{mutual exclusion}.

  \mode<presentation>{\vfill\pause}
  \item \textgood{Two synchronization functions}:
    \begin{itemize}

      \mode<presentation>{\vfill\pause}
      \item \textmark{Lock(k)}:
        \begin{itemize}
          \item Acquires the lock.
          \item If several processes try to acquire the lock, n-1 are kept waiting.
          \item If more processes arrive, they are kept to waiting.
        \end{itemize}

      \mode<presentation>{\vfill\pause}
      \item \textmark{Unlock(k)}:
        \begin{itemize}
          \item Releases the lock.
          \item Allow that a waiting process acquires the lock.
        \end{itemize}
    \end{itemize}
\end{itemize}
\end{frame}

\begin{frame}[t]{Waiting mechanisms}
\begin{itemize}
  \item \textgood{Two alternatives}: \textmark{busy waiting} and \textmark{blocking}.

  \mode<presentation>{\vfill\pause}
  \item \textmark{Busy waiting}:
    \begin{itemize}
      \item Process waits in a \textbad{loop} that \textmark{constantly} 
            queries the wait control variable value.
      \item \textmark{Spin-lock}.
    \end{itemize}

  \mode<presentation>{\vfill\pause}
  \item \textmark{Blocking}:
    \begin{itemize}
      \item Process remains suspended and yields processor to other process.
      \item If a process executes \textmark{unlock} and there are 
            \textbad{blocked} processes, one of them is un-blocked.
      \item Requires support from a scheduler (usually OS or \emph{runtime}).
    \end{itemize}

  \mode<presentation>{\vfill\pause}
  \item \textbad{Alternative selection depends on cost}.
\end{itemize}
\end{frame}

\begin{frame}[t]{Components}
\begin{itemize}
  \item Three \textgood{elements of design} in a locking mechanism:
        \textmark{acquisition}, \textmark{waiting} y \textmark{release}.

  \mode<presentation>{\vfill\pause}
  \item \textmark{Acquisition method}:
    \begin{itemize}
      \item Used to try to acquire the lock.
    \end{itemize}

  \mode<presentation>{\vfill\pause}
  \item \textmark{Waiting method}:
    \begin{itemize}
      \item Mechanism to wait until lock can be acquired.
    \end{itemize}

  \mode<presentation>{\vfill\pause}
  \item \textmark{Release method}:
    \begin{itemize}
      \item Mechanism to release one or several waiting processes.
    \end{itemize}
\end{itemize}
\end{frame}

\begin{frame}[t]{Simple locks}
\begin{itemize}
  \item \textmark{Shared} variable \cppid{k} with two values.
    \begin{itemize}
      \item \textgood{0} $\rightarrow$ \textmark{open}.
      \item \textgood{1} $\rightarrow$ \textmark{closed}.
    \end{itemize}

  \mode<presentation>{\vfill\pause}
  \item \textmark{Lock(k)}
    \begin{itemize}
      \item If \cppid{k=1} $\rightarrow$ \textmark{Busy waiting} while \cppid{k=1}.
      \item If \cppid{k=0} $\rightarrow$ \cppid{k=1}.
      \item \textbad{Do not allow} that 2 processes \textmark{acquire a lock simultaneously}.
        \begin{itemize}
          \item Use read-modify-write to close it.
        \end{itemize}
    \end{itemize}
\end{itemize}
\end{frame}

\begin{frame}[t,fragile]{Simple implementations}
\begin{columns}[T]

\column{.5\textwidth}
\begin{block}{Test and set}
\begin{lstlisting}
void lock(atomic_flag & k) {
  while (k.test_and_set())
    {}
}

void unlock(atomic_flag & k) {
  k.clear();
}
\end{lstlisting}
\end{block}

\pause
\column{.5\textwidth}
\begin{block}{Fetch and operate}
\begin{lstlisting}
void lock(atomic<int> & k) {
  while (k.fetch_or(1) == 1)
    {}
}

void unlock(atomic<int> & k) {
  k.store(0);
}
\end{lstlisting}
\end{block}

\end{columns}


\end{frame}


\begin{frame}[t,fragile]{Simple implementations}

\begin{columns}[T]

\column{.5\textwidth}

\begin{block}{Exchange IA-32}
\begin{lstlisting}[language={[x86masm]Assembler}]
do_lock:   mov eax, 1
repeat:    xchg eax, _k
           cmp eax, 1
           jz repeat
\end{lstlisting}
\end{block}

\end{columns}

\end{frame}

\begin{frame}[t,fragile]{Exponential delay}
\begin{itemize}
  \item \textgood{Goal}:
    \begin{itemize}
      \item \textmark{Reduce} number of memory accesses.
      \item \textmark{Limit} energy consumption.
    \end{itemize}
\end{itemize}

\mode<presentation>{\vfill\pause}

\begin{columns}

\column{.6\textwidth}
\begin{block}{Lock with exponential delay}
\begin{lstlisting}
void lock(atomic_flag & k) {
  while (k.test_and_set())
  {
    perform_pause(delay);
    delay *= 2;
  }
}
\end{lstlisting}
\end{block}

\column{.4\textwidth}

\begin{itemize}
  \item Time between invocations to 
        \cppid{test\_and\_set()}
        is incremented \textmark{exponentially} 
\end{itemize}

\end{columns}

\end{frame}


\begin{frame}[t,fragile]{Synchronization and modification}
\begin{itemize}
  \item \textmark{Performance can be improved} if using the
        \textgood{same variable to synchronize and communicate}.
    \begin{itemize}
      \item Avoid using \textbad{shared variables} only to synchronize.
    \end{itemize}
\end{itemize}

\mode<presentation>{\vfill\pause}

\begin{columns}

\column{.5\textwidth}
\begin{block}{Add a vector}
\begin{lstlisting}
double partial = 0;
for (int i=iproc; i<max; i+=nproc) {
  partial += v[i];
}
sum.fetch_add(partial);
\end{lstlisting}
\end{block}

\end{columns}
\end{frame}

\begin{frame}[t]{Locks and arrival order}
\begin{itemize}
  \item \textbad{Problem}:
    \begin{itemize}
      \item Simple implementations do not fix a lock acquisition order.
      \item Starvation might happen.
    \end{itemize}

  \mode<presentation>{\vfill\pause}
  \item \textgood{Solution}:
    \begin{itemize}
      \item Make the lock is acquired by request \textmark{age}.
      \item Guarantees FIFO order.
    \end{itemize}

\end{itemize}
\end{frame}

\begin{frame}[t]{Tagged locks}
\begin{itemize}
  \item \textgood{Two counters}:
    \begin{itemize}
      \item \textmark{Acquire counter}: Number of processes that have requested the lock.
      \item \textmark{Release counter}: Number of times the lock has been released.
    \end{itemize}

  \mode<presentation>{\vfill\pause}
  \textmark{Lock}:
    \begin{itemize}
      \item \textgood{Tag} $\rightarrow$ Acquisition counter value.
      \item Acquisition counter is incremented.
      \item Process remains waiting until the release counter matches the tag.
    \end{itemize}

  \mode<presentation>{\vfill\pause}
  \textmark{Unlock}:
    \begin{itemize}
      \item Increments release counter.
    \end{itemize}
\end{itemize}
\end{frame}


\begin{frame}[t]{Queue based locks}
\begin{itemize}
  \item Keep a \textgood{queue} with \textmark{processes waiting} 
        to enter into a \textbad{critical section}.

  \mode<presentation>{\vfill\pause}
  \item \textmark{Lock}:
    \begin{itemize}
      \item Check if queue is empty.
      \item If a process joins the queue it performs busy waiting in a variable.
      \begin{itemize}
        \item Each process performs busy waiting in a different variable.
      \end{itemize}
    \end{itemize}

  \mode<presentation>{\vfill\pause}
  \item \textmark{Unlock}:
    \begin{itemize}
      \item Removes process from queue.
      \item Modifies process waiting control variable.
    \end{itemize}

\end{itemize}
\end{frame}
