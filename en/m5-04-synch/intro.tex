\section{Introduction}

\begin{frame}[t]{Synchronization in shared memory}
\begin{itemize}
  \item Communication performed through shared memory.
    \begin{itemize}
      \item It is necessary to synchronize multiple accesses to shared variables.
    \end{itemize}

  \mode<presentation>{\vfill\pause}
  \item \textgood{Alternatives}:
    \begin{itemize}
      \item Communication 1-1.
      \item Collective communication (1-N).
    \end{itemize}
\end{itemize}
\end{frame}

\begin{frame}[t]{Communication 1 to 1}
\begin{itemize}
  \item Ensure that reading (\textmark{receive}) is performed after
        writing (\textmark{send}).
  
  \mode<presentation>{\vfill\pause}
  \item In case of \textmark{reuse} (loops):
    \begin{itemize}
      \item Ensure that writing (\textmark{send}) is performed after former reading (\textmark{receive}).
    \end{itemize}

  \mode<presentation>{\vfill\pause}
  \item Need to access with \textgood{mutual exclusion}.
    \begin{itemize}
      \item Only one of the processes accesses a variable at the same time.
    \end{itemize}

  \mode<presentation>{\vfill\pause}
  \item \textbad{Critical section}:
    \begin{itemize}
      \item Sequence of instructions accessing one or more variables with mutual exclusion.
    \end{itemize}

\end{itemize}
\end{frame}

\begin{frame}[t]{Collective communication}
\begin{itemize}
  \item Needs \textgood{coordination} of \textmark{multiple accesses} to variables.
    \begin{itemize}
      \item Writes without interferences.
      \item Reads \textmark{must wait} for data to be available.
    \end{itemize}

  \mode<presentation>{\vfill\pause}
  \item Must \textmark{guarantee accesses} to variable in \textgood{mutual exclusion}.

  \mode<presentation>{\vfill\pause}
  \item Must \textmark{guarantee} that \textbad{result is not read} 
        until all processes/threads have executed their critical section.
  
\end{itemize}
\end{frame}

\begin{frame}[fragile]{Adding a vector}
\mode<presentation>{\vspace{-1.25em}}
\begin{columns}[T]

\column{.5\textwidth}
\begin{block}{Critical section in loop}
\begin{lstlisting}
void f(const std::vector<double> & v) {
  double sum = 0;
  
  auto do_sum = [&](int start, int n) {
    for (int i=start; i<n; ++i) {
      sum += v[i];
    }
  }

  thread t1{do_sum, 0, std::ssize(v)/2};
  thread t2{do_sum, std::ssize(v)/2+1, std::ssize(v)};
  t1.join();
  t2.join();
}
\end{lstlisting}
\end{block}

\mode<presentation>{\pause}

\column{.5\textwidth}
\begin{block}{Critical section out of loop}
\begin{lstlisting}
void f(const std::vector<double> & v) {
  double sum = 0;
  
  auto do_sum = [&](int start, int n) {
    double local_sum = 0;
    for (int i=start; i<n; ++i) {
      local_sum += v[i];
    }
    sum += local_sum;
  }

  thread t1{do_sum, 0, std::ssize(v)/2};
  thread t2{do_sum, std::ssize(v)/2+1, std::ssize(v)};
  t1.join();
  t2.join();
}
\end{lstlisting}
\end{block}

\end{columns}
\end{frame}
