\section{Tasks}

\subsection{Compilation and execution}

In this task you will have to compile and run the program.
You will also compare the \emph{Debug} and \emph{Release} versions.

Compile the project and generate the \emph{Debug} version.

Run the generated program 3 times on the \textmark{stan} server.

Take note for each execution of the following information:
\begin{itemize}
\item The final value of \cppid{counter}.
\item The execution time measured by the program.
\end{itemize}

Now generate the \emph{Release} version.

Run the program again on the \textmark{stan} server and take note of the new results.

\subsection{Detecting data races}

To detect data races you will use a compiler \textmark{sanitizer}: 
\textgood{Thread Sanitizer}.

You can edit the \textgood{CMakePresets.json} file to add the compilation flag
to the \textmark{default} preset:

\begin{lstlisting}
"CMAKE_CXX_FLAGS": "-Wall -Wextra -Werror -pedantic -pedantic-errors -Wconversion -Wsign-conversion -fsanitize=thread",
\end{lstlisting}

As you can observe, an additional compilation flag
\textmark{-fsanitize=thread} has been added.

\subsection{Protecting with locks}

Protect the \cppid{value\_} data member with a \cppid{std::mutex}. For this:

\begin{itemize}
  \item Add a data member to class \cppid{counter} of type \cppid{std::mutex}.
  \item Protect by using \cppid{std::mutex} the critical sections.
\end{itemize}

Place these modifications in the directory \textemph{counter-mutex}.
Check that the program is free of race conditions with \textmark{Thread Sanitizer}.

\textbad{NOTE}: All evaluations shall be performed on the \textmark{stan} server.
Evaluate the performance for the cases of 2, 4, 8, 16, 32, 64, and 128 threads.

\subsection{Protecting with an atomic}

In C++20 you may use atomic types for protecting floating point values.
We will use this to provide a lock free solution.

Change the type of data member \cppid{value\_} to \cppid{std::atomic<double>}.

Place these modifications in the directory \textemph{counter-atomic}.
Check that the program is free of race conditions with \textmark{Thread Sanitizer}.

\textbad{NOTE}: All evaluations shall be performed on the \textmark{stan} server.
Evaluate the performance for the cases of 2, 4, 8, 16, 32, 64, and 128 threads.

\subsection{Protecting with a spin lock and sequential consistency}

Now we will study the cases where an atomic variable is not enough.
Consider the case of updating two values in the critical section.

\begin{itemize}

  \item Modify class \cppid{counter} to have a second data member named \cppid{time\_} 
        of type \cppkey{float}.

  \item Modify the \cppid{update()} function so that each invocation increments
        \cppid{time\_} in \cppid{0.25} units.

\end{itemize}

To control access to the critical section, use an object of type \cppid{spinlock\_mutex} 
as the following, with sequential consistency:
\begin{lstlisting}
class spinlock_mutex {
public:
  void lock() {
    while (flag_.test_and_set()) {
      std::this_thread::yield();
    }
  }

  void unlock() {
    flag_.clear();
  }

private:
  std::atomic_flag flag_;
};
\end{lstlisting}

Place these modifications in the directory \textemph{counter-spinseq}.
Check that the program is free of race conditions with \textmark{Thread Sanitizer}.

\textbad{NOTE}: All evaluations shall be performed on the \textmark{stan} server.
Evaluate the performance for the cases of 2, 4, 8, 16, 32, 64, and 128 threads.

\subsection{Protecting with a spin lock and release/acquire consistency}

Repeat the previous section with a \cppid{spinlock\_mutex} that uses
release/acquire consistency:

\begin{lstlisting}
class spinlock_mutex {
public:
  void lock() {
    while (flag_.test_and_set(std::memory_order_acquire)) {
      while (flag_.test(std::memory_order_relaxed)) {
        std::this_thread::yield();
      }
    }
  }

  void unlock() { flag_.clear(std::memory_order_release); }

private:
  std::atomic_flag flag_;
};
\end{lstlisting}

Place these modifications in the directory \textemph{counter-spinra}. Check
that the program is free of race conditions with \textmark{Thread
Sanitizer}.

\textbad{NOTE}: All evaluations shall be performed on the \textmark{stan} server.
Evaluate the performance for the cases of 2, 4, 8, 16, 32, 64, and 128 threads.

\subsection{Optimization}

An alternative to the release/acquire case is using
\cppid{wait()}/\cppid{notify()} primitives:

\begin{lstlisting}
class spinlock_mutex {
public:
  void lock() {
    while (flag_.test_and_set(std::memory_order_acquire)) {
      flag_.wait(true, std::memory_order_relaxed);
    }
  }

  void unlock() {
    flag_.clear(std::memory_order_release);
    flag_.notify_one();
  }

private:
  std::atomic_flag flag_;
};
\end{lstlisting}

Place these modifications in the directory \textemph{counter-spinwn}.
Check that the program is free of race conditions with \textmark{Thread
Sanitizer}.

\textbad{NOTE}: All evaluations shall be performed on the \textmark{stan} server.
Evaluate the performance for the cases of 2, 4, 8, 16, 32, 64, and 128 threads.
