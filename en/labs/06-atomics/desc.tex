\section{Lab description}

This lab starts from a project provided to you that contains a main program
that runs an application with multiple threads.

Note that for compiling a multi-threaded program with CMake you must keep in mind:

\begin{itemize}

\item You must include a line to find the compilation package for threads:
\begin{lstlisting}
find_package(Threads REQUIRED)
\end{lstlisting}

\item You must link your program with the threading library.
      To achieve this, you must include the following line before the executable definition.
      (it will affect to all subsequent executables).
\begin{lstlisting}
link_libraries(Threads::Threads)
\end{lstlisting}

\end{itemize}

You are provided a program \cppid{counter.cpp} with basic functionality.

\begin{itemize}

\item A class \cppid{counter} encapsulating a double precision floating point value, 
      initialized to zero.
      It has the following operations:
\begin{itemize}
  \item A \cppid{update()} function to increment the encapsulated value.
  \item A \cppid{print()} function to print the encapsulated value.
\end{itemize}

\item A main program launching multiple threads.
\begin{itemize}
  \item The number of threads is defined in constant \cppid{num\_threads}.
  \item The shared counter is local variable \cppid{count}.
  \item Each thread performs $100,000$ updates to the counter.
  \item When all threads have finished, the final result for the counter is printed.
  \item Additionally, the elapsed time for running all threads is printed.
\end{itemize}

\end{itemize}

To carry out this lab, you must carry out all evaluations in the
\textemph{avignon} cluster.
