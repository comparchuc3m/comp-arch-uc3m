\subsection{NUMA node characteristics}

Through the \emph{front-end} \textmark{avignon} you can have access to a
compute server (called \textmark{stan}).

Prepare a \emph{script} to see the hardware characteristics of
\textmark{stan} called \cppid{lscpu.sh}. Listing~\ref{lst:lscpu} presents
a possible script.

\lstinputlisting[
language=bash,style=terminal,
caption={Script to obtain hardware characteristics},
label={lst:lscpu},
]{lab/05-tbb/lscpu.sh}

Review the main hardware characteristics and analyze the differences
with the normal compute nodes and the \textmark{stan} node.

\subsection{Building the NUMA version}

In the support material you will find \emph{scripts} with the suffix
\textmark{stan}. These are designed to be used on this compute node.

You can build on this node by doing:

\begin{lstlisting}[language=bash,style=terminal]
sbatch -p stan build-stan.sh
\end{lstlisting}

To execute, you can use a script like \textgood{run-invoke-stan.sh} (see
listing~\ref{lst:run-invoke-stan}):

\lstinputlisting[
language=bash,style=terminal,
caption={Execution script for invoke on the stan node},
label={lst:run-invoke-stan}
]{lab/05-tbb/run-invoke-stan.sh}

\textbad{IMPORTANT}: 
To run a \emph{script} on \textmark{stan} you must specify the name of
the execution queue.

\begin{lstlisting}[style=terminal]
sbatch -p stan run-invoke-stan.sh
\end{lstlisting}

