\section{Lab description}

In this lab you will become familiar with parallel programming using
the TBB library (\emph{Threading Building Blocks}).

In total you will practice with the following exercises:

\begin{itemize}

\item \textmark{invoke}:
In this exercise you will practice parallel task execution.

\item \textmark{matrixmult}:
In this exercise we will start with a classic matrix product and see
different alternatives to optimize matrix product execution.

\item \textmark{picalc}:
In this exercise we will see how to accelerate a numerical method to calculate an
integral. We will apply these techniques to calculate the number $\pi$.

\item \textmark{pimontecarlo}:
In this exercise we will see how to parallelize an algorithm that depends on
a random number generator. We will apply this technique to a Monte Carlo
simulation to estimate the value of the number $\pi$.

\item \textmark{matrixmult-part}:
In this exercise we will see the different ways to distribute work among the
threads of a parallel program.

\end{itemize}

\subsection{Building the programs}

Please remember that to compile the optimized version of the program
you must generate a \emph{release} configuration.

\begin{tcolorbox}
[colframe=red!50!black,
colback=red!10!white, 
title={\textbf{IMPORTANT}}]
To run jobs on the \textmark{avignon} cluster, you must create a
\emph{script} to launch the build.
\end{tcolorbox}

For example, the script in listing~\ref{lst:build-normal} allows you to build on a
normal node of the \textmark{avignon} cluster. First, the
build configuration is generated in the \textemph{build/normal} directory. Then
the build is performed generating the corresponding
executables using the \textemph{gcc-release} build configuration
and the \textemph{Release} configuration.

\lstinputlisting[
caption={Build script for normal node of the \textmark{avignon} cluster},
label=lst:build-normal,
language=bash,style=terminal]
{lab/05-tbb/build-normal.sh}

\subsection{Source code structure}

The supplied source code has the following structure:

\begin{itemize}

\item \textgood{01-invoke}: 
Parallel task invocation exercise.

\item \textgood{02-matrixmult}:
Matrix product exercise.

\item \textgood{03-picalc}:
Exercise for calculating the number $\pi$.

\item \textgood{04-pimontecarlo}:
Exercise for estimating the number $\pi$ with the Monte Carlo method.

\item \textgood{05-matrixmult-part}:
Exercise using task partitioning schemes.

\end{itemize}

Note that during this lab, you will need to modify the examples
to complete them before running.
