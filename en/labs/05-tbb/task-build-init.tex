\subsection{Building the initial version}

Your first task is to build the original version of the exercises. You should
build only the \emph{release} version. To do this, use a
\emph{script} like the one shown in listing~\ref{lst:build-normal}.

You can run this script on a normal cluster node by doing:

\begin{lstlisting}[style=terminal,language=bash]
sbatch build-normal.sh
\end{lstlisting}

To run any of the programs, you can use a
\emph{script} like \cppid{run-invoke.sh} (see listing~\ref{lst:run-invoke-normal}).

\lstinputlisting[
language=bash,style=terminal,
caption={Execution script for invoke on a normal node},
label={lst:run-invoke-normal}
]{lab/05-tbb/run-invoke-normal.sh}

Note that in this script two queue system variables are defined:
\cppid{job-name} and {output}. The first sets the job name in the queue
system and the second defines the name of the job output file.
Remember that the logs directory must exist beforehand.

\begin{lstlisting}[style=terminal,language=bash]
mkdir logs
sbatch run-invoke-normal.sh
\end{lstlisting}

The output of the program execution can be found in the file
\textemph{logs/invoke-normal.txt}.

Now you can do the same by running the rest of the programs
by launching similar scripts. In the support material you will find the following
scripts:
\textemph{run-invoke-normal.sh},
\textemph{run-matrixmult-normal.sh},
\textemph{run-picalc-normal.sh},
\textemph{run-pimontecarlo-normal.sh} and
\textemph{run-matrixmult-part-normal.sh}.

