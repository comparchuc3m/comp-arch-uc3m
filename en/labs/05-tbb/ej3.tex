\clearpage
\subsection{Exercise: Calculating the number PI}

All the code for this exercise is in the
\textemph{03-picalc} folder.

In this exercise we will see how a reduction algorithm can be optimized.
We will apply it to an algorithm that calculates the area of a circle as a
previous step to calculating the value of the number $\pi$.

\subsubsection{Sequential calculation of the number PI}

The value of the number $\pi$ can be calculated from the following definite
integral:

\[
\int_{0}^{1} \frac{1}{1+x^2} dx =
\arctan{1} - \arctan{0} =
\frac{\pi}{4}
\]

Therefore, the value of $\pi$ can be approximated by dividing the integration
interval (in this case [0,1]) into $N$ sub-intervals of length $\Delta x =
\frac{1}{N}$ and using the following development:

\[
\pi \approx 4 \cdot \sum_{i=0}^{N-1} f(i \cdot \Delta x + \frac{\Delta x}{2}) \Delta x =
4 \cdot \Delta x \sum_{i=0}^{N-1} \frac{1}{1 + ( (i + \frac{1}{2}) \Delta x )^2}
\]

In the exercise the sequential implementation of this algorithm is already provided
in the function \cppid{compute\_pi\_seq()} (see listing~\ref{lst:pi-calc-seq}).

\begin{lstlisting}[label={lst:pi-calc-seq},
caption={Sequential calculation of the number pi}]
  double compute_pi_seq(std::size_t num_steps) {
    double const deltax = 1.0 / static_cast<double>(num_steps);
    double sum          = 0.0;
    for (std::size_t i = 0; i < num_steps; ++i) {
      double const x = (static_cast<double>(i) + 0.5) * deltax;
      sum += 1.0 / (1.0 + x * x);
    }
    return 4 * deltax * sum;
  }
\end{lstlisting}

\subsubsection{Parallel version of PI calculation}

To calculate the sum of the algorithm, a reduction based on a
block range can be used with the function \cppid{tbb::parallel\_reduce()} (see
listing~\ref{lst:parallel-reduce}). This function
takes a one-dimensional block range \cppid{tbb::blocked\_range<T>} and two
lambda expressions. The first lambda expression performs a partial calculation in the
iteration space of the range. The second lambda expression specifies how the
partial results are combined to perform the reduction itself.

\begin{lstlisting}[label={lst:parallel-reduce},
caption={Block range-based reduction}]
auto result = tbb::parallel_reduce(
    tbb::blocked_range<std::size_T>(0, max),
    0.0, // Initial value
    [=](tbb::blocked_range<std::size_t> const & r, double init) {
      for (auto i=r.begin(); i!=r.end(); ++i) {
        // Perform calculations and update init
      }
      return init;
    },
    [](double x, double y) { return x+y; }
);
\end{lstlisting}

Use this type of call to modify the function \cppid{compute\_pi\_par()} and
measure the achieved speedup.
