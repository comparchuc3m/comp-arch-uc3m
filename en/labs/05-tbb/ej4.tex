\clearpage
\subsection{Estimating the Number PI Using Monte Carlo Simulation}

All the code for this exercise is in the
\textemph{04-pimontecarlo} folder.

In this exercise we will address a possible approach to using random
numbers in parallel algorithms.

\subsubsection{Monte Carlo Simulations}

A possible way to approximate the value of the number $\pi$ is by generating random
coordinates within a square of side \cppid{1.0}. The proportion of points
whose distance from the lower left corner is less than or equal to \cppid{1.0}
coincides with the value $\frac{\pi}{4}$. The value will be more accurate the greater
the number of generated values.

The \cppid{compute\_pi\_seq()} function (see Listing~\ref{lst:pi-seq-montecarlo})
generates a number of points using a random number generator of type
\cppid{std::mt19937\_64}.

\begin{lstlisting}[label={lst:pi-seq-montecarlo},
caption={Approximation of the value of PI using Monte Carlo simulation}]
  double compute_pi_seq(std::size_t num_samples, std::size_t seed) {
    std::mt19937_64 rng{seed};
    std::uniform_real_distribution gen{0.0, 1.0};
    std::size_t num_hits = 0;
    for (std::size_t i = 0; i < num_samples; ++i) {
      double const x = gen(rng);
      double const y = gen(rng);
      if (x * x + y * y <= 1.0) {
        ++num_hits;
      }
    }
    return 4.0 * static_cast<double>(num_hits) / static_cast<double>(num_samples);
  }
\end{lstlisting}

\subsubsection{Monte Carlo Simulations in Parallel}

If you observe the code of \cppid{compute\_pi\_seq()} it is a reduction.
However, the problem is that all parallel tasks
would have to share a single random number generator (the variable
\cppid{rng}), which could lead to data races, in addition to not being
deterministic.

Fortunately, Mersenne-Twister type generators have an interesting
property. In general, sequences generated from consecutive integer
seeds give rise to independent pseudo-random sequences.

Therefore, the approach for the \cppid{compute\_pi\_par()} function could be
like the one presented in Listing~\ref{lst:pi-par-montecarlo}.

\begin{lstlisting}[label={lst:pi-par-montecarlo},
caption={Parallel approximation of the value of PI using Monte Carlo simulation}]
  double compute_pi_par(std::size_t num_samples, std::size_t seed) {
    std::size_t const sum = tbb::parallel_reduce(
        tbb::blocked_range<std::size_t>(0, num_samples), std::size_t{0},
        [seed](tbb::blocked_range<std::size_t> const & r, std::size_t init) {
          std::mt19937_64 local_rng{seed + r.begin()}; // Local RNG
          std::uniform_real_distribution gen{0.0, 1.0};
          for (std::size_t i = r.begin(); i != r.end(); ++i) {
            // Update init
          }
          return init;
        },
        [](std::size_t x, std::size_t y) { return x + y; });

    return 4.0 * static_cast<double>(sum) / static_cast<double>(num_samples);
  }
\end{lstlisting}

Complete the code for \cppid{compute\_pi\_par()} and measure the
achieved speedup.
