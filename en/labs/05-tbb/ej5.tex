\clearpage
\subsection{Advanced Task Partitioning and Matrix Multiplication}

All the code for this exercise is in the
\textemph{05-matrixmult-part} folder.

In this exercise we will evaluate the different task partitioners that
TBB offers.

\subsubsection{Task Partitioning}

In TBB any parallel algorithm generates a set of tasks. These tasks
can be divided into smaller ones. In this way, each thread maintains a
list of pending tasks to perform of different sizes. When the task list of
a thread becomes empty, this thread steals a task from another thread to
progress. This mechanism is very effective in maintaining load balance
among the different threads.

A fundamental element in work distribution is the \textmark{partitioner}.
That is, the object in charge of partitioning tasks into smaller
ones.

The library offers the following strategies:

\begin{itemize}

\item \cppid{simple\_partitioner}:
The range is divided until a grain size associated with the range is reached.
If no value is specified the grain size is unity.

\item \cppid{auto\_partitioner}:
The range is divided dynamically and division can be stopped even if the
grain size is not reached.
It is the default partitioner in most algorithms.

\item \cppid{static\_partitioner}:
The range is divided uniformly but there is no possibility of load balancing.
It is the partitioner with the least overhead, but may not be suitable if there is
imbalance among tasks.

\item \cppid{affinity\_partitioner}:
Combines characteristics of \cppid{auto\_partitioner} and
\cppid{static\_partitioner} improving cache usage if the same partitioner
object is repeatedly passed to the same algorithm. It starts generating a
static partition, but later adapts it dynamically.

\end{itemize}

\subsection{Evaluating Partitioners}

This exercise includes code to evaluate different matrix multiplication algorithms
with different partitioners. In each evaluation each algorithm is
executed 10 times and average values are taken.

In this exercise you do not have to modify the code. Analyze the results with
the different partitioners and try to find explanations for the
results.
