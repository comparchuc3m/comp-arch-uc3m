\section{Tasks}

To complete this laboratory different small programs are given with their
source code. These programs must be evaluated (see next sections). A
\textmark{CMakeLists.txt} file for their compilation is also provided. 

Create a directory named \cppid{lab3} in your home directory at \cppid{avignon}.

\begin{lstlisting}[style=terminal,aboveskip=1em,belowskip=1em]
usuario@avignon-frontend:~$ mkdir lab3
usuario@avignon-frontend:~$
\end{lstlisting}

From your local machine, transfer all lab support files, 
that you have previously downloaded:

\begin{lstlisting}[style=terminal,aboveskip=1em,belowskip=1em]
usuario@mimaquina:~/Downloads/lab3$ scp * usuario@avignon.lab.inf.uc3m.es:~/lab3/
usuario@avignon.lab.inf.uc3m.es password: 
aos.cpp                                                            100%  237    20.6KB/s   00:00    
CMakeLists.txt                                                     100%  369    35.7KB/s   00:00    
loop_merge.cpp                                                     100%  254    22.7KB/s   00:00    
loop_merge_opt.cpp                                                 100%  216    17.2KB/s   00:00    
soa.cpp                                                            100%  249    24.1KB/s   00:00    
usuario@mimaquina:~/Downloads/lab3$ 
\end{lstlisting}

To compile the programs to be evaluated, 
you may use the following commands in a script:

\begin{lstlisting}[style=terminal,aboveskip=1em,belowskip=1em]
cmake -S . -B debug -DCMAKE_BUILD_TYPE=Debug
cmake --build debug
\end{lstlisting}


\textbad{NOTE:} Your lab group will be assigned one of the following configurations:

\begin{itemize}
\item Configuration 1: Line size of 32 B and caches are all 2-way set associative.
\item Configuration 2: Line size of 32 B and caches are all 4-way set associative.
\item Configuration 3: Line size of 64 B and caches are all 4-way set associative.
\item Configuration 4: Line size of 64 B and caches are all 8-way set associative.
\end{itemize}

\clearpage
\subsection{Task 1: Loop merging}

In this task two programs must be analyzed: \cppid{loop\_merge.cpp} and
\cppid{loop\_merge-opt.cpp}.


\lstinputlisting[caption={loop\_merge.cpp},frame=single,numbers=left,basicstyle=\small]{lab/03-cache/loop_merge.cpp}
\lstinputlisting[caption={loop\_merge\_opt.cpp},frame=single,numbers=left,basicstyle=\small]{lab/03-cache/loop_merge_opt.cpp}

Both programs implement the same functionality: given two vectors $\vec{z}$ and
$\vec{t}$, they compute another two vectors $\vec{u}$ y $\vec{v}$:

\[
\vec{u} = \vec{z} + \vec{t}
\]
\[
\vec{v} = \vec{u} + \vec{t}
\]

For this purpose the programs use 4 fixed size arrays.  
The programs do not print any result.

The student is asked to: 

\begin{enumerate}

\item Run \cppid{loop\_merge} and \cppid{loop\_merge\_opt}
with the program \textmark{valgrind} and
the tool \textmark{cachegrind} for the following configurations of cache:

\begin{itemize}
\item Last level cache is fixed to 128 KiB.
\item Evaluate L1D cache sizes of 16 KiB, 32 KiB and 64 KiB.
\end{itemize}

\item Get the results obtained and inspect the code with the tool
\textmark{cg\_annotate}. Annotate the global results and pay special
attention to the results on loop bodies


\item Compare results for both programs. Discuss in your report the results for
Dr, D1mr, DLmr, Dw, D1mw y DLmw. 
\end{enumerate}

\clearpage
\subsection{Task 2: Structures and Arrays}

In this task two programs will be analyzed: \cppid{soa.cpp} and
\cppid{aos.cpp}. 
Both programs implement the same functionality: sum of the
coordinates of two sets of points (\cppid{a} and \cppid{b}) with coordinates in
a bi-dimensional space.  The second one uses three arrays of structures which
represent the points and the first one uses three structures with two array
inside each one. 
The programs do not print any result.

\lstinputlisting[caption={soa.cpp},frame=single,numbers=left,basicstyle=\small]{lab/03-cache/soa.cpp}
\lstinputlisting[caption={aos.cpp},frame=single,numbers=left,basicstyle=\small]{lab/03-cache/aos.cpp}

The student is asked to: 

\begin{enumerate}

\item Run \cppid{soa} and \cppid{aos} with the program
\textmark{valgrind} and the tool \textmark{cachegrind} for the following configurations of cache:

\begin{itemize}
\item Last level cache is fixed to 256 KiB
\item Evaluate L1D cache sizes of 8KiB, 16KiB and 32 KiB
\end{itemize}

\item Get the results obtained and inspect the code with the tool
\textmark{cg\_annotate}. Annotate the global results and pay special
attention to the loops.

\item Compare both results.
Study the results for Dr, D1mr, DLmr, Dw, D1mw y DLmw.

\end{enumerate}

