\subsection{Detecting memory leaks with \emph{valgrind}}

\begin{enumerate}

\item
Create a folder named \cppid{src/numbers}.

\item 
Add a file named
\cppid{src/numbers/primitive\_vecnum.hpp}

\begin{lstlisting}
#ifndef NUMBERS_PRIMITIVE_VECNUM_HPP
#define NUMBERS_PRIMITIVE_VECNUM_HPP

#include <cstddef>

class vecnum {
public:
  explicit vecnum(std::size_t n) : size_{n}, buffer_{new double[n]{}} {}
  [[nodiscard]] std::size_t size() const { return size_; }

  double operator[](std::size_t i) const { return buffer_[i]; }
  double &operator[](std::size_t i) { return buffer_[i]; }
private:
  std::size_t size_;
  double * buffer_;
};

#endif//NUMBERS_PRIMITIVE_VECNUM_HPP
\end{lstlisting}

\item
Write the following 
\cppid{src/numbers/main.cpp} file:

\begin{lstlisting}
#include "primitive_vecnum.hpp"
#include <print>

int main() {
  vecnum v(5);
  v[2] = 3.0;
  for (std::size_t i=0; i<v.size(); ++i) {
    std::println("v[{}] = {}", i, v[i]);
  }
}
\end{lstlisting}

\item
Run the program and look at the result.

\item
From terminal window run the following command and analyze generated output:

\begin{lstlisting}[style=terminal,escapechar=!]
root!\textgood{@}!d8e0264cf0b4:/workspace/out/build/default/Debug!\textgood{\#}! valgrind --tool=memcheck ./numbers
\end{lstlisting}

\end{enumerate}
