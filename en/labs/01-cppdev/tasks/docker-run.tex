\subsection{Container execution}

\subsubsection{Running a container}

You can run a container from an image with command
\textemph{docker run}. 
You can specify the command to be executed inside the container.

Run the following command

\begin{lstlisting}[style=terminal]
docker run --rm cpp-intro-lab g++ --version
\end{lstlisting}

This command starts a container from image \textgood{cpp-intro-lab}
and runs command \textemph{g++ -{}-version}, 
to get at the terminal output the version from command \textemph{g++} 
(GCC's compiler for the C++ programming language).
Option 
\textemph{-{}-rm} 
specifies that after finishing the container must be deleted.

\subsubsection{Interactively running a container}

You can als run an interactive command shell
(option \textemph{-{}-it}) inside the container.
This will allow you to run several commands without exiting the container.

\begin{lstlisting}[style=terminal,escapechar=@]
@\$@ docker run --rm -it cpp-intro-lab
root@\@@b8521933590a:/workspace# g++ --version
g++ (Ubuntu 14.2.0-4ubuntu2~24.04) 14.2.0
Copyright (C) 2024 Free Software Foundation, Inc.
This is free software; see the source for copying conditions.  There is NO
warranty; not even for MERCHANTABILITY or FITNESS FOR A PARTICULAR PURPOSE.

root@\@@b8521933590a:/workspace# clang++ --version
Ubuntu clang version 20.1.8 (++20250804090239+87f0227cb601-1~exp1~20250804210352.139)
Target: x86_64-pc-linux-gnu
Thread model: posix
InstalledDir: /usr/lib/llvm-20/bin
root@\@@b8521933590a:/workspace# cmake --version
cmake version 4.1.1

CMake suite maintained and supported by Kitware (kitware.com/cmake).
root@\@@b8521933590a:/workspace# exit
exit
@\$@
\end{lstlisting}

\subsubsection{File sharing}

When a container is executed no file is shared between the host machine and the
container. To do it you will need to mount a volue (option \textemph{-v}) when
running the container and to specify the work directory.

\begin{lstlisting}[style=terminal,escapechar=@]
docker run --rm -it -v @\$@(pwd):/workspace cpp-intro-lab
\end{lstlisting}

Keep in mind that directory \textmark{/workspace} 
is defined as work directoyr in file \cppid{Dockerfile}.
