\subsection{Simple program compilation}

To carry out this exercises we will use the file \cppid{hello.cpp}
with the following content:

\begin{lstlisting}
#include <print>

int main() {
  std::println("Hello");
}
\end{lstlisting}

We will use several options to observe different behavior from the command line
compiler 
\textmark{g++}.

\begin{itemize}
  \item \cppid{g++ -std=c++23 hello.cpp -E -o hello.ii}: 
  Runs only the pre-processor and gets the output in file \cppid{hello.ii}.
    \begin{itemize}
      \item Try to determine how many lines of C++ code are generated
            after preprocessing \cppid{hello.cpp}.
    \end{itemize}
\begin{lstlisting}[style=terminal,escapechar=!]
root!\textgood{@}!a4794809b16b:/workspace!\textgood{\#}! g++ -std=c++23 src/hello.cpp -E hello.ii | wc -l
\end{lstlisting}


  \item \cppid{g++ -std=c++23 hello.cpp -S -o hello.s}:
  Generates assembly code for the current translation unit.
    \begin{itemize}
      \item Intenta determinar cuantas líneas de código ensamblador se generan para \cppid{hello.cpp}.
    \end{itemize}
\begin{lstlisting}[style=terminal,escapechar=!]
root!\textgood{@}!a4794809b16b:/workspace!\textgood{\#}! g++ -std=c++23 src/hello.cpp -S -o hello.s
root!\textgood{@}!a4794809b16b:/workspace!\textgood{\#}! ls -l hello.s
-rw-r--r-- 1 root root 1403 Sep  3 13:16 hello.s
root!\textgood{@}!a4794809b16b:/workspace!\textgood{\#}! wc -l hello.s 
63 hello.s
root!\textgood{@}!a4794809b16b:/workspace!\textgood{\#}! g++ -std=c++23 src/hello.cpp -S -o hello.s && wc -l hello.s 
63 hello.s
\end{lstlisting}

  \item \cppid{g++ -std=c++23 hello.cpp -c -o hello.o}:
  Generates an object file \cppid{hello.o} that can be linked to other object
  files.
    \begin{itemize}
      \item Try to determine how many bytes are required for file \cppid{hello.o}.
    \end{itemize}
\begin{lstlisting}[style=terminal,escapechar=!]
root!\textgood{@}!a4794809b16b:/workspace!\textgood{\#}! g++ -std=c++23 src/hello.cpp -c -o hello.s
root!\textgood{@}!a4794809b16b:/workspace!\textgood{\#}! ls -l hello.o
-rw-r--r-- 1 root root 1896 Sep  3 13:22 hello.o
\end{lstlisting}

  \item \cppid{g++ -std=c++23 hello.cpp -o hello}:
        Generates an executable file named \cppid{hello}.
    \begin{itemize}
      \item Try to determine how many bytes are required for the binary
            executable \cppid{hello}.
\begin{lstlisting}[style=terminal,escapechar=!]
root!\textgood{@}!a4794809b16b:/workspace!\textgood{\#}! g++ -std=c++23 src/hello.cpp -o hello
root!\textgood{@}!a4794809b16b:/workspace!\textgood{\#}! ls -l hello
-rwxr-xr-x 1 root root 16352 Sep  3 13:23 hello
root!\textgood{@}!a4794809b16b:/workspace!\textgood{\#}! ./hello
Hello C++
\end{lstlisting}
    \end{itemize}
\end{itemize}

There are some other related tools that might be useful to a systems developer.
However they are not covered in this lab.
Try to find information about the following tools:

\begin{itemize}
  \item \cppid{nm}
  \item \cppid{readelf}
  \item \cppid{objdump}
  \item \cppid{c++filt}
\end{itemize}

