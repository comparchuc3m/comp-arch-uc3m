\subsection{Fixing memory leaks}

Try to fix the memory leak with at least one of the following alternatives:

\begin{enumerate}

\item 
Add a destructor to the existing class
\begin{itemize}
  \item Make sure that the destructor uses
        \cppkey{delete []} \cppid{var} 
        instead of \cppkey{delete} \cppid{var}.
\end{itemize}

\item 
Modify the data member of type \cppkey{double*} 
to become a data member of type
\cppid{std::unique\_ptr<double[]>}.
Keep in mind the following:

\begin{itemize}
  \item Include header \cppid{<memory>}.

  \item Modify data member \cppid{buffer\_} to type \cppid{std::unique\_ptr<double[]>}.

  \item Replace \cppkey{new} in the constructor to a call to 
        \cppid{std::make\_unique<double[]>(n)}.

  \item Include header \cppid{<algorithm>}.

  \item Add a copy constructor.

\begin{lstlisting}
vecnum(const vecnum & v) : size_{v.size_}, buffer_{std::make_unique<double[]>(size_) {
  std::copy_n(v.buffer_.get(), v.size_, buffer_.get());
}
\end{lstlisting}
\end{itemize}

\end{enumerate}

After that, analyze the effects on memory leaks when using the following
program 
\cppid{main.cpp}:

\begin{lstlisting}
#include <print>
#include "primitive_vecnum.hpp"

int main() {
  vecnum v(5);
  v[2] = 3.0;

  vecnum w{v};

  for (std::size_t i=0; i<v.size(); ++i) {
    std::println("v[{}] = {}", i, v[i]);
  }

  for (std::size_t i=0; i<w.size(); ++i) {
    std::println("w[{}] = {}", i, w[i]);
  }
}
\end{lstlisting}
