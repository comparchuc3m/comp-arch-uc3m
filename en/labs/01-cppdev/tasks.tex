\section{Tasks}

\subsection{Check your compiler version}

The compiler interface for the gcc compiler is the \textmark{g++} command.

First, you will check the compiler version you have:

\begin{lstlisting}[style=terminal]
user$machine:~$g++ --version
\end{lstlisting}

You should get the version number, if the g++ command is found.

\subsubsection{In your own machine}

\begin{enumerate}

\item Install the \textmark{build-essential} package:

\begin{lstlisting}[style=terminal]
user$machine:~$sudo apt install build-essential
\end{lstlisting}

\item Check the compiler version

\begin{lstlisting}[style=terminal]
user$machine:~$g++ --version
\end{lstlisting}

\item If you want to install multiple versions of the compiler,
      you may want to read the following pages:
  \begin{itemize}
    \item \url{https://itslinuxfoss.com/install-gcc-ubuntu-22-04}
    \item \url{https://linuxize.com/post/how-to-install-gcc-on-ubuntu-20-04/}
    \item \url{https://lindevs.com/install-gcc-on-ubuntu}
  \end{itemize}
\end{enumerate}

\subsection{Compile a simple program}

For this compilation exercise we will use the \cppid{hello.cpp} file with the following contents:

\begin{lstlisting}
#include <iostream>

int main() {
  using namespace std;

  cout << "Hello C++\n";
}
\end{lstlisting}

We will use a number of options to get different behaviors from the \textmark{g++} command line interface:

\begin{itemize}
  \item \cppid{g++ hello.cpp -E -o hello.ii}: 
  Run only the preprocessor and get the output to file \cppid{hello.ii}.
    \begin{itemize}
      \item Try to find out how many C++ lines of code are generated after preprocessiong \cppid{hello.cpp}.
    \end{itemize}

  \item \cppid{g++ hello.cpp -S -o hello.s}:
  Generate assembly code for the current translation unit.
    \begin{itemize}
      \item Try to find how many lines of assembly code are generated for \cppid{hello.cpp}.
    \end{itemize}

  \item \cppid{g++ hello.cpp -c -o hello.o}:
  Generate an object file \cppid{hello.o} that can be linked with other object files.
    \begin{itemize}
      \item Try to find how many bytes are required for object file \cppid{hello.o}.
    \end{itemize}

  \item \cppid{g++ hello.cpp -o hello}:
  Generate an executable file \cppid{hello}.
    \begin{itemize}
      \item Try to find how many bytes are required for binary executable \cppid{hello}.
    \end{itemize}
\end{itemize}

There are other related tools that might be useful for you as systems developer, but are not
covered in this lab. Try to find information about the following:

\begin{itemize}
  \item \cppid{nm}
  \item \cppid{readelf}
  \item \cppid{objdump}
  \item \cppid{c++filt}
\end{itemize}

\subsection{Using \textbf{CLion} development environment}

\begin{enumerate}

\item
Launch the CLion IDE.
When you are asked for a license, use the user and password
that you got when registering your student account at JetBrains site.

\item
Create a new project named \cppid{hello}:

\begin{itemize}
  \item Project type: C++ Executable.
  \item Language standard: C++20
\end{itemize}

\item
Check that the following files were generated:

\begin{itemize}
  \item \cppid{main.cpp}: Modify it to write our \cppid{hello.cpp} content in that file.
  \item \cppid{CMakeLists.txt}
\end{itemize}

\item
Build the program through \textmark{Build | Build Project} and check that the program was correctly compiled.

\item
Run the program through \textmark{Run | Run 'hello'}. Check that you can see the program output from the IDE.

\end{enumerate}

\subsection{Using multiple source files}

In this section we will write a project with multiple source files:

\begin{enumerate}

\item
Create a new project named \cppid{areas}.

\item
Change the extension for header files by following the following steps:
\begin{itemize}
  \item Menu \textmark{File | Settings}.
  \item In the dialog box select \textmark{Editor | Code Style | C/C++}.
  \item Select the tab for \textmark{New File Extensions}.
  \item Change \textmark{Header extensions} to \textgood{hpp}.
  \item Change \textmark{File Naming convention} to \textgood{snake\_case}.
\end{itemize}

\item
Add a file named \cppid{geom.hpp}

\begin{lstlisting}
#ifndef AREAS_GEOM_HPP
#define AREAS_GEOM_HPP

namespace geom {

  double area(double w, double h);

}

#endif
\end{lstlisting}

\item 
Add a file named \cppid{geom.cpp}:

\begin{lstlisting}
#include "geom.hpp"

namespace geom {
  double area(double w, double h) {
    return w * h;
  }
}
\end{lstlisting}

\item
Modify your program \cppid{main.cpp}:

\begin{lstlisting}
#include <iostream>
#include "geom.hpp"

int main() {
  double x = 2.0;
  double y = 3.0;
  std::cout << "area(" << x << "," << y << ")= " << geom::area(x,y) << "\n";
}
\end{lstlisting}

\item
Modify your \cppid{CMakeLists.txt} file to add some compile options to enable warnings:

\begin{lstlisting}
...
set(CMAKE_CXX_STANDARD 20)
add_compile_options(-Wall -Wextra -Werror -pedantic -pedantic-errors)

add_executable(shape main.cpp geom.hpp geom.cpp)
\end{lstlisting}

\item
Run the program.

\end{enumerate}

\subsection{Debug version versus Release Version}

When you use CLion the default compilation is the Debug version.
However, when you are not debugging you prefer to get an optimized \textgood{Release} version.

\begin{enumerate}

\item
Open the \textgood{settings dialog} through \textmark{File | Settings}.

\item
Select \textmark{Build, Execution, Deployment | CMake}.

\item 
You will see the \textgood{Debug} configuration.

\item 
Click in the \textmark{+} symbol to add another configuration. The \textgood{Release} configuration
will be added.

\item
Apply the configuration change.

\item
In the build drop down list, you now may select \cppid{areas | Debug} or \cppid{areas | Release}.

\end{enumerate}

\subsection{Finding memory leaks with valgrind}

\begin{enumerate}

\item
Create an application named \cppid{numbers}.

\item Add a file named \cppid{primitive\_vecnum.hpp}

\begin{lstlisting}
#ifndef NUMBERS_PRIMITIVE_VECNUM_HPP
#define NUMBERS_PRIMITIVE_VECNUM_HPP

class vecnum {
public:
  explicit vecnum(int n) : size_{n}, buffer_{new double[n]{}} {}
  [[nodiscard]] int size() const { return size_; }

  double operator[](int i) const { return buffer_[i]; }
  double &operator[](int i) { return buffer_[i]; }
private:
  int size_;
  double * buffer_;
};

#endif//NUMBERS_PRIMITIVE_VECNUM_HPP
\end{lstlisting}

\item
Write the following \cppid{main.cpp}

\begin{lstlisting}
#include <iostream>
#include "primitive_vecnum.hpp"

int main() {
  vecnum v(5);
  v[2] = 3.0;
  for (int i=0; i<v.size(); ++i) {
    std::cout << "v[" << i << "] = " << v[i] << '\n';
  }
}
\end{lstlisting}

\item
Run the program and check the results.

\item
Run the program through \textmark{Run | Run with valgrind memcheck} and analyze the output.

\end{enumerate}

\subsection{Performing dynamic analysis with sanitizers}

There are several \textmark{sanitizers} available in some C++ compilers:

\begin{itemize}
  \item \textmark{Address Sanitizer}: Detects memory errors.
  \item \textmark{Leak Sanitizer}: Subset of Address Sanitizer only for memory leaks.
  \item \textmark{Thread Sanitizer}: Detects concurrency errors.
  \item \textmark{Undefined Behavior Sanitizer}: Dectects some undefined behaviors.
  \item \textmark{Memory Sanitizer} (only with \cppid{clang++}): Detects use of uninitialized memory.
\end{itemize}

You can use them by adding extra profiles in CLion:

\begin{enumerate}

\item
Open the settings dialog through \textmark{File | Settings}.

\item
Select \textmark{Build, Execution, Deployment | CMake}.

\item
Copy the \textmark{Debug} profile into a new one and name it as \textmark{Debug-address-sanitizer}.

\item
In options field add \cppid{-DCMAKE\_CXX\_FLAGS="-fsanitize=address -fno-omit-frame-pointer"}.

\end{enumerate}

\subsection{Fixing the memory leak}

Try to fix the memory leak in at least one of the following ways:

\begin{enumerate}

\item Add a destructor to the existing class.
\begin{itemize}
  \item Make sure that you use in a destructor 
        a \cppkey{delete []} \cppid{var} 
        and not a \cppkey{delete} \cppid{var}.
\end{itemize}

\item Change the \cppkey{double*} data member by a \cppid{std::unique\_ptr<double[]>} data member.
Consider the following:

\begin{itemize}
  \item Include the \cppid{<memory>} header.
  \item Change the \cppid{buffer\_} data member to type \cppid{std::unique\_ptr<double[]>}.
  \item Replace the \cppkey{new} in the constructor by a call to \cppid{std::make\_unique<double[]>(n)}.
  \item Include the \cppid{<algorithm>} header.
  \item Add a copy constructor.
\begin{lstlisting}
vecnum(const vecnum & v) : size_{v.size_}, buffer_{std::make_unique<double[]>(size_) {
  std::copy_n(v.buffer_.get(), v.size_, buffer_.get());
}
\end{lstlisting}
\end{itemize}

\end{enumerate}

Then, analyze the effects on memory leaks when you use the following \cppid{main.cpp} program:

\begin{lstlisting}
#include <iostream>
#include "primitive_vecnum.hpp"

int main() {
  vecnum v(5);
  v[2] = 3.0;

  vecnum w{v};

  for (int i=0; i<v.size(); ++i) {
    std::cout << "v[" << i << "] = " << v[i] << '\n';
  }

  for (int i=0; i<w.size(); ++i) {
    std::cout << "w[" << i << "] = " << w[i] << '\n';
  }
}
\end{lstlisting}

