\section{Compiler optimizations}

\begin{frame}[t]{Compiler optimizations}
\begin{itemize}
  \item \textmark{Goal}:
        Generate code with a reduced number of instructions and data misses.

  \mode<presentation>{\vfill\pause}
  \item \textgood{Instructions}: 
    \begin{enumerate}[1.]
      \item Procedure reordering.
      \item Align code blocks to cache line start.
      \item Branch linearization.
    \end{enumerate}

  \mode<presentation>{\vfill}
  \item \textgood{Data}: 
    \begin{enumerate}[1.]
      \item Array merge.
      \item Loop interchange.
      \item Loop merge.
      \item Blocked access.
    \end{enumerate}
\end{itemize}
\end{frame}

\begin{frame}[t]{Procedure reordering}
\begin{columns}

\column{.7\textwidth}

\begin{itemize}
  \item \textgood{Goal}: Decrease \textbad{conflict misses} due to
        two concurrent procedures are mapped to the
        \textmark{same cache line}.

  \mode<presentation>{\vfill}
  \item \textmark{Technique}: Reorder procedures in memory.

\end{itemize}

\column{.3\textwidth}
\pause
\input{int/m3-02-cache/reorder-proc.tkz}

\end{columns}
\end{frame}

\begin{frame}[t]{Basic block alignment}
\begin{itemize}
  \item \textmark{Definition}: 
        A \textgood{basic block} is a set of instructions
        sequentially executed (contains no branches).

  \mode<presentation>{\vfill\pause}
  \item \textmark{Goal}: Decrease  the
        \textbad{cache misses} possibility for sequential codes.

  \mode<presentation>{\vfill\pause}
  \item \textmark{Technique}: 
        Align the \textgood{first instruction} in a basic block
        with the \textgood{first word} in cache line.
\end{itemize}
\end{frame}

\begin{frame}[t]{Branch linearization}
\begin{itemize}
  \item \textmark{Goal}: 
        Decrease cache misses due to conditional branches.

  \mode<presentation>{\vfill\pause}
  \item \textgood{Technique}: If the compiler detects a branch is likely
        to be taken, it may invert condition and interchanges basic blocks
        in both alternatives.
    \begin{itemize}
      \item Some compilers define extensions to hint the compiler.
      \item \textmark{Example}: \textgood{ISO C++} (\cppkey{[[likely]]}, \cppkey{[[unlikely]]}).
    \end{itemize}

\end{itemize}
\end{frame}

\begin{frame}[t,fragile]{Array merge}
\begin{columns}[T]

\column{.5\textwidth}
\begin{block}{Parallel arrays}
\begin{lstlisting}
vector<int> key;
vector<int> val;

for (int i=0;i<max;++i) {
  cout << key[i] << "," 
       << val[i] << "\n";
}
\end{lstlisting}
\end{block}

\pause
\column{.5\textwidth}
\begin{block}{Merged array}
\begin{lstlisting}
struct entry {
  int key;
  int val;
};
vector<entry> v;

for (int i=0;i<max;++i) {
  cout << v[i].key << "," 
       <<  v[i].val << "\n";
}
\end{lstlisting}
\end{block}

\end{columns}

\begin{itemize}
  \item Decrease conflicts.
  \item Improve spatial locality.
\end{itemize}
\end{frame}

\begin{frame}[t,fragile]{Loop interchange}
\begin{columns}[T]

\column{.5\textwidth}
\begin{block}{Striped accesses}
\begin{lstlisting}
for (int j=0; j<100; ++j) {
  for (int i=0; i<5000; ++i) {
    v[i][j] = k * v[i][j];
  }
}
\end{lstlisting}
\end{block}

\pause
\column{.5\textwidth}
\begin{block}{Sequential accesses}
\begin{lstlisting}
for (int i=0; i<5000; ++i) {
  for (int j=0; j<100; ++j) {
    v[i][j] = k * v[i][j];
  }
}
\end{lstlisting}
\end{block}
\end{columns}

\mode<presentation>{\vfill}
\begin{itemize}
  \item \textmark{Goal}: Improve spatial locality.
  \item Depends on the storage model defined by the programming language.
    \begin{itemize}
      \item FOTRAN versus C.
    \end{itemize}
\end{itemize}
\end{frame}

\begin{frame}[t,fragile]{Loop merge}
\begin{columns}[T]

\column{.5\textwidth}
\begin{block}{Independent loops}
\begin{lstlisting}
for (int i=0; i<rows; ++i) {
  for (int j=0; j<cols; ++j) {
    a[i][j] = b[i][j] * c[i][j];
  }
}
for (int i=0; i<rows; ++i) {
  for (int j=0; j<cols; ++j) {
    d[i][j] = a[i][j] + c[i][j];
  }
}
\end{lstlisting}
\end{block}

\pause
\column{.5\textwidth}
\begin{block}{Merged loop}
\begin{lstlisting}
for (int i=0; i<rows; ++i) {
  for (int j=0; j<cols; ++j) {
    a[i][j] = b[i][j] * c[i][j];
    d[i][j] = a[i][j] + c[i][j];
  }
}
\end{lstlisting}
\end{block}
\end{columns}

\mode<presentation>{\vfill}
\begin{itemize}
  \item \textmark{Goal}: Improve temporal locality.
  \item \textbad{Beware}: It may decrease spatial locality.
\end{itemize}

\end{frame}

\begin{frame}[t,fragile]{Blocked access}
\begin{columns}[T]

\column{.4\textwidth}
\begin{block}{Original product}
\begin{lstlisting}
for (int i=0; i<size; ++i) {
  for (int j=0; j<size; ++j) {
    double r = 0;
    for (int k=0; k<size; ++k) {
      r += b[i][k] * c[k][j];
    }
    a[i][j] = r;
  }
}
\end{lstlisting}
\end{block}

\pause
\column{.6\textwidth}
\begin{block}{Blocked product}
\begin{lstlisting}
for (bj=0; bj<size; bj+=bsize) {
  for (bk=0; bk<size; bk +=bsize) {
    for (i=0; i<size; ++i) {
      for (j=bj; j<std::min(bj+bsize,size); ++j) {
        double r = 0;
        for (k=bk; k<std::min(bk+bsize,size); ++k) {
          r += b[i][k] * c[k][j];
        }
        a[i][j] += r;
      }
    }
  }
}
\end{lstlisting}
\end{block}
\end{columns}

\mode<presentation>{\vfill}
\begin{itemize}
  \item \cppid{bsize}: Block factor
\end{itemize}

\end{frame}
