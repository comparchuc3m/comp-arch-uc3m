\begin{acexercise}\end{acexercise}

Given a computer working continuously and without failures until time $t=20$ months.

\begin{enumerate}

\item What is the reliability of that computer for $t=0$, $t=24$ and $t=30$ (where $t$ is expressed in months).

\item If during two years of usage, the computer has two failures whose repair times are $3.85$ and $4.15$ days respectively, what would be its availability?

\end{enumerate}

\begin{acsolution}\end{acsolution}

Reliability ($R$) is the probability that the system lifetime ($X$)
is greater than a given time $t$. That is $P[X>t]$.
Reliability is a function of time and fulfills the following:

\[ R(t=0)=1 \]
\[ R(t=\infty)=0 \]
\[ R(0<t<\infty) \in [0,1] \]

In the example:

\[ R(t=0)=1, R(t=24)=0, R(t=30)=0 \]

Availability ($A$) is the fraction of time that the system is working correctly, or free of errors.
Formally, the average availability will be:

\[
A=\frac{MTTF}{MTTF+MTTR}
\]

Where $MTTF$ is the average time between failures and $MTTR$ is the average time to repair.

Consequently, the availability of the system in the example:

\[ MTTR=3.85+4.15=8 \]
\[ MTTF=365*2-8=722 \]
\[ A=\frac{722}{730}= 0,9890 \rightarrow 98.9\% \]

