\begin{acexercise}\end{acexercise}

An application allows to process a very high resolution image where a certain
fraction can be parallelized, while other part must be run sequentially. 

Assume that there is no upper bound to the number of processes that can be used
for the parallelization. We need to get a global speedup of at least 10 for
the parallel version. 

Express the fraction of code that must be parallelized as a function of the
parallelism degree (number of processes running in parallel).


\begin{acsolution}\end{acsolution}

\[
S = \frac{1}{\left(1 - F \right) + \displaystyle \frac{F}{n} }
\]

\[
S \times \left( (1-F) + \frac{F}{n} \right) = 1
\]

\[
S - S \times F + \frac{S \times F}{n} = 1
\]

\[
n \times S - n \times S \times F + S \times F = n
\]

\[
n \times S - n = n \times S \times F - S \times F
\]

\[
S \times F \times \left( n - 1 \right) = n \times \left( S - 1 \right)
\]

\[
F = \frac{n \times (S-1)}{S \times(n-1)}
\]

For the case of $S = 10$

\[
F = \frac{n \times (10-1)}{10 \times (n-1)} = 
\frac{9 \times n}{10 \times n - 10}
\]

As $F$ must be less or equal than $1$:

\[
F \leq 1
\]

\[
\frac{9 \times n}{10 \times n -10} \leq 1
\]

\[
9 \times n \leq 10 \times n-10
\]

\[
n \geq 10
\]


