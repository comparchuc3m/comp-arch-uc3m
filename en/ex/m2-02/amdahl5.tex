\begin{acexercise}\end{acexercise}

A single core computer runs a finance risk assessment application. The
application is computation intensive (computations take 90\% of total 
execution time). The remaining 10\% is devoted for waiting for I/O operations.

The time that the applications is running computation instructions is divided into
75\% for floating point operations and 25\% for other instructions. Executing
a floating point operation requires, on average, 12 CPI. The rest of instructions
require, on average, 4 CPI.

Migrating this application to a new machine is being evaluated. The following
alternatives are considered. In both alternatives, there is no improvement in
the I/O time for disk.

\begin{itemize}
  \item \textbf{Alternative A}: 
        A single-core process with clock frequency 50\% higher than the original
        machine, where floating point instructions require 10\% more cycles per
        instruction and the rest of instructions require 25\% more cycles per
        instruction.

  \item \textbf{Alternative B}: 
        A four-core process with a clock frequency 50\% lower than the original
        machine, where floating point instructions require 20\% less cycles per
        instruction and the rest of instructions the same number of cycles per
        instructions.
\end{itemize}

State the following questions giving an appropriate reasoning:
\begin{enumerate}

  \item Which is the global speedup/slowdown for the application in case A?

  \item Which is the global speedup/slowdown for the application in case B,
        assuming that the computation part can be fully parallelized while
        the I/O part cannot be improved at all?
\end{enumerate}


\begin{acsolution}\end{acsolution}

Time for executing instructions in the original computer will be:

\begin{equation}
T_{orig} = 
0.75 \times 12 \times IC \times P + 0.25 \times 4 \times IC \times P = 
(9+1) \times IC \times P
\end{equation}

\paragraph{Alternative A}

Time for executing instructions in computer A will be:

\begin{equation}
T_{A} =
(0.75 \times (1,1 \times 12) + 0.25 \times (1.25 \times 4)) \times IC \times \frac{P}{1.5} = 
\frac{(9.9 + 1.25) \times IC  \times P}{1.5} = 
\frac{11.15}{1.5} \times IC * P 
\end{equation}

\begin{equation}
T_{A} = 7.433 \times IC \times P
\end{equation}

Speedup due to instructions will be:

\begin{equation}
S_{A}^{I} = 
\frac{T_{orig}}{T_{A}} = 
\frac{10}{7.433} = 
1.345
\end{equation}

Applying Amdahl's law, the global speedup will be:

\begin{equation}
S_{A} = \frac{1}{0.1 + \frac{0.9}{1.345}} = 1.3
\end{equation}

\paragraph{Alternative B}

In this case, assuming complete parallelization for the computing part,
we may consider the number of instructions to be executed in each core is
one fourth from the original.

\begin{equation}
T_{B} = 
(0.75 \times 0.8 \times 12 + 0.25 \times 4) \times \frac{IC}{4} \times \frac{P}{0.5} = 
( 7.2 + 1) \times \frac{2}{4} \times IC \times P = 
\end{equation}

\begin{equation}
T_{B} = 4.1 \times IC \times P
\end{equation}

The speedup due to instructions will be:

\begin{equation}
S^{I}_{B} = \frac{T_{orig}}{T_{B}} = \frac{10}{4.1} = 2.439
\end{equation}

Applying Amdahl's law, the global speedup will be:

\begin{equation}
S_{B} = \frac{1}{0.1 + \frac{0.9}{2.439}} = 2.132
\end{equation}

