\begin{acexercise}\end{acexercise}

A computer has a value of 1.0 for CPI in ideal conditions
(when all accesses are hits).

25\% of instructions are of load/store type.
There is no other kind of instruction accessing to memory.
Miss penalty is 50 cycles and miss rate is 5\%.

What is the speedup when there is no miss compared to the case
when there are misses?

\begin{acsolution}\end{acsolution}

\[
t_{cpu} = (cycles_cpu + cycles_{stall}) \times t\_{cycle}
\]

For the ideal case, there are no memory stalls and consequently:

\[
t_{cpu} = IC \times CPI \times t_{cycle} =
IC \times t_{cycle}
\]

For the case with misses, we have:

\[
cycles_{stall} =
IC \times acceses_{instr} \times (1 - h) \times {penalty}_{fallo}
\]

Instrucction acceses are $0.25$ as 25\% of instructions perform a data access.
The hit rate $h$ is $0.95$.
Miss penalty is $50$ cycles.

\[
cycles_{stall} =
IC \times 0.25 \times 0.05 \times 50 =
0.625 IC
\]

Thus, CPU time is:

\[
t_{cpu} = (IC \times 1.0 + 0.625 \times IC) \times t_{cycle} =
1.625 \times IC \times t_{cycle}
\]

And speedup is:

\[
S = \frac{1.625 \times IC \times t_{cycle}}{IC \times t_{cycle}} = 1.625
\]
