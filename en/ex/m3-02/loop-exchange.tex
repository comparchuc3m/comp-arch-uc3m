\begin{acexercise}\end{acexercise}

Consider the following code:

\begin{lstlisting}
for (i=0; i<64; i++)
  for (j=0; j<1024; j=j+2 )
    v[i][j] = v[i][j] * v[i][j+1] +  b[i]
\end{lstlisting}

Assume an architecture with 256 KB size cache and a block size of 64 bytes and
word size of 8 bytes. Cache memory is fully associative and uses an LRU replacement
policy. Let \cppid{v} and \cppid{b} be 8 byte real number matrices stored by rows
(C programming language style) and sizes
 $64 \times 1024$ (\cppid{v}), $64$ (\cppid{b}). 
Assume that index variables are stored in processor registers and cache is initially empty.

Your are asked to determine:

\begin{enumerate}
  \item The number of cache misses.

  \item The average miss rate.

  \item Reason whether there is some time a replacement of a previously loaded cache line.
        It is not needed to identify the replacements but it is enough to reason
        whether they could happen or not.

  \item Reason if the loop exchanges optimization technique would improve the
        average miss rate for the cache (it is not needed to compute it).

\end{enumerate}


\begin{acsolution}\end{acsolution}

\paragraph{Point 1}

Cache of 256 KB $\rightarrow$ $2^{15}$ elements = $2^{12}$  blocks

Block if 64B = 8 elements

Table~\ref{tab:sol-oct-2013b} shows the access pattern.

\begin{table}
\begin{tabular}{|l|p{0.3\textwidth}|l|p{0.3\textwidth}|}
\hline
Access to \cppid{v} &
Misses &
Access to \cppid{b} &
Misses
\\
\hline
\hline

0,0 & YES &
0 & YES
\\
\hline

0,2 & No because it is on the same cache line as \cppid{v[0,0]} &
0 & No because it is on the same cache line as \cppid{b[0,0]}
\\
\hline


\ldots & &
0 & No
\\
\hline

0,7 & NO &
0 & No
\\
\hline

0,8 & SI &
0 & No
\\
\hline
	
0,10 & NO &
0 & No
\\
\hline

\ldots & & 
 & \\

0,16 & YES & 
0 & No
\\
\hline

\ldots & & 
 & \\
			
0,1022 & NO &
0 & No
\\
\hline

1,0 & YES &
1 & NO\\
\hline 

1,1 & No, Because it is on the same cache line as \cppid{v[1,0]} &
1 & NO\\
\hline 

\ldots & & & \\
				
2,0 & YES & 2 & NO\\
\hline

\ldots &&&\\
				
8,0 & YES & 8 & YES\\
\hline

\ldots &&&\\

31,0 & YES & 31 & NO\\
\hline

\end{tabular}
\caption{Access pattern of the Exercise~\ref{sol:oct-2013b}.}
\label{tab:sol-oct-2013b}
\end{table}

Cache misses occur when accessing
\cppid{v[0,0]}, \cppid{v[0,8]}, \cppid{v[0,16]}, \ldots 
\cppid{v[1,0]}, \cppid{v[1,8]}, \cppid{v[0,16]}, \ldots
\cppid{[2,0]}, \ldots \cppid{v[31,0]}, \ldots 
and there are hits in all the other cases. Since the array has $64 \times 1024$ 
elements and there is only one miss in 8, the total number of misses is $\frac{64 \cdot 1024}{8} = 8192$.

In \cppid{b}, there are misses at 1, 8, 16, 24, 32, 40, 48 and 56.
There are 8 misses in total.

The total number of misses is $8192 + 8 = 8200$.

\paragraph{Point 2}

Total accesses.

In each instruction, there are 4 accesses and the calculation instruction is 
done $\frac{64 \cdot 1024}{2}$ times. Therefore, we have $\frac{4 \cdot 64 \cdot 1024}{2} = 131072$ accesses and the miss rate is:

\[
m = \frac{8200}{131072} = 0.0625 \Rightarrow 6,25 \%
\] 

\paragraph{Point 3}

Cache size 256 KB = $2^{15}$ elements = $2^{12}$ and the matrix v 
has $64 \cdot 1024 = 2^{16}$ elements = $2^{13}$ blocks.

Therefore,  the \cppid{v} array does not fit in the cache. There will be a replacement 
of the line containing the \cppid{v [0,0]} position when we get approximately 
to the middle of the \cppid{v} array.

If the matrix \cppid{b} does not exist, the replacement would occur on reaching 
the middle of the matrix (because only half of the matrix fits in the cache), that is, 
in the position \cppid{v [32,0]} .

Given that the array \cppid{b} also occupies cache and when we reach the element 
\cppid{v [32,0]} we have occupied 4 lines of cache with the array \cppid{b} (elements from 
\cppid{b[0]} to \cppid{b[32]}), we can conclude that the replacement will 
produce 4 lines of cache before reaching \cppid{v[32,0]}. This is in the position 
\cppid{v[$31,8 \cdot \frac{1024}{8}-4$]} = \cppid{v[31,992]}.  This number is not 
explicitly asked in the exercise.

Similarly, a replacement of the \cppid{b [0]} cache line will occur at approximately 
the same time.

There will also be successive replacements of \cppid{v [0,8]}, \cppid{v [0,16]}, 
\ldots as space is needed for positions of \cppid{v}.

\paragraph{Point 4}

The loops exchange would decrease the spatial locality in the accesses 
to \cppid{v}, increasing the miss rate.
