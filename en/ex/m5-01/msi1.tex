\begin{acexercise}\end{acexercise}
\label{ex:m5-01:msi-01}

Given a multiprocessor with a symmetric shared memory architecture based
on bus with a snooping protocol. Each processor has a cache
whose consistency is kept using the MSI protocol. Each cache uses
direct mapping and has four blocks each with two words. This
cache uses the whole memory address as the tag field.

The following tables show the status of each cache memory, with the less
significant word on the left.

\medskip

\begin{tabular}{|l|l|l|l|}
\hline
\multicolumn{4}{|c|}{Processor P0}\\
\hline
\textbf{Block} &
\textbf{State} &
\textbf{State} &
\textbf{Data}
\\
\hline
\hline

B0 & I & 0x00100700 & 0x00000000 0x7FAABB11\\
\hline

B1 & S & 0x00100708 & 0x00000000 0x00001234\\
\hline

B2 & M & 0x00100710 & 0x00000000 0x0077AABB\\
\hline

B3 & I & 0x00100718 & 0x00000000 0x7FAABB11\\
\hline
\end{tabular}

\medskip

\begin{tabular}{|l|l|l|l|}
\hline
\multicolumn{4}{|c|}{Processor P1}\\
\hline
\textbf{Block} &
\textbf{State} &
\textbf{State} &
\textbf{Data}
\\
\hline
\hline

B0 & I & 0x00100700 & 0x00000000 0x7FAABB11\\
\hline
B1 & M & 0x00100728 & 0x00000000 0xFF000000\\
\hline
B2 & I & 0x00100710 & 0x00000000 0xEEEE7777\\
\hline
B3 & S & 0x00100718 & 0x00000000 0x7FAABB11\\
\hline
\end{tabular}

\medskip

\begin{tabular}{|l|l|l|l|}
\hline
\multicolumn{4}{|c|}{Processor P2}\\
\hline
\textbf{Block} &
\textbf{State} &
\textbf{State} &
\textbf{Data}
\\
\hline
\hline

B0 & S & 0x00100720 & 0x00000000 0x1111AAAA\\
\hline
B1 & S & 0x00100708 & 0x00000000 0X00001234\\
\hline
B2 & I & 0x00100710 & 0x00000000 0x7FAABB11\\
\hline
B3 & I & 0x00100718 & 0x00001234 0x1111AABB\\
\hline

\end{tabular}

\medskip

For each of the cases that are presented below, start with the
initial situation of the problem, without taking into account changes in
previous cases. Indicate the changes that occur in the caches. 
In the case of reads, also indicate what is the value actually read.


\begin{enumerate}
  \item P2: write 0x00100708, 0xFFFFFFFF
  \item P2: read 0x00100708
  \item P2: read 0x00100718
\end{enumerate}

In each case you must fill in a table with the following format,
justifying the answer:

\begin{tabular}{|l|l|l|l|l|l|p{2cm}|p{2cm}|}

\hline
\textbf{Processor} &
\textbf{Block} &
\textbf{State} &
\textbf{State} &
\textbf{State} &
\multicolumn{2}{|c|}{\textbf{Data}}
\\

&
&
\textbf{previous} &
\textbf{Current} &
&
\multicolumn{2}{|c|}{}
\\
\hline
\hline

& & & & & \multicolumn{1}{|p{2cm}|}{} & \\
\hline
& & & & & \multicolumn{1}{|p{2cm}|}{} & \\
\hline
& & & & & \multicolumn{1}{|p{2cm}|}{} & \\
\hline
& & & & & \multicolumn{1}{|p{2cm}|}{} & \\
\hline
\end{tabular}

\begin{acsolution}\end{acsolution}


\paragraph{Section 1}

A write occurs in block B1 of the P2 cache that is in
state S. This produces the following changes in P2:


\begin{itemize}

\item Transition to modified state (M).

\item A invalidation is placed on the bus

\end{itemize}

The invalidation is ignored by the processor P1, but is treated by the
P0 processor.

In P0 the following changes occur:


\begin{itemize}

\item Transition to invalid state (I).

\end{itemize}

In this case there are no changes in the main memory.

The states in the different processors are shown in the Table ~\ref{tab:ex:m5-01:msi-01:1}.

\begin{table}[htbp]

\begin{tabular}{|l|l|l|l|l|}

\hline
\multicolumn{5}{|c|}{\textbf{Processor P0}}\\
\hline
Block & State & Label & \multicolumn{2}{c|}{Data}\\
\hline
\hline
B0 & I & 0x00100700 & 0x00000000 & 0x7FAABB11\\
\hline
B1 & I & 0x00100708 & 0x00000000 & 0x00001234\\
\hline
B2 & M & 0x00100710 & 0x00000000 & 0x0077AABB\\
\hline
B3 & I & 0x00100718 & 0x00000000 & 0x7FAABB11\\
\hline

\hline
\multicolumn{5}{|c|}{\textbf{Processor P1}}\\
\hline
Block & State & Label & \multicolumn{2}{c|}{Data}\\
\hline
\hline

B0 & I & 0x00100700 & 0x00000000 & 0x7FAABB11\\
\hline
B1 & M & 0x00100728 & 0x00000000 & 0xFF000000\\
\hline
B2 & I & 0x00100710 & 0x00000000 & 0xEEEE7777\\
\hline
B3 & S & 0x00100718 & 0x00000000 & 0x7FAABB11\\
\hline


\hline
\multicolumn{5}{|c|}{\textbf{Processor P2}}\\
\hline
Block & State & Label & \multicolumn{2}{c|}{Data}\\
\hline
\hline

B0 & S & 0x00100720 & 0x00000000 & 0x1111AAAA\\
\hline
B1 & {\color{red}M} & 0x00100708 & {\color{red}0xFFFFFFFF} & 0X00001234\\
\hline
B2 & I & 0x00100710 & 0x00000000 & 0x7FAABB11\\
\hline
B3 & I & 0x00100718 & 0x00001234 & 0x1111AABB\\
\hline


\end{tabular}

\caption{Cache contents for Section 1, exercise~\ref{ex:m5-01:msi-01}.}
\label{tab:ex:m5-01:msi-01:1}

\end{table}

The states in the different processors are shown in the Table~\ref{tab:ex:m5-01:msi-01:2}.

\begin{table}[htbp]

\begin{tabular}{|l|l|l|l|l|l|l|}

\hline
Processor 	& Block 	& State 	& 		&  		& \multicolumn{2}{c|}{}\\
		&		& Previous state	& New state	& Label	& \multicolumn{2}{c|}{Data}\\
\hline
\hline
P2 & B1 & I & M & 0X00100708 & 0XFFFFFFFF & 0X00001234\\
\hline
P0 & B1 & S & I & 0X00100708 & 0X00000000 & 0X00001234\\
\hline

\end{tabular}

\caption{State transitions for section 1, exercise~\ref{ex:m5-01:msi-01}.}
\label{tab:ex:m5-01:msi-01:2}
\end{table}

\paragraph{Section 2}

A read is perfomed in P2's cache block B1 that is in shared state. It is a cache hit the returns the value 0x00000000

\paragraph{Section 3}

A read miss is produced in P2's cache for block B3 that is invalid (I). The read miss is placed on the bus, the block is adquired and the state changes to shared (S).

The processor P0, has this block in invalid State and ignores the read miss.

The processor P1, has this Block in shared State (S) and it continues in this State.

The value read is 0x00000000

The States in the different processors are shown in the Table~\ref{tab:ex:m5-01:msi-01:3}.

\begin{table}[htbp]

\begin{tabular}{|l|l|l|l|l|}

\hline
\multicolumn{5}{|c|}{\textbf{Processor P0}}\\
\hline
Block & State & Label & \multicolumn{2}{c|}{Data}\\
\hline
\hline

B0 & I & 0x00100700 & 0x00000000 & 0x7FAABB11\\
\hline
B1 & S & 0x00100708 & 0x00000000 & 0x00001234\\
\hline
B2 & M & 0x00100710 & 0x00000000 & 0x0077AABB\\
\hline
B3 & I & 0x00100718 & 0x00000000 & 0x7FAABB11\\
\hline

\hline
\multicolumn{5}{|c|}{\textbf{Processor P1}}\\
\hline
Block & State & Label & \multicolumn{2}{c|}{Data}\\
\hline
\hline

B0 & I & 0x00100700 & 0x00000000 & 0x7FAABB11\\
\hline
B1 & M & 0x00100728 & 0x00000000 & 0xFF000000\\
\hline
B2 & I & 0x00100710 & 0x00000000 & 0xEEEE7777\\
\hline
B3 & S & 0x00100718 & 0x00000000 & 0x7FAABB11\\
\hline


\hline
\multicolumn{5}{|c|}{\textbf{Processor P2}}\\
\hline
Block & State & Label & \multicolumn{2}{c|}{Data}\\
\hline
\hline

B0 & S & 0x00100720 & 0x00000000 & 0x1111AAAA\\
\hline
B1 & S & 0x00100708 & 0x00000000 & 0X00001234\\
\hline
B2 & I & 0x00100710 & 0x00000000 & 0x7FAABB11\\
\hline
B3 & S & 0x00100718 & {\color{red}0x00000000} & {\color{red}0x7FAABB11}\\
\hline

\end{tabular}

\caption{States for Section 3, exercise~\ref{ex:m5-01:msi-01}.}
\label{tab:ex:m5-01:msi-01:3}

\end{table}

The state transitions are shown in Table~\ref{tab:ex:m5-01:msi-01:4}.

\begin{table}[htbp]

\begin{tabular}{|l|l|l|l|l|l|l|}

\hline
Processor 	& Block 	& State 	& 		&  		& \multicolumn{2}{c|}{}\\
		&		& previous state	& New state 	& Label	& \multicolumn{2}{c|}{Data}\\
\hline
\hline

P2 & B3 & I & S & 0X00100718 & 0X00000000 & 0X7FAABB11\\
\hline

\end{tabular}

\caption{State transitions for section 3, exercise~\ref{ex:m5-01:msi-01}.}
\label{tab:ex:m5-01:msi-01:4}
\end{table}
