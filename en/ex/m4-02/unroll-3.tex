\begin{acexercise}
%May 2012 exam.
\end{acexercise}

Given the following code section:

\lstinputlisting[language=generalasm2]{int/ex/m4-02/unroll-3-ex.asm}

Consider that it runs in a machine with additional latencies between
instructions as expressed in Table~\ref{tab:ex-may-2012:lat}.

\begin{table}[htbp]
\caption{Additional latencies}
\label{tab:ex-may-2012:lat}

\begin{tabular}{|p{.4\textwidth}|p{.4\textwidth}|p{.1\textwidth}|}

\hline
Instruction producing the result (previous) &
Instruction using the result (subsequent) &
Latency
\\
\hline
\hline

FP ALU operation &
FP ALU operation &
\multicolumn{1}{r|}{5}
\\
\hline

FP ALU operation &
Load/store double &
\multicolumn{1}{r|}{4}
\\
\hline

FP ALU operation &
Branch instruction &
\multicolumn{1}{r|}{4}
\\
\hline

Load double &
FP ALU operation &
\multicolumn{1}{r|}{2}
\\
\hline

Load double &
Load double &
\multicolumn{1}{r|}{1}
\\
\hline

Store double &
FP ALU operation &
\multicolumn{1}{r|}{2}
\\
\hline

\end{tabular}

\end{table}

The \emph{branch} instruction has a latency of one cycle and no 
\emph{delay slot}. 

Besides, assume that the machine is able to issue one instruction per cycle,
except waiting due to stalls and that the processor has a pipeline with a
single data path.

\begin{enumerate}

  \item Identify all the data dependencies.

  \item Determine the total number of cycles needed to run the complete section of code.

  \item Modify the code to reduce stalls through the loop scheduling technique. Determine
        the obtained speedup versus to the non-scheduled version.

  \item Modify the code performing a loop unrolling with two loop iterations. Determine
        the obtained speedup versus the non-scheduled version.

\end{enumerate}


\begin{acsolution}
\end{acsolution}

\paragraph{Section 1}

The following dependencies are produced:

\begin{itemize}
\item \asmlabel{I4} $\rightarrow$ \asmlabel{I2} (\asmreg{f0}: RAW), 
      \asmlabel{I4} $\rightarrow$ \asmlabel{I3} (\asmreg{f2}: RAW)
\item \asmlabel{I5} $\rightarrow$ \asmlabel{I4} (\asmreg{f4}: RAW)
\item \asmlabel{I6} $\rightarrow$ \asmlabel{I1} (\asmreg{r1}: WAR), 
      \asmlabel{I6} $\rightarrow$ \asmlabel{I2} (\asmreg{r1}: WAR), 
      \asmlabel{I6} $\rightarrow$ \asmlabel{I5} (\asmreg{r1}: WAR)
\item \asmlabel{I7} $\rightarrow$ \asmlabel{I3} (\asmreg{r2}: WAR)
\item \asmlabel{I8} $\rightarrow$ \asmlabel{I6} (\asmreg{r1}: RAW), 
      \asmlabel{I8} $\rightarrow$ \asmlabel{I1} (\asmreg{r3}: WAR)
\end{itemize}


\paragraph{Section 2}

The following are the stalls in the execution:

\input{int/ex/m4-02/unroll-3-lst1.asm}

A cycle is required for the initiation code. Each iteration needs 13 cycles. As the loop is executed 5 times, we have a total of:

\[
1+5 \cdot 13 = 66 \; \textrm{cycles}
\]

Because there is no delay slot it is not necessary to add a stall cycle after the branch instruction (\asminst{BLE}).

\paragraph{Section 3}

Next, the modified code:

\input{int/ex/m4-02/unroll-3-lst2.asm}

A total time of $1 + 5 \cdot 11 = 56$. Again, we need a cycle for the initiation code. Each iteration needs $13$ cycles. Because there is no delay slot, it is not necessary to add a stall after the jump instruction (\asminst{BLE}).

\[
\textrm{Speedup} = 66/56 = 1.17
\]

\paragraph{Section 4}

Next, the modified code:

\input{int/ex/m4-02/unroll-3-lst3.asm}

In total, we now have 2 iterations inside the loop. In addition, the fifth iteration of the original loop is now done at the end. The time required is:
\[
1 + 2 \cdot 23 + 10 = 57
\]

If you also re-schedule the instruction, you can have:

\input{int/ex/m4-02/unroll-3-lst4.asm}

The new time is:

\[
1 + 2 \cdot 13 + 10 = 37 \; \textrm{cycles}
\]

\clearpage
