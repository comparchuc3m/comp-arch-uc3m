\begin{acexercise}
With contributions from: Alejandro Calderón.
\end{acexercise}

A RISC-V like processor has an integer register file with the following 32 registers.

\begin{itemize}
  \item \asmreg{zero}: Always zero.
  \item \asmreg{t0}--\asmreg{t6}: Temporal registers.
  \item \asmreg{s0}--\asmreg{s11}: Saved registers.
  \item \asmreg{a0}--\asmreg{a1}: Arguments / result.
  \item \asmreg{a2}--\asmreg{a7}: Arguments.
  \item \asmreg{ra}: Regurn address.
  \item \asmreg{sp}: Stack pointer.
  \item \asmreg{gp}: Global pointer.
  \item \asmreg{tp}: Thread pointer.
\end{itemize}

Additionally, it has a floating point register file.
All registers can be used both as single or double precission.

\begin{itemize}
  \item \asmreg{ft0}--\asmreg{ft11}: Temporal registers.
  \item \asmreg{fs0}--\asmreg{fs11}: Saved registers.
  \item \asmreg{fa0}--\asmreg{fa1}: Arguments / result.
  \item \asmreg{fa2}--\asmreg{fa7}: Arguments.
\end{itemize}

Assume that in this processor the latencies between instructions are the ones
give in Table~\ref{tab:unroll-1:latencies}. 
There are no other stalls that those derived from data dependencies latencies
and that processor may fetch one instruction in every clock cycle.

\begin{table}[htb]
\begin{tabular}{|l|l|c|}
\hline
\textbf{Producer instruction} &
\textbf{Consumer instruction} &
\textbf{Latency}
\\
\hline

FP Operation & FP Operation & 6\\
\hline

FP Operation & Memory write & 3\\
\hline

Memory read & FP Operation & 2\\
\hline

Memory read & Memory write & 0\\
\hline

\end{tabular}
\caption{Latencies between instructions}
\label{tab:unroll-1:latencies}
\end{table}

In this processor, the following code fragment is executed.

\lstinputlisting[language=generalasm2]{int/ex/m4-02/unroll-1-ex.asm}

In this fragment registers \asmreg{t1}, \asmreg{t2}, and \asmreg{t3} 
contain start addresses for three arrays of double precission vaues.
Register \asmreg{t4} contains value \asmreg{t1}\cppid{+4000}.

Answer the following questions:

\begin{enumerate}

  \item Determine how many clock cycles are required to execute
        all loop iterations.

  \item Determine how many clock cycles are required to execute all loop iterations,
        when the compiler performs instruction scheduling.
        Calculate the speedup over the initial version.

  \item Determine how many clock cycles are required to execute all loop iterations,
        when the compiler performs loop unrolling for every two iterations and instruction scheduling.
        Calculate the speedup over the initial version.

  \item Determine how many clock cycles are required to execute all loop iterations,
        when the compiler performs loop unrolling for every four iterations and instruction scheduling.
        Calculate the speedup over the initial version.

\end{enumerate}

\begin{acsolution}\end{acsolution}

\paragraph{Initial execution}. 
The give code traverses a \cppid{4000} bytes space.
As each iteration advances \cppid{8} positions, 
a total of $\frac{4000}{8} = 500$ iterations are executed.

Running the code, the following stalls are identfied:

\lstinputlisting[language=generalasm2]{int/ex/m4-02/unroll-1-stalls.asm}

In total, 13 cycles per iteration are needed.
Thus, the total number of cycles is:

\[
cycles = 13 \times 500 = 6500
\]

\paragraph{Instruction scheduling}.
To schedule instructions, the compiler tries to make use of the unused stall cycles.

\lstinputlisting[language=generalasm2]{int/ex/m4-02/unroll-1-sched.asm}

In total, 10 cycles per iteration are needed.
Thus, the total number of cycles is:

\[
cycles = 10 \times 500 = 5000
\]
\[
S = \frac{6500}{5000} = 1.3
\]

\paragraph{Loop unrolling with factor 2}.
Starting from the non-scheduled fragment, a first version
of the unrolled loop is obtained.

\lstinputlisting[language=generalasm2]{int/ex/m4-02/unroll-1-unrollx2.asm}

In this first version, the loop body is repeated adjusting offsets
and making use of different registers from the register file when approppriate.

In total, 22 cycles per every two original iterations are needed.
Considering that the number of iterations in the new loop is now 250,
the total number of cycles is:

\[
cycles = 22 \times 250 = 5500
\]

Scheduling those instructions, we get:

\lstinputlisting[language=generalasm2]{int/ex/m4-02/unroll-1-unrollx2-sched.asm}

In total, 12 cycles per every two original iterations are needed.
Considering that the number of iterations in the new loop is 250,
the total number of cycles is:

\[
cycles = 12 \times 250 = 3000
\]
\[
S = \frac{6500}{3000} = 2.17
\]

\paragraph{Loop unrolling with factor 4}.
Starting from the non-scheduled fragment, a first version
of the unrolled loop is obtained.

\lstinputlisting[language=generalasm2]{int/ex/m4-02/unroll-1-unrollx4.asm}

In this first version, the loop body is repeated adjusting offsets
and making use of different registers from the register file when approppriate.

In total, 40 cycles per every four original iterations are needed.
Considering that the number of iterations in the new loop is now 125,
the total number of cycles is:

\[
cycles = 40 \times 125 = 5000
\]

Scheduling those instructions, we get:

\lstinputlisting[language=generalasm2]{int/ex/m4-02/unroll-1-unrollx4-sched.asm}

In total, 20 cycles per every four original iterations are needed.
Considering that the number of iterations in the new loop is 125,
the total number of cycles is:

\[
cycles = 20 \times 125 = 2500
\]
\[
S = \frac{6500}{2500} = 2.6
\]
