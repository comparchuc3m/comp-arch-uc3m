\newpage
\begin{acexercise}\end{acexercise}

Given a processor with architecture \textbf{Intel P6 or later}, in which
there are two variables (\cppid{z} and \cppid{t}) that initially have the value
\cppid{42}. Two threads run concurrently.


Thread 1 executes:
\begin{lstlisting}[language={[x86masm]Assembler},basicstyle=\normalsize]
mov [_z], 1
mov r1, [_z]
mov r2, [_t]
\end{lstlisting}

Thread 2 executes:
\begin{lstlisting}[language={[x86masm]Assembler},basicstyle=\normalsize]
mov [_t], 1
mov r3, [_t]
mov r4, [_z]
\end{lstlisting}

Is it possible that at the end of execution of thread 2 the records \asmreg{r2} and \asmreg{r4}
have both the value of \asmlabel{42}? Justify your answer.

\begin{acsolution}\end{acsolution}

It is possible because writes \textbf{can be perceived in a different order
for each processor}. In this way, in thread 1, you can have \cppid{r1=1} and
\cppid{r2=42}, while in thread 2, you can have \cppid{r3=1} and
\cppid{r4=42}.
