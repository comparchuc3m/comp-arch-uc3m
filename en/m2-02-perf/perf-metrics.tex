\subsection{Performance metrics}

\begin{frame}[t]{Execution speed}
\begin{itemize}
  \item What does it mean that computer \textmark{A} 
        is faster than computer \textmark{B}?

    \mode<presentation>{\pause\vfill}
    \begin{itemize}
      \item \textgood{Desktop}.
        \begin{itemize}
          \item My program runs in less time.
          \item I want to decrease execution time.
        \end{itemize}

    \mode<presentation>{\pause\vfill}
      \item \textgood{Website admin}.
        \begin{itemize}
          \item I can process more transactions per hour.
          \item I want to increase throughput.
        \end{itemize}
    \end{itemize}
\end{itemize}
\end{frame}

\begin{frame}[t]{Performance and execution time}
\begin{itemize}
  \item Performance $P(x)$ is a metric, inverse to
        execution time $T(x)$.
\end{itemize}

\begin{columns}
\begin{column}{.4\textwidth}
\begin{block}{Performance}
\begin{math}
P(x) = \frac{1}{T(x)}
\end{math}
\end{block}
\end{column}
\begin{column}{.6\textwidth}
\begin{itemize}
  \item High Performance $\rightarrow$ Low execution time.
\end{itemize}
\end{column}
\end{columns}

\mode<presentation>{\pause\vfill}
\begin{itemize}
  \item $x$ runs $n$ times faster than $Y$.
\end{itemize}
\begin{block}{Speedup}
\begin{math}
n=\frac{T(x)}{T(y)}=
\frac{
\frac{1}{P(x)}
}{
\frac{1}{P(y)}
}
=
\frac{P(y)}{P(x)}
\end{math}
\end{block}
\end{frame}

\begin{frame}[t]{Metrics}
\begin{itemize}
  \item The \alert{only} reliable metric for comparing computer performance
        is the execution of \textmark{real programs}.
    \begin{itemize}
      \item Any other metric is error-prone.
      \item Any alternative other than real programs is error-prone.
    \end{itemize}

  \mode<presentation>{\pause\vfill}
  \item \textgood{Execution time}.
    \begin{itemize}
      \item \textbf{Response time}: 
            Total elapsed time.
      \item \textbf{Perceived by the user}:
            CPU time: Time the CPU has been busy.
    \end{itemize}
\end{itemize}
\end{frame}

