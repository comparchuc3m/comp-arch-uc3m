\section{Atomic types}

\begin{frame}[t]{Atomic operations}
\begin{itemize}
\item They are \textmark{indivisible operations}.
    \begin{itemize}

      \mode<presentation>{\vfill\pause}
      \item If a thread performs an \textmark{atomic read} from a variable and  
            other thread performs an \textmark{atomic write } on the same variable and there is no
            \textmark{more threads accessing}:
        \begin{itemize}
          \item The read returns the \textmark{previous value} to the write or the \textmark{written value}.
        \end{itemize}

      \mode<presentation>{\vfill\pause}
      \item If any of the operations (read or write) is \textmark{non atomic} 
            the \textbad{behavior is not defined}.
        \begin{itemize}
          \item A value can be obtained that is not the previous or the subsequent one.
        \end{itemize}
    \end{itemize}
\end{itemize}
\end{frame}

\begin{frame}[t]{Atomic types}
\begin{itemize}
  \item A generic type \cppid{atomic<T>} allows to define atomic variables for type \cppid{T}, where \cppid{T} is:
    \begin{itemize}
      \item An integral type.
      \item A pointer type.
      \item Type \cppid{bool}.
      \item Limited interface real number types (\cppid{float}, \cppid{double}).
      \item Also available for user defined types fulfilling some constraints.
    \end{itemize}

  \mode<presentation>{\vfill\pause}
  \item All atomic types have a member \cppid{is\_lock\_free()}.
    \begin{itemize}
      \item Determine if their implementation is \textmark{lock-free}.
    \end{itemize}

  \mode<presentation>{\vfill\pause}
  \item Additionally there is a type \cppid{atomic\_flag}:
    \begin{itemize}
      \item The only type that is guaranteed to be \textmark{lock-free}.
    \end{itemize}
\end{itemize}
\end{frame}

\begin{frame}[t]{Operations on atomic types}
\begin{itemize}
  \item Operations on atomics may optionally specify a memory order.
    \begin{itemize}
      \item By default \cppid{memory\_order\_seq\_cst}.
    \end{itemize}
  \item Store operations:
    \begin{itemize}
      \item \cppid{memory\_order\_relaxed}, \cppid{memory\_order\_release}, \cppid{memory\_order\_seq\_cst}.
    \end{itemize}
  \item Read operations:
    \begin{itemize}
      \item \cppid{memory\_order\_relaxed}, \cppid{memory\_order\_consume}, \cppid{memory\_order\_acquire}, \cppid{memory\_order\_seq\_cst}
    \end{itemize}
  \item Read-modify-write operations:
    \begin{itemize}
      \item \cppid{memory\_order\_relaxed}, \cppid{memory\_order\_consume}, \cppid{memory\_order\_acquire}, \cppid{memory\_order\_release}, \cppid{memory\_order\_acq\_rel}, \cppid{memory\_order\_seq\_cst}.
    \end{itemize}
\end{itemize}
\end{frame}

\begin{frame}[t,fragile]{\texttt{\textbf{atomic\_flag}}}
\begin{itemize}
  \item \textmark{Most simple possible} atomic type.
    \begin{itemize}
      \item \textgood{Two possible states}: \textmark{enabled} o \textmark{disabled}.
      \item It is always lock-free.

      \mode<presentation>{\vfill\pause}
      \item Always must be explicitly initiated to disabled.
\begin{lstlisting}
std::atomic_flag f1 = ATOMIC_FLAG_INIT;
\end{lstlisting}

      \mode<presentation>{\vfill\pause}
      \item \textgood{Operations}:
        \begin{itemize}
          \item \textmark{Disable}: 
\begin{lstlisting}
f1.clear();
\end{lstlisting}
          \item \textmark{Enable and check} previous value: 
\begin{lstlisting}
f1.test_and_set();
\end{lstlisting}
        \end{itemize}
      \item May provide memory order for operation.
    \end{itemize}
\end{itemize}
\end{frame}

\begin{frame}[t,fragile]{Example: A \emph{spin lock}}
\begin{itemize}
  \item Lock not using OS services.
    \begin{itemize}
      \item Useful for very short lockings when you desire to avoid context switching problems.
    \end{itemize}
\end{itemize}
\begin{block}{spin lock mutex}
\begin{lstlisting}[basicstyle=\tiny]
class spinlock_mutex {
private:
  std::atomic_flag f;
public:
  spinlock_mutex() : f{ATOMIC_FLAG_INIT} {}

  void lock() {
    while (f.test_and_set()) {}
  }
  void unlock() {
    flag.clear();
  }
};
\end{lstlisting}
\end{block}
\end{frame}

\begin{frame}[t,fragile]{\texttt{\textbf{atomic\_bool}}}
\begin{itemize}
  \item More operations than \cppid{atomic\_flag}.
    \item Can be initiated and assigned with \cppid{bool}s.
    \item Cannot be copied from another \cppid{atomic<bool>}.
    \item Modification: \cppid{a.store(order)}
    \item Query: \cppid{a.exchange(b, order)}
    \item Automatic conversion to \cppid{bool} (seq. consistency): \cppid{a.load(order)}.
\end{itemize}
\begin{block}{Example}
\begin{lstlisting}
std::atomic<bool> a;
bool x = a.load(std::memory_order_acquire);
a.store(true);
x = a.exchange(false, std::memory_order_acq_rel);
\end{lstlisting}
\end{block}
\end{frame}

\begin{frame}[t,fragile]{Compare and exchange}
\begin{itemize}
  \item Compares atomic value with an \textmark{expected} value.
    \begin{itemize}
      \item If both are equal, the \textmark{desired} value is stored in the atomic.
      \item If not equal, atomic is left unmodified.
      \item It always returns success/failure indication.
    \end{itemize}

  \mode<presentation>{\vfill\pause}
  \item Two versions:
    \begin{enumerate}
      \item \cppid{a.compare\_exchange\_weak(e,d)}:
        \begin{itemize}
          \item Allows spurious failures (context switch) in some architectures.
          \item May behave as if \cppid{*}\cppkey{this}\cppid{!=e} even if they are equal.
        \end{itemize}
      \item \cppid{a.compare\_exchange\_strong(e,d)}:
        \begin{itemize}
          \item Does not allow for spurious failures.
        \end{itemize}
    \end{enumerate}
%  \item Consecuencia:
%    \begin{itemize}
%      \item Cuando se usa en un bucle, la versión \emph{weak} da mejor rendimiento en algunas plataformas.
%      \item Cuando \emph{weak} necesita un bucle pero \emph{strong} no, \emph{strong} es mejor.
%    \end{itemize}
\end{itemize}
\end{frame}

\begin{frame}[t,fragile]{\texttt{\textbf{atomic\_address}}}
\begin{itemize}
  \item Atomic access to a memory address.
  \item Cannot be copied.
  \item Can copy a (\cppkey{void*}) pointer.
  \item Interface similar to \cppid{atomic<bool>}:
    \begin{itemize}
      \item \cppid{is\_lock\_free()}, \cppid{load()}, \cppid{store()}, \cppid{exchange()},
            \cppid{compare\_exchange\_weak()}, \cppid{compare\_exchange\_strong()}.
    \end{itemize}
  \item Additional operations.
    \begin{itemize}
      \item \cppid{fetch\_add()}, \cppid{fetch\_sub()}.
        \begin{itemize}
          \item Allow for memory ordering specification.
          \item Return value previous to change.
        \end{itemize}
    \end{itemize}
      \item \cppid{+=}, \cppid{-=}.
        \begin{itemize}
          \item Return the value after the change.
          \item All operations allow byte arithmetic.
        \end{itemize}
      \item Other arithmetics with \cppid{atomic<T*>}.
\end{itemize}
\end{frame}

\begin{frame}[t,fragile]{\texttt{\textbf{atomic<integral>}}}
\begin{itemize}
  \item Can be applied to all integral types.
  \item General operations:
    \begin{itemize}
      \item \cppid{is\_lock\_free()}, \cppid{load()}, \cppid{store()}, \cppid{exchange()}, 
            \cppid{compare\_exchange\_weak()}, \cppid{compare\_exchange\_strong()}.
    \end{itemize}
  \item Arithmetic operations.
    \begin{itemize}
      \item \cppid{fetch\_add()}, \cppid{fetch\_sub()}, \cppid{fetch\_and()}, \cppid{fetch\_or()}, \cppid{fetch\_xor()}.
      \item \cppid{+=}, \cppid{-=}, \cppid{\&=}, \cppid{|=}, \cppid{\^{}=}.
      \item \cppid{++x}, \cppid{x++}, \cppid{--x}, \cppid{x--}
      \item There are no other arithmetic operations (\cppid{*}, \cppid{/}, \cppid{\%}).
    \end{itemize}
\end{itemize}
\end{frame}

