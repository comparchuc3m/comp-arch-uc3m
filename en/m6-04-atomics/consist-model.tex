\section{Consistency models}

\begin{frame}[t]{Sequential consistency}
\begin{itemize}
  \item \cppid{memory\_order\_seq\_cst}.
  \item The program is consistent with a \textmark{sequential view}.
  \item If all the operations on atomics are \textmark{sequentially consistent}, 
        multi-threaded program behavior is as if all the operations would be
        performed in some particular order in a single thread.
  \item There cannot be reorderings.
  \item It is the simplest model to reason about.
  \item It is the most costly model in terms of performance.
\end{itemize}
\end{frame}

\begin{frame}[fragile]
\begin{columns}

\begin{column}{.6\textwidth}
\begin{block}{Access}
\begin{lstlisting}[basicstyle=\tiny]
std::atomic<bool> x, y;
std::atomic<int> z;

void f() {
  x.store(true, std::memory_order_seq_cst);
}

void g() {
  y.store(true, std::memory_order_seq_cst);
}

void h() {
  while (!x.load(std::memory_order_seq_cst)) {}
  if (y.load(std::memory_order_seq_cst))  ++ z;
}

void i() {
  while (!y.load(std::memory_order_seq_cst)) {}
  if (x.load(std::memory_order_seq_cst)) ++z;
}
\end{lstlisting}
\end{block}
\end{column}

\pause

\begin{column}{.4\textwidth}
\begin{block}{Threads launching}
\begin{lstlisting}[basicstyle=\tiny]
int main() {
  x = false;
  y = false;
  z = 0;

  std::thread t1{f};
  std::thread t2{g};
  std::thread t3{h};
  std::thread t4{i};

  t1.join();
  t2.join();
  t3.join();
  t4.join();

  assert(z.load() !=0);

  return 0;
}
\end{lstlisting}
\end{block}
\end{column}

\end{columns}
\end{frame}

\begin{frame}[t]{Sequential consistency: Analysis}
\input{en/m6-04-atomics/seqconsist-ej2a.tkz}
\end{frame}

\begin{frame}[t]{Sequential consistency: Analysis}
\input{en/m6-04-atomics/seqconsist-ej2b.tkz}
\end{frame}

\begin{frame}[t,fragile]{Non-sequentially consistent orders}
\begin{itemize}
\item There is no \textmark{global order of events}.
  \begin{itemize}
    \item Each thread may have \textbad{a different view}.
      \begin{itemize}
        \item Threads might not agree on the same order of events.
      \end{itemize}

    \pause
    \item But, \ldots
      \begin{itemize}
        \item \textgood{All threads must agree in the modifications order for each variable}.
      \end{itemize}
  \end{itemize}
  \mode<presentation>{\vfill\pause}
\item \textgood{Alternatives}:
  \begin{itemize}
    \item \textmark{relaxed} ordering.
    \item \textmark{release/acquire} ordering.
  \end{itemize}
\end{itemize}
\end{frame}

\begin{frame}[t]{Relaxed ordering}
\begin{itemize}
\item \cppid{memory\_order\_relaxed}
  \item Relaxed operations on atomics \textmark{do not participate} in \textgood{synchronizes-with} relationship.

  \mode<presentation>{\vfill\pause}
  \item Operations on same variable in the same thread \textmark{do fulfill} \textgood{happens-before} relationship.
    \begin{itemize}
      \mode<presentation>{\vfill}
      \item Accesses to an atomic variable within the same thread \textbad{cannot be reordered}.
      \mode<presentation>{\vfill}
      \item Once a thread has seen a value from variable \textbad{it cannot see} an older value of that variable.
    \end{itemize}
\end{itemize}
\end{frame}

\begin{frame}[fragile]{Example}
\begin{columns}

\begin{column}{.6\textwidth}
\begin{block}{Data access}
\begin{lstlisting}[basicstyle=\tiny]
std::atomic<bool> x, y; std::atomic<int> z;

void f() {
  x.store(true, std::memory_order_relaxed);
  y.store(true, std::memory_order_relaxed);
}
void g() {
  while (!y.load(std::memory_order_relaxed)) {}
  if (x.load(std::memory_order_relaxed)) { ++z; }
}

int main() {
  x=false; y=false; z=0;
  std::thread t1{f}; std::thread t2{g};
  t1.join(); t2.join();
  return 0;
}

\end{lstlisting}
\end{block}
\end{column}

\begin{column}{.4\textwidth}
\input{en/m6-04-atomics/relconsist-ej.tkz}
\end{column}

\end{columns}
\end{frame}


\begin{frame}[t]{Release/acquire ordering}
\begin{itemize}
  \item \cppid{memory\_order\_acquire}, \cppid{memory\_order\_release}, \cppid{memory\_order\_acq\_rel}.
  
  \mode<presentation>{\vfill}
  \item \textmark{Intermediate} level of synchronization.
  
  \mode<presentation>{\vfill\pause}
  \item A \textgood{release} operation \textmark{writing a value} 
        \textgood{synchronizes-with} an \textgood{acquire} operation \textmark{reading that value}.
  
  \mode<presentation>{\vfill}
  \item \textgood{Impact}:
    \begin{itemize}
      \item \textbad{Different threads may see different orders}.
      \item \textbad{Not all orders are possible}.
    \end{itemize}
\end{itemize}
\end{frame}


\begin{frame}[t,fragile]
\begin{columns}

\begin{column}{.6\textwidth}
\begin{block}{Access}
\begin{lstlisting}[basicstyle=\tiny]
std::atomic<bool> x, y;
std::atomic<int> z;

void f() {
  x.store(true, std::memory_order_release);
}

void g() {
  y.store(true, std::memory_order_release);
}

void h() {
  while (!x.load(std::memory_order_acquire)) {}
  if (y.load(std::memory_order_acquire))  ++ z;
}

void i() {
  while (!y.load(std::memory_order_acquire)) {}
  if (x.load(std::memory_order_acquire)) ++z;
}
\end{lstlisting}
\end{block}
\end{column}

\begin{column}{.4\textwidth}
\begin{block}{Threads launching}
\begin{lstlisting}[basicstyle=\tiny]
int main() {
  x = false;
  y = false;
  z = 0;

  std::thread t1{f};
  std::thread t2{g};
  std::thread t3{h};
  std::thread t4{i};

  t1.join();
  t2.join();
  t3.join();
  t4.join();

  assert(z.load() !=0);
  return 0;
}
\end{lstlisting}
\end{block}
\end{column}

\end{columns}
\end{frame}

\begin{frame}[t]{Analysis}
\input{en/m6-04-atomics/relacq-consist-ej.tkz}

\mode<presentation>{\vfill\pause}
\begin{itemize}
  \item Multiple orders are possible as there is no relationship
         \emph{acquire} $\rightarrow$ \emph{release} .
\end{itemize}
\end{frame}

\begin{frame}[fragile]{Combining orderings}
\begin{itemize}
\item \textmark{Equivalent} effect to \textgood{sequential consistency} 
      with \textmark{lower cost}.
\end{itemize}
\begin{columns}

\begin{column}{.5\textwidth}
\begin{block}{Access}
\begin{lstlisting}[basicstyle=\tiny]
std::atomic<bool> x, y; std::atomic<int> z;

void f() {
  x.store(true, std::memory_order_relaxed);
  y.store(true, std::memory_order_release);
}
void g() {
  while (!y.load(std::memory_order_acquire)) {}
  if (x.load(std::memory_order_relaxed)) ++z;
}
int main() {
  x = false; y = false; z = 0;
  std::thread t1{f}; std::thread t2{g};
  t1.join(); t2.join();
  assert(z.load() !=0);

  return 0;
}
\end{lstlisting}
\end{block}
\end{column}

\begin{column}{.5\textwidth}
\input{en/m6-04-atomics/comb-consist-ej.tkz}
\end{column}

\end{columns}
\end{frame}

