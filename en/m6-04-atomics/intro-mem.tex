\section{Memory model}

\begin{frame}[t]{C++ and memory consistency}
\begin{itemize}
  \item C++ defines its own \textmark{concurrency model} as part of the language.

  \mode<presentation>{\vfill\pause}
  \item \textgood{Goal}: 
        Avoid the need to write code in lower level languages (C, assembler, \ldots)
        to obtain better performance.
    \begin{itemize}
      \item Atomic types.
      \item Low level synchronization mechanisms.
    \end{itemize}

  \mode<presentation>{\vfill\pause}
  \item Allows to build \textmark{lock free data structures}.
\end{itemize}
\end{frame}

\begin{frame}[t]{Objects and memory locations}
\begin{itemize}
  \item \textgood{Object}: Is a storage region.
    \begin{itemize}
      \item A sequence of one or more bytes.
    \end{itemize}

  \mode<presentation>{\vfill\pause}
  \item \textgood{Memory location}: 
        Is an object of \textmark{scalar type} or a sequence of contiguous
        \textmark{bit fields}.

    \begin{itemize}
      \item \textgood{Scalar type}: 
            Arithmetic types, enumeration types, or pointer types.
      \item \textgood{Bit field}:
            Integer data member with explicit size in bits.
    \end{itemize}

  \mode<presentation>{\vfill\pause}
  \item \textmark{An object is stored in one or more memory locations}.
\end{itemize}
\end{frame}

\begin{frame}[t,fragile]{Example}
\begin{columns}[T]

\column{.4\textwidth}
\begin{itemize}
  \item Structure:
\begin{lstlisting}
struct data {
  int i;
  char c;
  int d: 10;
  int e: 16;
  double f;
};

data x;
\end{lstlisting}
\end{itemize}

\mode<presentation>{\vfill\pause}
\column{.4\textwidth}
\begin{itemize}
  \item Memory locations:
    \begin{enumerate}
      \item \cppid{x.i}.
      \item \cppid{x.c}.
      \item \cppid{x.d}, \cppid{x.e}.
      \item \cppid{x.f}.
    \end{enumerate}
\end{itemize}

\end{columns}
\end{frame}

\begin{frame}[t]{Rules}
\begin{itemize}
  \item Two threads may access to \textmark{different memory locations} 
        simultaneously.

  \mode<presentation>{\vfill\pause}
  \item Two threads may access to the \textmark{same memory locations}
        simultaneously if both accesses are for \textmark{reading}.

  \mode<presentation>{\vfill\pause}
  \item If two threads try to access simultaneously to the \textmark{same memory location} 
        and any access is a \textmark{write}, there is a 
        \textmark{potential race condition}.
    \begin{itemize}
      \item Depends on whether an \textmark{ordering between both accesses} is stablished.
    \end{itemize}
\end{itemize}
\end{frame}

\begin{frame}[t]{Ordering and race conditions}
\begin{itemize}
  \item \textgood{Classic solution}: Use \textmark{synchronization} mechanisms.
    \begin{itemize}
      \item Allow to guarantee \textmark{mutual exclusion}.
      \item Based on OS $\rightarrow$ Might be costly.
    \end{itemize}

  \mode<presentation>{\vfill\pause}
  \item \textbf{Alternative}: Use \textmark{atomic operations} to ensure 
        \textmark{ordering}.
 
  \mode<presentation>{\vfill\pause}
    \begin{itemize}
      \item If \textmark{ordering between two accesses} to a memory location is not established,
      \item some of the accesses \textmark{is not atomic}, 
      \item and at least one of the accesses is a \textmark{write}, 
      \item those are a \textbad{data race} and  
        \textbad{program behavior is not defined}.
    \end{itemize}
\end{itemize}
\end{frame}

\begin{frame}[t]{Modification order}
\begin{itemize}
  \item \textgood{Modification order}: Sequence of writes on an object.
    \begin{itemize}
      \item If two threads see different modification orders on an object there is a \textbad{data race}.
      \item Modifications do not need to be visible in the same instant in all threads.
    \end{itemize}

  \mode<presentation>{\vfill\pause}
  \item A subsequent read to a write on the same thread observes the written value or
        a subsequent value in its \textmark{modification order}.
\end{itemize}
\end{frame}

