\section{Compilation techniques and ILP}

\begin{frame}[t]{Taking advantage of ILP}
\begin{itemize}
  \item ILP directly applicable to basic blocks.
    \begin{itemize}
      \item \textgood{Basic block}: sequence of instructions without branching.
      \item \textmark{Typical} program in RISC processors:
        \begin{itemize}
          \item Basic block average size from 3 to 6 instructions.
          \item Low ILP exploitation within block.
        \end{itemize}
      \item Need to exploit ILP across basic blocks.
    \end{itemize}
\end{itemize}

\mode<presentation>{\vfill\pause}
\begin{columns}

  \begin{column}{.4\textwidth}
    \begin{block}{Example}
      \lstinputlisting{int/m4-02-ilp/comp-ilp-ex1.cpp}
    \end{block}
  \end{column}

  \begin{column}{.6\textwidth}
    \begin{itemize}
      \item \textgood{Loop level parallelism}.
        \begin{itemize}
          \item Can be transformed to ILP.
          \item By compiler or hardware.
        \end{itemize}
      \item \textgood{Alternative}:
        \begin{itemize}
          \item Vector instructions.
          \item SIMD instructions in processor.
        \end{itemize}
    \end{itemize}
  \end{column}
\end{columns}
\end{frame}

\begin{frame}[t]{Scheduling and loop unrolling}
\begin{itemize}
  \item \textgood{Parallelism exploitation}:
    \begin{itemize}
      \item Interleave execution of unrelated instructions.
      \item Fill stalls with instructions.
      \item Do not alter original program effects.
    \end{itemize}

\mode<presentation>{\vfill\pause}
  \item Compiler can do this with detailed knowledge of the architecture.
\end{itemize}
\end{frame}

\begin{frame}[t]{ILP exploitation}
\begin{columns}

\begin{column}{.4\textwidth}
\begin{block}{Example}
\lstinputlisting{int/m4-02-ilp/comp-ilp-ex2.cpp}
\end{block}
\end{column}

\begin{column}{.6\textwidth}
\begin{itemize}
  \item Each iteration body is independent.
\end{itemize}
\end{column}

\end{columns}

\mode<presentation>{\vfill}
\begin{block}{Latencies between instructions}
{\footnotesize
\begin{tabular}{|*{2}{p{.3\textwidth}|}p{0.25\textwidth}|}

\hline
Instruction &
Instruction &
Latency (cycles)
\\
producing result &
using result &
\\
\hline
\hline

FP ALU operation & FP ALU operation & 3\\
\hline

FP ALU operation & Store double & 2\\
\hline

Load double & FP ALU operation & 1\\
\hline

Load double & Store double & 0\\
\hline

\end{tabular}
}
\end{block}

\end{frame}

\begin{frame}[t]{Compiled code}
\begin{itemize}
  \item \asmreg{x1} $\rightarrow$ Last array element.
  \item \asmreg{f2} $\rightarrow$ Scalar \cppid{s}.
  \item \asmreg{x2} $\rightarrow$ Precomputed so that \asmlabel{8(R2)} is the first element in array.
\end{itemize}

\mode<presentation>{\vfill}
\begin{block}{Assembler code}
\lstinputlisting[language=generalasm2]{int/m4-02-ilp/comp-ilp-ex3.asm}
\end{block}

\end{frame}

\begin{frame}[t]{Stalls in execution}

\begin{columns}[T]

\begin{column}{.45\textwidth}
\begin{block}{Original}
\lstinputlisting[language=generalasm2]{int/m4-02-ilp/comp-ilp-ex4.asm}
\end{block}
\end{column}

\pause
\begin{column}{.45\textwidth}
\begin{block}{Stalls}
\lstinputlisting[language=generalasm2]{int/m4-02-ilp/comp-ilp-ex4b.asm}
\end{block}
\end{column}

\end{columns}
\end{frame}

\begin{frame}[t]{Loop scheduling}

\begin{columns}[T]

\begin{column}{.45\textwidth}
\begin{block}{Original}
\lstinputlisting[language=generalasm2]{int/m4-02-ilp/comp-ilp-ex4b.asm}
\end{block}

\begin{itemize}
  \item 8 cycles per element.
\end{itemize}
\end{column}

\pause
\begin{column}{.45\textwidth}
\begin{block}{Scheduled}
\lstinputlisting[language=generalasm2]{int/m4-02-ilp/comp-ilp-ex4c.asm}
\end{block}
\begin{itemize}
  \item 7 cycles per element.
\end{itemize}
\end{column}

\end{columns}
\end{frame}

\begin{frame}[t]{Loop unrolling}
\begin{itemize}
  \item \textgood{Idea}:
    \begin{itemize}
      \item Replicate loop body several times.
      \item Adjust termination code.
      \item Use different registers for each iteration replica to reduce dependencies.
    \end{itemize}

  \mode<presentation>{\vfill}
  \item \textmark{Effect}:
    \begin{itemize}
      \item Increase basic block length.
      \item Increase use of available ILP.
    \end{itemize}
\end{itemize}
\end{frame}

\begin{frame}[t]{Unrolling}

\begin{columns}

\begin{column}{.45\textwidth}
\begin{block}{Unrolling (x4)}
\lstinputlisting[language=generalasm2,lastline=7]{int/m4-02-ilp/loop-unroll0.asm}
\end{block}
\end{column}

\begin{column}{.45\textwidth}
\begin{block}{Unrolling (x4)}
\lstinputlisting[language=generalasm2,firstline=8]{int/m4-02-ilp/loop-unroll0.asm}
\end{block}
\end{column}

\end{columns}
\mode<presentation>{\vfill}
\begin{itemize}
  \item 4 iterations require more registers.
  \item This example assumes that array size is multiple of 4.
\end{itemize}
\end{frame}

\begin{frame}[t,shrink=10]{Stalls and unrolling}

\begin{columns}

\begin{column}{.45\textwidth}
\begin{block}{Unrolling (x4)}
\lstinputlisting[language=generalasm2,firstline=1,lastline=14]{int/m4-02-ilp/loop-unroll1.asm}
\end{block}
\end{column}

\begin{column}{.45\textwidth}
\begin{block}{Unrolling (x4)}
\lstinputlisting[language=generalasm2,firstline=15]{int/m4-02-ilp/loop-unroll1.asm}
\end{block}
\end{column}

\end{columns}
\mode<presentation>{\vfill}
\begin{itemize}
  \item 26 cycles for every 4 iterations $\rightarrow$ $6.5$ cycles per element.
\end{itemize}
\end{frame}

\begin{frame}[t]{Scheduling and unrolling}

\begin{columns}

\begin{column}{.45\textwidth}
\begin{block}{Unrolling (x4)}
\lstinputlisting[language=generalasm2]{int/m4-02-ilp/loop-unroll2.asm}
\end{block}
\end{column}

\begin{column}{.45\textwidth}
\begin{itemize}
  \item Code reorganization.
    \begin{itemize}
      \item Preserve dependencies.
      \item Semantically equivalent.
      \item \textgood{Goal}: Make use of \emph{stalls}.
    \end{itemize}
  \item Update of \asmreg{x1} at enough distance from
        \asminst{BNE}.
  \item 14 cycles for every 4 iterations $\rightarrow$ $3.5$ cycles per iteration.
\end{itemize}
\end{column}

\end{columns}

\end{frame}

\begin{frame}[t]{Limits of loop unrolling}
\begin{itemize}
  \item \textmark{Improvement} is \textgood{decreased} with each additional unrolling.
    \begin{itemize}
      \item Improvement limited to stalls removal.
      \item Overhead amortized among iterations.
    \end{itemize}

  \mode<presentation>{\vfill\pause}
  \item \textgood{Increase} in code \textmark{size}.
    \begin{itemize}
      \item May affect to instruction cache miss rate.
    \end{itemize}

  \mode<presentation>{\vfill\pause}
  \item \textgood{Pressure} on \textmark{register file}.
    \begin{itemize}
      \item May generate shortage of registers.
      \item Advantages are lost if there are not enough available registers.
    \end{itemize}

\end{itemize}
\end{frame}

