\section{Task}

\subsection{Software Development}


\subsubsection{Program versions}
\label{sec:tasks:versions}

This task consists of the development of a sequential version
of the described application using C++20.
Two executable programs will be prepared:
\cppkey{image-soa} and \cppkey{image-aos} 
implementing two strategies:

\begin{itemize}
  \item \textmark{SOA -- Structure of Arrays}:
        An image shall be represented by using three independent vectors,
        each of them with values from \cppid{0} to \cppid{255}.
  \item \textmark{AOS -- Array of Structures}:
        An image will be represente by using a single vector of values from pixel type.
        Each pixel represents a structure with three values from \cppid{0} to \cppid{255}.
\end{itemize}

\subsubsection{Components}

The following componentes shall be developed:

\begin{itemize}
  \item \cppid{common}: 
        Library with source files that are common to both versions.

  \item \cppid{soa}: 
        Library with source files that are specific to the \textmark{AOS} version.

  \item \cppid{aos}:
        Library with source files that are specific to the \textmark{AOS} version.

  \item \cppid{image-soa}: 
        Executable program with the \textmark{SOA} version.

  \item \cppid{image-aos}: 
        Executable program with the \textmark{AOS} version.

\end{itemize}

Below, the five components are described in more detail:


\paragraph{common}: Common components:

It will contain, at least, the following source files:
\begin{itemize}
  \item \cppid{progargs.cpp}: 
        Argument handling for \cppid{main} parameters. 
\end{itemize}

\paragraph{soa}: Components library for \textmark{SOA}.

It will contain, at least, the following source files:
\begin{itemize}
  \item \cppid{imagesoa.cpp}: 
        Image handling with \textmark{SOA} representation.
\end{itemize}

\paragraph{aos}: Components library for \textmark{AOS}.

It will contain, at least, the following source files:
\begin{itemize}
  \item \cppid{imageaos.cpp}: 
        Image handling with \textmark{SOA} representation.
\end{itemize}

\paragraph{image-soa}: Program with \textmark{SOA} version.

It will contain a single source file:
\begin{itemize}
  \item \cppid{imgsoa}: 
        It shall only contain a \cppid{main()} function for this version.
        It shall not include any additional function.
\end{itemize}

\paragraph{image-aos}: Program with \textmark{AOS} version.

It will contain a single source file:
\begin{itemize}
  \item \cppid{imgaos}: 
        It shall only contain a \cppid{main()} function for this version.
        It shall not include any additional function.
\end{itemize}

\subsubsection{Compiling the project}

All source files must compile without problems and shall no emit no compiler warning.
A CMake configuration file shall be included with the following options:

\begin{lstlisting}[title={CmakeLists.txt},frame=single]
cmake_minimum_required(VERSION 3.23)
project(image LANGUAGES CXX)

set(CMAKE_CXX_STANDARD 20)
set(CMAKE_CXX_STANDARD_REQUIRED ON)
set(CMAKE_CXX_EXTENSIONS_OFF)
add_compile_options(-Wall -Wextra -Werror -pedantic -pedantic-errors)

set(CMAKE_CXX_FLAGS_RELEASE "-march=native")

add_library(common file1-common.cpp file2-common.cpp file3-common.cpp)
add_library(aos file1-aos.cpp file2-aos.cpp)
add_library(soa file1-soa.cpp file2-soa.cpp)

add_executable(imgaos image-aos.cpp)
target_link_libraries(imgaos common aos)
add_executable(imgsoa image-soa.cpp)
target_link_libraries(imgsoa common soa)
\end{lstlisting}

This rules are an example. Keep in mind the following:

\begin{itemize}

  \item The library \cppkey{common} consists of all the \cppid{.cpp} files
        that are considered needed (names of choice by designers) and
        which offer common functionality to both versions of the program.

  \item The library \cppkey{aos} consists of all the \cppid{.cpp} files
        that are considered needed (names of choice by designers) and
        which offer functionality which is exclusive of the
        \textemph{array of structures} implementation.

  \item The library \cppkey{soa} consists of all the \cppid{.cpp} files
        that are considered needed (names of choice by designers) and
        which offer functionality which is exclusive of the
        \textemph{structure of arrays} implementation.

\end{itemize}



Keep in mind that all evaluations shall be performed with the compiler optimizations
enabled with CMake option
\cppid{-DCMAKE\_BUILD\_TYPE=Release}.


\textbad{Important}: 
The only allowed library is the C++ standard library.
No external library is allowed.

\subsubsection{Code quality rules}

Source code must be well structured and organized, 
as well as approppriately documented.
The \textemph{C++ Core Guidelines}
(\url{http://isocpp.github.io/CppCoreGuidelines/CppCoreGuidelines})
are recommended but not required.

In any case, the following rules must be strictly followed:

\begin{itemize}
  \item No global variables are allowed, except constants.
  \item No use of the \emph{Singleton} pattern is allowed.
  \item No array shall be passed to a function as a pointer.
  \item No function or member function shall have more than four parameters.
  \item No function shall have more than 25 lines when appropriately formatted.
  \item All parameters shall be passed to functions by value,
        by reference of by const reference.
  \item No explicit use of the following functions is allowed:
        \cppid{malloc()} or \cppid{free()} functions,
        operators \cppkey{new} or \cppkey{delete}.
  \item No macro use is allowed, except for defining include guards.
  \item No cast is allowed except 
        \cppkey{static\_cast}, \cppkey{dynamic\_cast}, 
        \cppkey{const\_cast} or \cppkey{reinterpret\_cast}.
\end{itemize}


\subsubsection{Unit tests}

A set of unit tests shall be defined and included in the submission.
You are recommended, but no trequired, to make use of GoogleTest
(\url{https://github.com/google/googletest}).

In any case, evidence of enough unit tests shall be included in the
submission.


\subsection{Performance and energy evaluation}

This task consists of a comparative performance evaluation of the 
two implementation strategies.
\cppid{image-aos} and \cppid{image-soa}.

To conduct the performance evaluation, the total execution time shall be measured.
In addition, energy use must be alse measured.
The power use must be derived.

All performance evaluations shall be performed in a node from the
\cppid{avignon} cluster.

Represent in a graphic all total execution times, energy use and power
for images with different sizes.
Evaluation images will be provided.

\textbf{The project report shall include conclusions drawn from results}.
Please, do not limit to simply describing data.
Try to find convincing explanations of those results.
