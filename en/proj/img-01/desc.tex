\section{Project description}

In this project you will develop an image processing application.

In the following sections, requirements to be fulfilled are presented.

\subsection{General overview}

You will develop an application that applies several operations
on ain image in BMP format.

For each image the following operations will be implemented:

\begin{itemize}
\item Image copy.
\item Histogram computation.
\item Grayscale transformation.
\item Gaussian blur.
\end{itemize}

The application will take the following parameters

\begin{itemize}

\item Directory with input images.
\item Directory with output images.
\item Operation to be executed.

\end{itemize}

The operation to be executed will be one of the following:

\begin{itemize}

\item \textmark{copy}: 
Does not perform any transformation. It only copies files.

\item \textmark{histo}: 
Generates a text file with the histogram for each channel (RGB).

\item \textmark{mono}: 
Converts to gray scale.

\item \textmark{gauss}: 
Applies the gaussian blur filter.

\end{itemize}

\textbad{NOTE}: 
Two different programs will be implemented: \cppid{image-aos} adn \cppid{image-soa}.
However, for simplicity, in the rest of this section the generic name \cppid{image}
is used to refere to any of those two programs.

For details on these two programs, please go to section~\ref{sec:tasks:versions}.

In any case, the application will read all the files in the input directory,
it will apply the corresponding tansformation, and it will write the 
corresponding files with the same name in the output directory.
mismo nombre en el directorio de salida.

The program must check that the number of parameters is 3:

\begin{lstlisting}[style=terminal]
$image
Wrong format:
  image in_path out_path oper
    operation: copy, histo, mono, gauss
$
\end{lstlisting}

If the third argument does not take a valid valu
(\cppid{copy}, \cppid{histo}, \cppid{mono} or \cppid{gauss}), 
an error message will presented before terminating:

\begin{lstlisting}[style=terminal]
$image indir outdir gaus
Unexpected operation:gaus
  image in_path out_path oper
    operation: copy, histo, mono, gauss
$
\end{lstlisting}

When the input directory does not exist, or it cannot be opened,
an error message will be presented before terminating:

\begin{lstlisting}[style=terminal]
$image inx outdir copy
Input path: inx
Output path: out
Cannot open directory [inx]
  image in_path out_path oper
    operation: copy, histo, mono, gauss
$
\end{lstlisting}

When the output directory does not exist, 
an error message will be presented before terminating:

\begin{lstlisting}[style=terminal]
$image-seq sobel indir outx
Input path: indir
Output path: outx
Output directory [outx] does not exist
  image in_path out_path oper
    operation: copy, histo, mono, gauss
$
\end{lstlisting}

In any other case, all files in the input directory will be processed
and results will be stored in the otuput directory.
For each processed file the time dedicated to each task will be printed.
All values will be expressed in microseconds:

\begin{lstlisting}[style=terminal]
$image-seq indir outdir gauss
Input path: indir
Output path: outdir
File: "indir/62096.bmp"(time: 43994)
  Load time: 3442
  Gauss time: 38909
  Store time: 1642
File: "indir/169012.bmp"(time: 37209)
  Load time: 2602
  Gauss time: 32950
  Store time: 1655
$
\end{lstlisting}


\subsection{BMP format}

You will be using the BMP image format.
In this format files have a 54 bytes header with the information specified in
Table~\ref{tab:bmp}.

\begin{table}[!htb]
\centering
\begin{tabular}{|l|l|l|}
\hline
\textbf{Start} & \textbf{Length} & \textbf{Description}\\
\hline
\hline
0 & 2 & Characters \cppid{'B'} and \cppid{'M'}.\\
\hline
2 & 4 & File size.\\
\hline
6 & 4 & Reserved.\\
\hline
10 & 4 & Start of image data.\\
\hline
14 & 4 & Size of bitmap header.\\
\hline
18 & 4 & Width in pixels.\\
\hline
22 & 4 & Height in pixels.\\
\hline
26 & 2 & Number of planes.\\
\hline 
28 & 2 & Point size in bits.\\
\hline
30 & 4 & Compression.\\
\hline
34 & 4 & Image size.\\
\hline
38 & 4 & Horizontal resolution.\\
\hline
42 & 4 & Vertical resolution.\\
\hline
46 & 4 & Color table size.\\
\hline
50 & 4 & Important color counter.\\
\hline
\end{tabular}
\caption{Header fields for the BMP format.}
\label{tab:bmp}
\end{table}

The following constraints need to be considered
for a file to be considered valid.

\begin{itemize}
\item The number of plains shall be \cppid{1}.
\item The size of each point shall be \cppid{24} bits.
\item The compression value shall be \cppid{0}.
\end{itemize}

Note that it is not necessary to use certain fields.

Pixels are stored starting from the position specified by the field
\emph{Start of image data}.
Each pixel is represented by 3 consecutive bytes representing the
blue, green and red blue for that pixel.
Those values are always in the range from \cppid{0} to \cppid{255}.

At the end of the pixels in a row additional padding bytes are added,
so that the next pixels row starts at the start of a word.

\subsection{Image histogram}

For every image a text file  with extension \cppid{.hst} will be generated
containing consecutively the information from the three histograms.
That is, a histogram for R channel, another histogram for the G channel,
and a third one for the B channel.

A histogram represents the absolute frequency (number of occurrences)
for each numeric value (from $0$ to $255$).

The output file will be a text file containing a total of 768 ($256 \times 3$) lines.
The first 256 lines will contain the absolute frequency values for each value of the R component.
The next 256 lines will be for the G component.
The last 256 lines will be for the B component.

\subsection{Gray scale conversion}

An image in gray scale has the same value for the three values RGB for each pixel.

The transformation is performed in four stages.

\begin{enumerate}

\item \textmark{Normalization}: Values are normalized to a real scal from \cppid{0} to \cppid{1}.

\item Normalización: Se normalizan los valores a una escala real de \cppid{0} A \cppid{1}.

\[
n_r = \frac{R}{255}
\]
\[
n_g = \frac{G}{255}
\]
\[
n_b = \frac{B}{255}
\]

\item Linear intensity transform:
For each normalized color
$(n_r, n_g, n_b)$ a new color $(c_r, c_g, c_b)$ is obtained.
The same transform is applied for the three channels.

\[
c_i = \frac{n_i}{12.92} \quad\quad \text{if} \quad n_i \leq 0.04045
\]
\[
c_i = \left(\frac{n_i + 0.055}{1.055}\right)^{2.4} \quad\quad \text{if} \quad n_i > 0.04045
\]

\item Linear transform:
A linear transform is applied to each color:
\[
c_l = 0.2126 \cdot c_r + 0.7152 \cdot c_g + 0.0722 \cdot c_b
\]

\item Gamma correction: 
\[
c_g = 12.92 \cdot c_l \quad\quad \text{if} \quad c_l \leq 0.0031308
\]
\[
c_g = 1.055 \cdot c_l^{\frac{1}{2.4}} - 0.055 \quad\quad \text{if} \quad c_l > 0.0031308
\]

\item Denormalization: A transformation is performed from scale $[0,1]$ to scale scale $[0,255]$.
\[
g = c_g \cdot 255
\]

\end{enumerate}

\textbad{NOTE}:
The final value obtained for $g$ in the three componentes (R, G, and B) of the pixel.

\subsection{Gaussian blur}

Gaussian blur has as a goal to decre
Gaussian blur reduces image noise and also image detail. It is usually
used as a previous step to edge detection.

In general, a square matrix is used as a mask.
For a mask of $5 \times 5$ the resulting image $res$, is obtained
from the mask $m$ and the original image $im$ applying the following
expression.


\[
res(i,j) = \frac{1}{w} \sum_{s=-2}^2 \sum_{t=-2}^2 m(s+3,t+3) \cdot
im(i+s,j+t)
\]

where the mask matrix $m$ and the weight $w$ are:

\[
m =
\begin{pmatrix}
1 & 4 & 7 & 4 & 1\\ 
4 & 16 & 26 & 16 & 4\\
7 & 26 & 41 & 26 & 7\\
4 & 16 & 26 & 16 & 4\\
1 & 4 & 7 & 4 & 1\\
\end{pmatrix}
\qquad
\qquad
w = 273
\]

\textbf{Important}: 
Note that when applying the mas in image boundaries, the image will be
considered as it was surrounded by imaginary pixels with value
\cppid{0}.

