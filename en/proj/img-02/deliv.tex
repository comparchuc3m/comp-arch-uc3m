\section{Grading}

Final grade for this project is obtained accordingly to the following criteria:

\begin{itemize}
  \item Achieved performance: 20\%.
  \item Achieved energy use: 20\%.
  \item Modified code and explanation: 15\%.
  \item Performance and energy evaluation: 20\%.
  \item Work organization: 10\%.
  \item Conclusions: 15\%.
\end{itemize}

\textgood{Warnings}:

\begin{itemize}
  \item If the submitted code does not compile, 
        the final grade for the project will be 0.

  \item If a quality coding rule is ignored without justification,
        the final grade for the project will be 0.
        
  \item In case of copying all implied groups will get a grade of 0.
        Beside the head of the school will be notified for
        the corresponding disciplinary actions.
\end{itemize}


\section{Submission procedures}

Project submission will be performed through Aula Global.
Two submission links will be enabled:

\begin{itemize}

\item \textgood{Report}. 
      It will contain the project report,
      which will be a file in PDF format named \textmark{report.pdf}.

\item \textgood{Source code}: 
      It will contain all the source code that is needed to compile.
\begin{itemize}
  \item It shall be a compressed file (ZIP format) named
        \textmark{image.zip}.
\end{itemize}

\end{itemize}

The project report shall not exceed 15 pages with a font size of 10 points
or higher, including title-page and all sections. If there are more than
15 pages, page 16 and above will not be considered during grading.
The report shall contain, at least, the following sections:

\begin{itemize}

\item \textmark{Title-page}: It shall contain the following information:
  \begin{itemize}
    \item Project name.
    \item Reduced group where students are enrolled.
    \item Team number.
    \item Name and NIA of all authors.
  \end{itemize}

\item \textmark{Parallelization}.
      It shall include an explanation of modifications performed on the
      original code.

\item \textmark{Performance and energy evaluation}: 
      It shall include performance and energy evaluations carried out by the team.
      Evaluations shall consider 1, 2, 4, 8, and 16 threads.
      They will also study the impact of different OpenMP scheduling policies.

\item \textmark{Work organization}:
      It must describe the work organization among team members, 
      making explicit the tasks that each person in the team has performed.
      It must include the time dedicated by each person to each task.
      More than one person cannot be assigned to the same task.

\item \textmark{Conclusions}.
      Special value will be given to conclusions derived from performance evaluation
      results, as well as those conclusions relating the performed work with
      contents of the course.
\end{itemize}
