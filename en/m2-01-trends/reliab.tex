\section{Reliability and availability}

\subsection{Reliability}

\begin{frame}[t]{Reliability}
\begin{itemize}
  \item The lifetime of a system represented as a random variable
        $X$.

  \item System reliability defined as function $R(t)$
\begin{displaymath}
R(t) = P(X > t) : R(0) = 1 \quad y \quad R(\infty) = 0
\end{displaymath}

  \mode<presentation>{\vfill\pause}
  \item Obtained from study of components failures.

  \mode<presentation>{\vfill\pause}
  \item \textmark{Reliability}:
        Probability that a device works properly during a given period of 
        time under specific operating conditions.
        
\end{itemize}
\end{frame}

\begin{frame}[t]{Reliability distributions}
\begin{itemize}
  \item Examples of distributions used for reliability:
    \begin{itemize}
      \mode<presentation>{\vfill}
      \item \textgood{Exponential}:
        \begin{itemize}
          \item If error rate is constant (generally true for electronic components),
                reliability follows an exponential distribution.
        \end{itemize}
      \mode<presentation>{\vfill}
      \item \textgood{Weibull}:
        \begin{itemize}
          \item Models failure distribution when failure rate is proportional to a power of time.
        \end{itemize}
    \end{itemize}
\end{itemize}
\end{frame}


\begin{frame}[t]{Serial systems}
\begin{itemize}
  \item Let $R_i(t)$ reliability for component \textmark{i}.
  \item System fails when some component fails.
\end{itemize}
\begin{center}
\input{int/m2-01-trends/rel-serial.tkz}
\end{center}
\begin{itemize}
  \item If failures are independent then:
\end{itemize}
\begin{equation*}
R(t) = \prod_{i=1}^{N} R_i(t)
\end{equation*}
\mode<presentation>{\pause}
\begin{itemize}
  \item System reliability is lower:
\end{itemize}
\begin{equation*}
R(t) < R_i(t) \forall i
\end{equation*}
\end{frame}

\begin{frame}[t]{Parallel system}
\begin{itemize}
  \item System fails when all components fail.
\end{itemize}
\begin{equation*}
R(t) = 1 - \prod_{i=1}^N Q_i(t) : Q_i(t) = 1 - R_i(t)
\end{equation*}
\begin{center}
\input{int/m2-01-trends/rel-parallel.tkz}
\end{center}
\end{frame}

\begin{frame}[t]{Example}
\begin{center}
\input{en/m2-01-trends/rel-ex-legend.tkz}
\end{center}

\begin{columns}[T]

\column{.5\textwidth}

\input{int/m2-01-trends/rel-serial-ex.tkz}

\begin{equation*}
R(t) = 0.9 \cdot 0.9 \cdot 0.9 = 0.729
\end{equation*}

\pause
\column{.5\textwidth}

\input{int/m2-01-trends/rel-parallel-ex.tkz}
\begin{equation*}
R(t) = 1 - (1 - 0.9)^3 = 0.999
\end{equation*}

\end{columns}

\end{frame}

\subsection{Availability}

\begin{frame}[t]{Availability}
\begin{itemize}
  \item In many cases, it is more interesting to know availability.
  \item Availability of a system $A(t)$ defined as the
        probability that the system is working correctly at instant $t$.
    \begin{itemize}
      \item Reliability considers interval $[0,t]$.
      \item Availability considers a concrete instant in time.
    \end{itemize}
  \item A system modeled as following state diagram.
\end{itemize}
\begin{center}

\input{en/m2-01-trends/avail.tkz}

\end{center}
\end{frame}

\begin{frame}[t]{Availability measurement}
\begin{itemize}
  \item \textgood{FIT}: Failure in time (usually for $10^9$ hours).
  \item \textgood{MTTF}: Mean Time to Failure.
  \item \textgood{MTTR}: Mean Time to Repair.
\end{itemize}

\begin{equation*}
FIT = \frac{10^9}{MTTF}
\quad\quad
A = \frac{MTTF}{MTTF + MTTR}
\end{equation*}

\begin{itemize}
  \item What does a reliability of 99\% mean?	
    \begin{itemize}
      \item In 365 days, it works correctly $\frac{99 \cdot 365}{100} = 361.35$ days.
      \item Out of service $3.65$ days.
    \end{itemize}
\end{itemize}
\end{frame}

\begin{frame}{Annual time without service}
\begin{center}
{\small
\begin{tabular}{|l|l|}
\hline
Availability (\%) & Days without service in a year\\
\hline
\hline
98\% & 7.3 days\\
\hline
99\% & 3.65 days\\
\hline
99.8\% & 17 hours y 30 minutes\\
\hline
99.9\% & 8 hours y 45 minutes\\
\hline
99.99\% & 52 minutes y 30 seconds\\
\hline
99.999\% & 5 minutes y 15 seconds\\
\hline
99.9999\% & 31.5 seconds\\
\hline
\end{tabular}
}
\end{center}
\end{frame}

\begin{frame}{Computing availability}
\begin{itemize}
  \item Elements availability
    \begin{itemize}
      \item HW: 99.99\%
      \item Disk: 99.9\%
      \item OS: 99.99\%
      \item Application: 99.9\%
      \item Communications: 99.9\%
    \end{itemize}

  \mode<presentation>{\vfill\pause}
  \item System availability:
    \begin{itemize}
      \item Product of elements availability.
    \end{itemize}
\end{itemize}
\begin{equation*}
A(t) = \prod_{i=1}^{N} A_i(t) = 99.6804 \Rightarrow 1.17 \text{days without service}
\end{equation*}
\end{frame}

