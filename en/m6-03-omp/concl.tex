\section{Conclusion}

\begin{frame}[t]{Summary}
\begin{itemize}
  \item \textgood{OpenMP} allows to annotate sequential code
        to make use of 
        \textmark{fork-join parallelism}.
    \begin{itemize}
      \item Based in the concept of parallel region.
    \end{itemize}

  \mode<presentation>{\vfill\pause}
  \item Synchronization mechanisms may be
        \textmark{high level} or 
        \textmark{low level}.

  \mode<presentation>{\vfill\pause}
  \item Parallel loops combined with reductions allow to preserve
        original code for many algorithms.

  \mode<presentation>{\vfill\pause}
  \item \textmark{Storage attributes} allow to control
        copies and data sharing in parallel regions.

  \mode<presentation>{\vfill\pause}
  \item OpenMP offers multiple scheduling approaches.
\end{itemize}
\end{frame}


\begin{frame}[t]{References}
\begin{itemize}
  \item Books:
    \begin{itemize}
      \item \textmark{The OpenMP Common Core}
            T. Mattson, Y. Hen, A.E. Koniges.
            MIT Press, 2019.
    \end{itemize}

  \mode<presentation>{\vfill}
  \item Web:
    \begin{itemize}
      \item OpenMP: \url{http://www.openmp.org}.
      \item Lawrence Livermore National Laboratory Tutorial: \url{https://computing.llnl.gov/tutorials/openMP/}.
    \end{itemize}
\end{itemize}
\end{frame}
