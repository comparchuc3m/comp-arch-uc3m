\section{Sections and scheduling}

\begin{frame}[t,fragile]{Sections}
\begin{columns}[T]

\column{.6\textwidth}
\begin{itemize}
  \item Defines a set of code sections.
  \item Each section is passed to a different thread.
  \item Implicit barrier at the end of the \cppkey{section} block.
\end{itemize}

\mode<presentation>{\vfill\pause}
\column{.4\textwidth}
\begin{block}{Example}
\begin{lstlisting}
#pragma omp parallel
{
  #pragma omp sections
  {
    #pragma omp section
    f();
    #pragma omp section
    g();
    #pragma omp section
    h();
  }
}
\end{lstlisting}
\end{block}

\end{columns}
\end{frame}

\begin{frame}[t]{Loop scheduling}
\begin{itemize}
  \item \cppkey{schedule(static)} | \cppkey{schedule(static,n)}:
    \begin{itemize}
      \item Schedules iteration blocks (size $n$) for each thread.
    \end{itemize}
  
  \mode<presentation>{\vfill\pause}
  \item \cppkey{schedule(dynamic)} | \cppkey{schedule(dynamic,n)}:
    \begin{itemize}
      \item Each thread takes a block of $n$ iterations from a queue until all have been processed.
    \end{itemize}
  
  \mode<presentation>{\vfill\pause}
  \item \cppkey{schedule(guided)} | \cppkey{schedule(guided,n)}:
    \begin{itemize}
      \item Each thread takes an iteration block until all have been processed. 
            Starts with a large block size and it is decreased until size $n$ is reached.
    \end{itemize}
  
  \mode<presentation>{\vfill\pause}
  \item \cppkey{schedule(runtime)} | \cppkey{schedule(runtime,n)}:
    \begin{itemize}
      \item Uses scheduling specified by \cppid{OMP\_SCHEDULE} or 
            the runtime library.
    \end{itemize}
  
\end{itemize}
\end{frame}
