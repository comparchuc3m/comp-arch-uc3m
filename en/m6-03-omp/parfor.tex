\section{Parallel loops}

\begin{frame}[t,fragile]{Parallel for}
\begin{itemize}
  \item \textgood{Loop work-sharing}: 
        Splits iterations from a loop among available threads.
\end{itemize}

\mode<presentation>{\vfill\pause}
\begin{columns}[T]

\column{.35\textwidth}
\begin{block}{Syntax}
\begin{lstlisting}
#pragma omp parallel
{
  #pragma omp for
  for (i=0; i<n; ++i) {
    f(i);
  }
}
\end{lstlisting}
\end{block}

\column{.65\textwidth}
\begin{itemize}
  \item \cppkey{omp for} $\rightarrow$ For loop work-sharing.
  \item A private copy of \cppid{i} is generated for each thread.
    \begin{itemize}
      \item Can also be done with \cppkey{private(}\cppid{i}\cppkey{)}
    \end{itemize}
\end{itemize}

\end{columns}

\end{frame}

\begin{frame}[t,fragile]{Example}
\begin{block}{Sequential code}
\begin{lstlisting}
for (i=0;i<max;++i) { u[i] = v[i] + w[i]; }
\end{lstlisting}
\end{block}

\mode<presentation>{\vfill\pause}
\begin{columns}[T]

\column{.5\textwidth}

\begin{block}{Parallel region}
\begin{lstlisting}[basicstyle=\tiny]
#pragma omp parallel
{
  int id = omp_get_thread_num();
  int nthreads = omp_get_num_threads();
  int istart = id * max / nthreads;
  int iend = (id==nthreads-1) ?  ((id + 1) * max / nthreads):max;
  for (int i=istart;i<iend;++i) { 
    u[i] = v[i] + w[i]; 
  }
}
\end{lstlisting}
\end{block}

\column{.5\textwidth}

\mode<presentation>{\vfill\pause}
\begin{block}{Parallel region + parallel loop}
\begin{lstlisting}
#pragma omp parallel
#pragma omp for
for (i=0;i<max;++i) { 
  u[i] = v[i] + w[i]; 
}
\end{lstlisting}
\end{block}

\end{columns}
\end{frame}

\begin{frame}[t,fragile]{Combined construct}
\begin{itemize}
  \item An \textmark{abbreviated form} can be used by combining both directives.
\end{itemize}

\begin{columns}[T]

\column{.5\textwidth}

\mode<presentation>{\vfill\pause}
\begin{block}{Two directives}
\begin{lstlisting}
vector<double> vec(max);
#pragma omp parallel
{
  #pragma omp for
  for (i=0; i<max; ++i) {
    vec[i] = generate(i);
  }
}
\end{lstlisting}
\end{block}

\column{.5\textwidth}

\mode<presentation>{\vfill\pause}
\begin{block}{Combined directive}
\begin{lstlisting}
vector<double> vec(max);
#pragma omp parallel for
for (i=0; i<max; ++i) {
  vec[i] = generate(i);
}
\end{lstlisting}
\end{block}

\end{columns}
\end{frame}

\begin{frame}[t,fragile,shrink=10]{Reductions}
\begin{columns}

\column{.4\textwidth}

\begin{block}{Example}
\begin{lstlisting}
double sum = 0.0; 
vector<double> v(max);
for (int i=0; i<max; ++i) {
  sum += v[i];
}
\end{lstlisting}
\end{block}

\column{.6\textwidth}
\begin{itemize}
  \item A \textgood{reduction} performs a reduction on the variables that appear in its list in a parallelized loop.
  \pause
  \item Reduction clause: \cppkey{reduction (op1:var1, op2:var2)}
\end{itemize}

\end{columns}

\mode<presentation>{\vfill\pause}

\begin{columns}

\column{.4\textwidth}

\begin{itemize}
  \item \textgood{Effects}:
    \begin{itemize}
      \item Private copy for each variable.
      \item Local copy updated in each iteration.
      \item Local copies combined at the end.
    \end{itemize}
\end{itemize}

\column{.6\textwidth}

\pause
\begin{block}{Example}
\begin{lstlisting}
double sum = 0.0; 
vector<double> v(max);
#pragma omp parallel for reduction(+:sum)
for (int i=0; i<max; ++i) {
  sum += v[i];
}
\end{lstlisting}
\end{block}

\end{columns}

\end{frame}

\begin{frame}[t]{Reduction operation}
\begin{itemize}
  \item Associative operations.
\[
(a \oplus b) \oplus c = a \oplus (b \oplus c)
\]

  \item Initial value defined by the operation.

  \mode<presentation>{\vfill\pause}
  \item \textgood{Basic operators}:
    \begin{itemize}
      \item \cppkey{+} (initial value: \cppid{0}).
      \item \cppkey{*} (initial value: \cppid{1}).
      \item \cppkey{-} (initial value: \cppid{0}).
    \end{itemize}

  \mode<presentation>{\vfill\pause}
  \item \textgood{Advanced operators}:
    \begin{itemize}
      \item \cppkey{\&} (initial value: \cppid{~0}).
      \item \cppkey{|} (initial value: \cppid{0}).
      \item \cppkey{\^} (initial value: \cppid{0}).
      \item \cppkey{\&\&} (initial value: \cppid{1}).
      \item \cppkey{||} (initial value: \cppid{0}).
    \end{itemize}

\end{itemize}
\end{frame}

\begin{frame}[t]{Exercise 4}
\begin{itemize}
  \item Modify the $\pi$ computation program.
  \item Transform the program for obtaining a similar version to the 
        original sequential program using a reduction.
\end{itemize}
\end{frame}
