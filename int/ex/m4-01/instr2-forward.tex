\begin{tabular}{|l|l|*{15}{>{\footnotesize}c|}}
\hline
\# &
\textbf{Instr.} &
1 & 2 & 3 & 4 & 5 &
6 & 7 & 8 & 9 & 10 &
11 & 12 & 13 & 14 & 15 
\\
\hline
\hline

I1 &
lw t4, 0(t1)
& IF & ID & EX & M & M & WB
\\
\hline

I2 &
lw t5, 0(t2)
& 
& IF & ID & EX & -- & M & M & WB
\\
\hline

I3 &
mul t5, t4, t5
& &
& IF & ID & -- & -- & -- & EX & M & WB
\\
\hline

I4 &
sw t5, 0(t3)
& & &
& IF & -- & -- & -- & ID & EX & M & M & M & WB
\\
\hline

I5 &
addi t1, t1, 4
& & & & & & &
& IF & ID & EX & -- & -- & M & WB
\\
\hline

I6 &
addi t2, t2, 4
& & & & & & & &
& IF & ID & -- & --& EX & M & WB
\\
\hline

I7 &
addi t3, t3, 4
& & & & & & & & &
& IF & -- & -- & ID & EX & M
\\
\hline

I8 &
bne t3, zero, loop
& & & & & & & & & & & &
& IF & ID & EX
\\
\hline

I9 & 
--
& & & & & & & & & & & & &
& IF & ID
\\
\hline

\end{tabular}

\vspace{1em}

\begin{tabular}{|l|l|*{5}{>{\footnotesize}c|}}
\hline
\# &
\textbf{Instr.} &
16 & 17 & 18 & 19 & 20
\\
\hline
\hline

I7 &
addi t3, t3, 4
& WB
\\
\hline

I8 &
bne t3, zero, loop
& M & WB
\\
\hline

I9 &
--
& --
\\
\hline

% en RISC-V después de un salto hay que descartar dos instrucciones

I1 &
lw t4, 0(t1)
& IF & ID & EX & M & WB   % al ser lw (acceso a memoria de lectura) debería de tener & M & M
\\
\hline

\end{tabular}
